\documentclass[14pt]{elegantbook}

\newcommand{\CN}{BIOS 7650\\[0.5cm] Stat Learning in Data Analysis}
\newcommand{\Ti}{Homework 1}
\newcommand{\Pf}{Dr. Li}
\newcommand{\FN}{Zehao}
\newcommand{\LN}{Wang}
\usepackage[fontsize=14pt]{fontsize}

\usepackage{enumitem}
\renewcommand{\chaptername}{Homework}
\begin{document}
\begin{titlepage}
	\begin{center}
		
		\includegraphics[width=0.6\textwidth]{Tulane.png}\\[1cm]

		\textsc{\Huge \CN}\\[0.5cm]
		\textsc{\large \Pf}\\[1.0cm]

		\textsc{\LARGE \Ti}\\[0.5cm]
		\textsc{\large \LN, \FN}\\
		{Master student in Statistics of Math Dept.}

		\vfill

		{\Large \emph{Update: \today}}

	\end{center}
\end{titlepage}

\tableofcontents
\setcounter{page}{0}
\chapter{}

\begin{exercise*}[5]
    What are the advantages and disadvantages of a very flexible (versus a less flexible) approach for regression or classification? Under what circumstances might a more flexible approach be preferred to a less flexible approach? When might a less flexible approach be preferred? 
\end{exercise*}

\begin{solution}
    The advantages of the flexible approach are that it can fit more complex data. But the flexible models are less interpretable. When we want the model to fit the data more accurately but don't care about the interpretability, we can use the flexible approach. When we want the model to be more interpretable, or we need to focus on some parameters in the model, we can use the less flexible approach. 
\end{solution}

\begin{exercise*}[6]
    Describe the differences between a parametric and a non-parametric statistical learning approach. What are the advantages of a parametric approach to regression or classification (as opposed to a non-parametric approach)? What are its disadvantages? 
\end{exercise*}

\begin{solution}
    The parametric approach assumes that the data is generated by a certain distribution. The non-parametric approach doesn't assume the distribution of the data. The parametric approach is more interpretable. But the non-parametric approach is more flexible. 
\end{solution}

\end{document}