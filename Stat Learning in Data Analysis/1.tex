\documentclass[12pt]{elegantbook}

\newcommand{\CN}{BIOS 7040\\[0.5cm] Statistical Inference I}
\newcommand{\Ti}{Homework 1}
\newcommand{\Pf}{Dr. Srivastav}
\newcommand{\FN}{Zehao}
\newcommand{\LN}{Wang}

\usepackage{enumitem}
\renewcommand{\chaptername}{Homework}
\begin{document}
\begin{titlepage}
	\begin{center}
		
		\includegraphics[width=0.6\textwidth]{Tulane.png}\\[1cm]

		\textsc{\Huge \CN}\\[0.5cm]
		\textsc{\large \Pf}\\[1.0cm]

		\textsc{\LARGE \Ti}\\[0.5cm]
		\textsc{\large \LN, \FN}\\
		{Master student in Statistics of Math Dept.}

		\vfill

		{\Large \emph{Update: \today}}

	\end{center}
\end{titlepage}

\tableofcontents
\setcounter{page}{0}
\chapter{}

    \setcounter{exer}{1}
    \begin{exercise}
        Verify the following identities. 
        \begin{enumerate}[(a)]
            \item $A\backslash B=A\backslash (A\cap B)=A\cap B^c$
            \item $B=(B\cap A)\cup (B\cap A^c)$
            \item $B\backslash A=B\cap A^c$
            \item $A\cup B=A\cup (B\cap A^c)$
        \end{enumerate}
    \end{exercise}

    \begin{solution}
        \begin{enumerate}[(a)]
            \item $A\backslash B \Leftrightarrow \{x: x\in A, x\notin B\}\Leftrightarrow \{x:x\in A, x\notin A\cap B\}\Leftrightarrow A\backslash (A\cap B)$. 
            
            $A\backslash B \Leftrightarrow \{x: x\in A, x\notin B\}\Leftrightarrow\{x:x\in A, x\in B^c\}\Leftrightarrow A\cap B^c$. 
            \item From distributive property, 
            \[
                (B\cap A)\cup (B\cap A^c) \Leftrightarrow B\cap (A\cup A^c)\Leftrightarrow B. 
            \]
            \item Same with (a). Just exchange $A$ and $B$. 
            \item Using (b), \[
                A\cup B\Leftrightarrow A\cup (B\cap A)\cup (B\cap A^c)\Leftrightarrow (A\cup B\cap A)\cup (B\cap A^c)\Leftrightarrow A\cup (B\cap A^c). 
            \]
        \end{enumerate}
    \end{solution}

    \begin{exercise}
        Finish the proof of Theorem 1.1.4. For any events $A$, $B$ and $C$ defined on a sample space $S$, show that
        \begin{enumerate}[(a)]
            \item $A\cup B=B\cup A$ and $A\cap B=B\cap A$. 
            \item $A\cup (B\cup C)=(A\cup B)\cup C$ and $A\cap (B\cap C)=(A\cap B)\cap C$. 
            \item $(A\cup B)^c=A^c\cap B^c$ and $(A\cap B)^c=A^c\cup B^c$. 
        \end{enumerate}
    \end{exercise}

    \begin{solution}
        \begin{enumerate}[(a)]
            \item $x\in A\cup B\Leftrightarrow x\in A$ or $x\in B\Leftrightarrow x\in B$ or $x\in A\Leftrightarrow x\in B\cup A$. 
            
            $x\in A\cap B\Leftrightarrow x\in A$ and $x\in B\Leftrightarrow x\in B$ and $x\in A\Leftrightarrow x\in B\cap A$. 
            \item $x\in A\cup (B\cup C)\Leftrightarrow x\in A$ or $x\in B\cup C\Leftrightarrow$ $x\in A$ or $x\in B$ or $x\in C\Leftrightarrow x\in A\cup B$ or $x\in C \Leftrightarrow x\in (A\cup B)\cup C$. 
            
            $x\in A\cap (B\cap C)\Leftrightarrow x\in A$ and $x\in B\cap C\Leftrightarrow$ $x\in A$ and $x\in B$ and $x\in C\Leftrightarrow x\in A\cap B$ and $x\in C \Leftrightarrow x\in (A\cap B)\cap C$. 
            \item $x\in (A\cup B)^c\Leftrightarrow A\notin A\cup B \Leftrightarrow x\notin A$ and $x\notin B\Leftrightarrow x\in A^c$ and $x\in B^c\Leftrightarrow x\in A^c\cap B^c$. 
            
            $x\in (A\cap B)^c\Leftrightarrow A\notin A\cap B \Leftrightarrow x\notin A$ or $x\notin B\Leftrightarrow x\in A^c$ or $x\in B^c\Leftrightarrow x\in A^c\cup B^c$. 
        \end{enumerate}
    \end{solution}

    \begin{exercise}
        For events $A$ and $B$, find formulas for the probabilities of the following events in terms of the quantities $P(A)$, $P(B)$ and $P(A\cap B)$. 
        \begin{enumerate}[(a)]
            \item either $A$ or $B$ or both. 
            \item either $A$ or $B$ but not both. 
            \item at least one of $A$ and $B$.
            \item at most one of $A$ and $B$.
        \end{enumerate}
    \end{exercise}

    \begin{solution}
        \begin{enumerate}[(a)]
            \item Either $A$ or $B$ or both means $A\cup B$. 
            
            $P(A\cup B)=P(A)+P(B)-P(A\cap B)$. 
            \item Either $A$ or $B$ but not both means $(A\cup B)\backslash(A\cap B)$, 
            
            $P(A\cup B)-P(A\cap B)=P(A)+P(B)-2P(A\cap B)$. 
            \item $P(A\cup B)=P(A)+P(B)-P(A\cap B)$. 
            \item $P(\neg (A\cap B))=1-P(A\cap B)$. 
        \end{enumerate}
    \end{solution}

    \setcounter{exer}{10}
    \begin{exercise}
        Let $S$ be a sample space. 
        \begin{enumerate}[(a)]
            \item Show that the collection $\mathcal{B}=\{\varnothing, S\}$ is a sigma algebra. 
            \item Let $\mathcal{B}=\{\text{all subsets of $S$, including $S$ itself}\}$. Show that $\mathcal{B}$ is a sigma algebra. 
            \item Show that the intersection of two sigma algebra is a sigma algebra.
        \end{enumerate}
    \end{exercise}

    \begin{solution}
        \begin{enumerate}[(a)]
            \item \begin{enumerate}[(i)]
                \item $\varnothing \in \mathcal{B}$, 
                \item $\varnothing^c=S\in\mathcal{B}$, $S^c=\varnothing\in\mathcal{B}$, 
                \item $\varnothing\cup S=S\in\mathcal{B}$.
            \end{enumerate}
            So, $\mathcal{B}$ is a sigma algebra. 
            \item \begin{enumerate}[(i)]
                \item $\varnothing$ is the subset of any set. So, $\varnothing\in\mathcal{B}$, 
                \item $\forall\,A\in\mathcal{B}$, $A\in S$, because $A$ must be a subset of $S$, $A^c$ is also a subset of $S$. Hence, $A^c\in\mathcal{B}$, 
                \item For $A_1, A_2, \cdots \in\mathcal{B}$, they are all subsets of $S$. So, $\cup_{i=1}^\infty A_i$ is subset of $S$, i.e., $\cup_{i=1}^\infty A_i\in\mathcal{B}$. 
            \end{enumerate}
            So, $\mathcal{B}$ is a sigma algebra. 
            \item Let $\mathcal{B}_1$, $\mathcal{B}_2$ be two different sigma algebras, 
            \begin{enumerate}[(i)]
                \item $\varnothing\in\mathcal{B}_1$ and $\varnothing\in\mathcal{B}_2\Leftrightarrow\varnothing\in\mathcal{B}_1\cap\mathcal{B}_2$,  
                \item $\forall\,A\in\mathcal{B}_1\cap\mathcal{B}_2$, $A\in\mathcal{B}_1\Leftrightarrow A^c\in\mathcal{B}_1$, $A\in\mathcal{B}_2\Leftrightarrow A^c\in\mathcal{B}_2$. So, $A^c\in\mathcal{B}_1\cap\mathcal{B}_2$, 
                \item For $A_1, A_2, \cdots \in\mathcal{B}_1\cap\mathcal{B}_2$, $\cup_{i=1}^\infty A_i\in\mathcal{B}_1$, $\cup_{i=1}^\infty A_i\in\mathcal{B}_2$, i.e., $\cup_{i=1}^\infty A_i\in\mathcal{B}_1\cap\mathcal{B}_2$. 
            \end{enumerate}
            So, $\mathcal{B}$ is a sigma algebra. 
        \end{enumerate}
    \end{solution}

    \setcounter{exer}{12}
    \begin{exercise}
        If $P(A)=1/3$ and $P(B^c)=1/4$, can $A$ and $B$ be disjoint? Explain. 
    \end{exercise}

    \begin{solution}
        If $A$ and $B$ are disjoint, then
        \[P(A\cup B)=P(A)+P(B)=\frac{1}{3}+1-\frac{1}{4}=\frac{13}{12}>1. \]
        So, we get contradiction and $A$ and $B$ can not be disjoint. 
    \end{solution}

    \setcounter{exer}{32}
    \begin{exercise}
        Suppose that $5\%$ of men and $0.25\%$ of women are color-blind. A person is chosen at random and that person is color-blind. What is the probability that the person is male? (Assume males and females to be in equal numbers.)
    \end{exercise}

    \begin{solution}
        With Bayes' rule, 
        \begin{align*}
            P(male|color\text{-}blind)&=\frac{P(color\text{-}blind|male)P(male)}{P(color\text{-}blind)}\\
            &=\frac{P(color\text{-}blind|male)P(male)}{P(color\text{-}blind|male)P(male)+P(color\text{-}blind|female)P(female)}\\
            &=\frac{0.05\times0.5}{0.05\times0.5+0.0025\times0.5}=95.24\%.
        \end{align*}
    \end{solution}

    \setcounter{exer}{34}
    \begin{exercise}
        Prove that if $P(\cdot)$ is a legitimate probability function and $B$ is a set with $P(B)>0$, then $P(\cdot|B)$ also satisfied Kolmogorov's Axioms. 
    \end{exercise}

    \begin{solution}
        $P$ is a probability function, i.e., $\forall\,A\in S, P(A)\geq 0$, and $P(S)=1$. So $P(\cdot|B)\geq0$,
        \[
            P(S|B)=\frac{P(B|S)P(S)}{P(B)}=1. 
        \]
        Then, if $A_1, A_2, \cdots$ are mutually disjoint, 
        \begin{align*}
            P\left(\bigcup_{i=1}^\infty A_i\big|B\right)=\frac{P\left(\cup_{i=1}^\infty A_i \bigcap B\right)}{P(B)}=\frac{P(\cup_{i=1}^\infty(A_i\cap B))}{P(B)}=\frac{\sum_{i=1}^\infty P(A_i|B)P(B)}{P(B)}=\sum_{i=1}^\infty P(A_i|B). 
        \end{align*}
    \end{solution}

    \setcounter{exer}{37}
    \begin{exercise}
        Prove each of the following statements. (Assume that any conditioning event has positive probability.)
        \begin{enumerate}[(a)]
            \item If $P(B)=1$, then $P(A|B)=P(A)$ for any $A$. 
            \item If $A\subset B$, then $P(B|A)=1$ and $P(A|B)=P(A)/P(B)$. 
            \item If $A$ and $B$ are mutually exclusive, then \[
                P(A|A\cup B)=\frac{P(A)}{P(A)+P(B)}. 
            \]
            \item $P(A\cap B\cap C)=P(A|B\cap C)P(B|C)P(C)$. 
        \end{enumerate}
    \end{exercise}

    \begin{solution}
        \begin{enumerate}[(a)]
            \item $\forall\,A\in S$, \begin{align*}
                P(A|B)=\frac{P(B|A)P(A)}{P(B)}=P(B\cap A), 
            \end{align*}
            Because $P(B\cap A)+P(B^c\cap A)=P(A)$, and $P(B^c\cap A)\leq P(B^c)=1-P(B)=0$. So, $P(A)=P(B\cap A)$, i.e., $P(A|B)=P(A)$. 
            \item Because $A\subset B$, $A\cap B=A$, 
            \[
                P(B|A)=\frac{P(A\cap B)}{P(A)}=\frac{P(A)}{P(A)}=1. 
            \]
            \[
                P(A|B)=\frac{P(B|A)P(A)}{P(B)}=\frac{P(A)}{P(B)}. 
            \]
            \item Using Bayes' rule,
            \[
                P(A|A\cup B)=\frac{P(A\cup B|A)P(A)}{P(A\cup B)}=\frac{P(A)}{P(A)+P(B)}. 
            \]
            \item \[
                P(A\cap B\cap C)=P(A\cap B|C)P(C)=P(A|B\cap C)P(B|C)P(C). 
            \]
        \end{enumerate}
    \end{solution}

    \begin{exercise*}[2]
        A set of $n$ items contains $k$ defective items, and $m$ are sampled randomly for inspection. How should the value of $m$ be chosen so that the probability that at least one defective item turns up is $0.90$? Apply your answer to 
        \begin{enumerate}[(a)]
            \item $n=1000$, $k=10$. 
            \item $n=10000$, $k=100$.
        \end{enumerate}
        Hint: It's ok to derive an approximate answer by assuming $k$ is small compared to $n$. 
    \end{exercise*}

    \begin{solution}
        \begin{align*}
            P(\text{at least one defective})&=1-P(\text{no defective})=1-\frac{\binom{n-k}{m}}{\binom{n}{m}}\\
            &=1-\frac{(n-k)!}{m!(n-m-k)!}\big/\frac{n!}{m!(n-m)!}\\
            &=1-\frac{(n-m)\cdots(n-m-k+1)}{n\cdots(n-k+1)}
        \end{align*}
        \begin{enumerate}[(a)]
            \item Using \emph{WolframAlpha} calculate, 
            \[\frac{(1000-m)\cdots(991-m)}{1000\cdots991}=0.1\Leftrightarrow m\approx 204.744 \]
            But if we ignore $k$ because it is much smaller than $n$, we get,
            \[
                \left(\frac{1000-m}{1000}\right)^{10}=0.1\Leftrightarrow \ln(1000-m)-\ln(1000)=0.1\ln(0.1)\Leftrightarrow m\approx 205.672
            \]
            \item \[
                \left(\frac{10000-m}{10000}\right)^{100}=0.1\Leftrightarrow \ln(10000-m)-\ln(10000)=0.01\ln(0.1)\Leftrightarrow m\approx 227.628
            \]
        \end{enumerate}
    \end{solution}

    \begin{exercise*}[3]
        A group of $60$ second graders is to be randomly assigned to two classes of $30$ each. (The random assignment is ordered by the school district to ensure against any bias.) Five of the second graders, Marcelle, Sarah, Michelle, Katy, and Camerin, are close friends. What is the probability that they will all be in the same class? What is the probability that exactly four of them? 
    \end{exercise*}

    \begin{solution}

        1, All possible outcomes are $\binom{60}{30}$, and if these 5 people are in a same class, then another class can only be chosen from the rest 55 people, and there are two different class, so we have
        \[
            P(\text{all in same class})=2\frac{\binom{55}{30}}{\binom{60}{30}}\approx 0.0523, 
        \]

        2, If only 4 people are in a same class, there are $\binom{5}{1}=5$ combinations, with the same way, we can get 
        \[
            P(\text{only 4})=2\binom{5}{1}\frac{\binom{55}{29}}{\binom{60}{30}}\approx 0.0301. 
        \]
    \end{solution}
\end{document}