\documentclass[14pt]{elegantbook}

\newcommand{\CN}{BIOS 7650\\[0.5cm] Stat Learning in Data Analysis}
\newcommand{\Ti}{Homework 2}
\newcommand{\Pf}{Dr. Li}
\newcommand{\FN}{Zehao}
\newcommand{\LN}{Wang}
\usepackage[fontsize=14pt]{fontsize}

\definecolor{LightGray}{gray}{0.9}
\usepackage{minted}

\usepackage{enumitem}
\renewcommand{\chaptername}{Homework}
\begin{document}
\begin{titlepage}
	\begin{center}
		
		\includegraphics[width=0.6\textwidth]{Tulane.png}\\[1cm]

		\textsc{\Huge \CN}\\[0.5cm]
		\textsc{\large \Pf}\\[1.0cm]

		\textsc{\LARGE \Ti}\\[0.5cm]
		\textsc{\large \LN, \FN}\\
		{Master student in Statistics of Math Dept.}

		\vfill

		{\Large \emph{Update: \today}}

	\end{center}
\end{titlepage}

\tableofcontents
\setcounter{page}{0}
\setcounter{chapter}{1}
\chapter{}

\begin{exercise*}[3.2]
  Carefully explain the differences between the KNN classifier and KNN regression methods. 
\end{exercise*}

\begin{solution}
  KNN classifier uses the majority vote of the K nearest neighbors to predict the class labels of the data. KNN regression uses the average of the K nearest neighbors to predict the continuous values of the data. 
\end{solution}

\begin{exercise*}[4.1]
  Using a little bit of algebra, prove that (4.2) is equivalent to (4.3). In other words, the logistic function representation and logit representation for the logistic regression model are equivalent. 
\end{exercise*}

\begin{solution}
  \begin{align*}
    P(X)&=\frac{e^{\beta_0+\beta_1X}}{1+e^{\beta_0+\beta_1X}}\\
    &=1-\frac{1}{1+e^{\beta_0+\beta_1X}}\\
    &\Rightarrow\frac{1}{1+e^{\beta_0+\beta_1X}}=1-P(X)\\
    &\Rightarrow e^{\beta_0+\beta_1X}=\frac{1}{1-P(X)}-1=\frac{P(X)}{1-P(X)}. 
  \end{align*}
\end{solution}

\begin{exercise*}[4.4]
  When the number of features $p$ is large, there tends to be a deterioration in the performance of KNN and other local approaches that perform prediction using only observations that are near the test observation for which a prediction must be made. This phenomenon is known as the curse of dimensionality, and it ties into the fact that curse of non-parametric approaches often perform poorly when $p$ is large. We will now investigate this curse. 
  \begin{enumerate}[(a)]
    \item Suppose that we have a set of observations, each with measurements on $p = 1$ feature, $X$. We assume that $X$ is uniformly (evenly) distributed on $[0, 1]$. Associated with each observation is a response value. Suppose that we wish to predict a test observation's response using only observations that are within $10\%$ of the range of $X$ closest to that test observation. For instance, in order to predict the response for a test observation with $X = 0.6$, we will use observations in the range $[0.55,0.65]$. On average, what fraction of the available observations will we use to make the prediction? 
    \item Now suppose that we have a set of observations, each with measurements on p = 2 features, X1 and X2. We assume that $(X1, X2)$ are uniformly distributed on $[0, 1] \times [0, 1]$. We wish to predict a test observation's response using only observations that are within $10 \%$ of the range of $X1$ and within $10 \% $ of the range of $X2$ closest to that test observation. For instance, in order to predict the response for a test observation with $X1 = 0.6$ and $X2 = 0.35$, we will use observations in the range $[0.55, 0.65]$ for $X1$ and in the range $[0.3,0.4]$ for $X2$. On average, what fraction of the available observations will we use to make the prediction? 
    \item Now suppose that we have a set of observations on $p = 100$ features. Again the observations are uniformly distributed on each feature, and again each feature ranges in value from $0$ to $1$. We wish to predict a test observation's response using observations within the $10 \%$ of each feature's range that is closest to that test observation. What fraction of the available observations will we use to make the prediction? 
    \item Using your answers to parts (a)-(c), argue that a drawback of KNN when p is large is that there are very few training observations “near” any given test observation.
    \item Now suppose that we wish to make a prediction for a test observation by creating a p-dimensional hypercube centered around the test observation that contains, on average, $10 \%$ of the training observations. For $p = 1,2$, and $100$, what is the length of each side of the hypercube? Comment on your answer. 
    
    Note: A hypercube is a generalization of a cube to an arbitrary number of dimensions. When $p = 1$, a hypercube is simply a line segment, when $p = 2$ it is a square, and when $p = 100$ it is a 100-dimensional cube. 
  \end{enumerate}
\end{exercise*}

\begin{solution}
  \begin{enumerate}[(a)]
    \item \[2\int_0^{0.05}\frac{x+0.05}{1}dx+0.9\times 0.1=0.0975.\]
    \item $(9.75\%)^2$. 
    \item $(9.75\%)^{100}$. 
    \item If $p=100$, the number of training observations is almost zero. So, even the total observations are large, the number of training observations near the test observation is small. 
    \item The Length is $0.1$, $\sqrt{0.1}$, $\sqrt[100]{0.1}$ when $p=1,2,100$. 
  \end{enumerate}
\end{solution}

\begin{exercise*}[4.5]
  We now examine the differences between LDA and QDA. 
  \begin{enumerate}[(a)]
    \item If the Bayes decision boundary is linear, do we expect LDA or QDA to perform better on the training set? On the test set? 
    \item If the Bayes decision boundary is non-linear, do we expect LDA or QDA to perform better on the training set? On the test set?
    \item In general, as the sample size $n$ increases, do we expect the test prediction accuracy of QDA relative to LDA to improve, decline, or be unchanged? Why?
    \item True or False: Even if the Bayes decision boundary for a given problem is linear, we will probably achieve a superior test error rate using QDA rather than LDA because QDA is flexible enough to model a linear decision boundary. Justify your answer. 
  \end{enumerate}
\end{exercise*}

\begin{solution}
  \begin{enumerate}[(a)]
    \item LDA performs better on the test set and QDA performs better on the training set. Because QDA will over-fit the training data.
    \item When the Bayes decision boundary is non-linear, QDA performs better on the training set and test set.
    \item If the Bayes decision boundary is linear, the accuracy of LDA relative to QDA will improve with $n$ increasing. If the Bayes decision boundary is non-linear, the accuracy of QDA will improve. 
    \item False. For a linear decision boundary, LDA will perform better than QDA. 
  \end{enumerate}
\end{solution}

\begin{exercise*}
  In this question, we will work on a number of lectured methods and look into some related issues. Examples include the effects of the number of neighbors in KNN; differences/similarity among KNN, logistic regression, and LDA/QDA; and practice on shrinkage methods. 

  For this question, the data file has been attached as “HW1data.txt”. It is actually the file for DNA microarray expression example mentioned in ESL (Example 4 in Chapter 1). The file is a text file: the first column is cancer name (string type data), the second column for group $(0/1)$, and the remaining columns for different genes. In this question, we will only use the group information as our $Y$, and not worrying about the particular cancer types. All the gene expression levels will be used as potential predictors. 
  \begin{enumerate}
    \item With one predictor each time, use logistic regression to find the ten most significant genes
    \item Use the ten genes from $1$ to conduct KNN with $K=15$. Compare the training error with what you had obtained in Homework 1
    \item Conduct logistic regression and find out its training error
    \item Conduct LDA analyses and find out its training error
    \item Compare the training error rates from KNN, logistic regression, LDA. 
  \end{enumerate}
\end{exercise*}

\begin{solution}
  \begin{enumerate}
    \item \begin{minted}[frame=lines,
      framesep=2mm,
      baselinestretch=1.2,
      bgcolor=LightGray,
      fontsize=\footnotesize]{R}
library("tidyverse")
library("class")

set.seed(0927)

gene_data <-
  read_table(
    "/Users/zehao/Documents/GitHub
    /MS-Stat-Tulane/Stat Learning in Data Analysis
    /HW02/HW1data.txt",
    col_names = FALSE
  )
# X2 is Y

p <- c()

Y <- pull(gene_data, 2)

for (i in 3:6832) {
  X <- pull(gene_data, i)
  single_p <-
    summary(
      glm(Y ~ X, family = binomial(link = "logit"))
      )$coefficients[2, 4]
  p = append(p, single_p)
}

gene_10 <- gene_data[, (order(p)[1:10] + 2)]
    \end{minted}
    \item \begin{minted}[frame=lines,
      framesep=2mm,
      baselinestretch=1.2,
      bgcolor=LightGray,
      fontsize=\footnotesize]{R}
# Use all data as train data

train_data <- gene_10

train_data_label <- Y

knn_15 <-
  as.integer(as.character(
    knn(
      train = train_data,
      test = train_data,
      cl = train_data_label,
      k = 15
    )
  ))

Err_15 <- mean(abs(knn_15 - train_data_label))
    \end{minted}
    The error rate is $0.046875$, which is much smaller than the error rate in Homework 1 that the 10 gene is randomly selected. 
    \item \begin{minted}[frame=lines,
      framesep=2mm,
      baselinestretch=1.2,
      bgcolor=LightGray,
      fontsize=\footnotesize]{R}
logi <-
  glm(Y ~ .,
      family = binomial(link = "logit"),
      data = cbind(Y, gene_10))

predict_result <-
  round(predict.glm(logi, gene_10, type = "response"), 2)

mean(abs(Y - predict_result))
    \end{minted}
    The error rate for training is $0$. 
    \item \begin{minted}[frame=lines,
      framesep=2mm,
      baselinestretch=1.2,
      bgcolor=LightGray,
      fontsize=\footnotesize]{R}
library(MASS)

lda_result <-
  lda(Y ~ ., data = cbind(Y, gene_10))
lda_result <-
  predict(lda_result, gene_10)$class

mean(abs(Y - (as.numeric(lda_result) - 1)))
    \end{minted}
    The LDA error rate for training is $0.03125$.
    \item The error rate for KNN is $0.046875$, the error rate for logistic regression is $0$, and the error rate for LDA is $0.03125$. The error rate for logistic regression is the smallest, and the error rate for KNN is the largest.
  \end{enumerate}
\end{solution}

\end{document}