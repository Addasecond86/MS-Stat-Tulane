\documentclass[14pt]{elegantbook}

\newcommand{\CN}{BIOS 7040\\[0.5cm] Statistical Inference I}
\newcommand{\Ti}{Homework 4}
\newcommand{\Pf}{Dr. Srivastav}
\newcommand{\FN}{Zehao}
\newcommand{\LN}{Wang}
\usepackage[fontsize=14pt]{fontsize}

\usepackage{enumitem}
\renewcommand{\chaptername}{Homework}
\begin{document}
\begin{titlepage}
	\begin{center}
		
		\includegraphics[width=0.6\textwidth]{Tulane.png}\\[1cm]

		\textsc{\Huge \CN}\\[0.5cm]
		\textsc{\large \Pf}\\[1.0cm]

		\textsc{\LARGE \Ti}\\[0.5cm]
		\textsc{\large \LN, \FN}\\
		{Master student in Statistics of Math Dept.}

		\vfill

		{\Large \emph{Update: \today}}

	\end{center}
\end{titlepage}

\tableofcontents
\setcounter{page}{0}
\setcounter{chapter}{3}
\chapter{}
    \setcounter{chapter}{2}
    

    \setcounter{exer}{0}
    \begin{exercise}
        In each of the following find the p.d.f. of $Y$. Show that the p.d.f. integrates to $1$.
        \begin{enumerate}[(a)]
            \item $Y=X^3$, and $f_X(x)=42x^5(1-x)$, $0<x<1$. 
            \item $Y=4X+3$ and $f_X(x)=7e^{-7x}$, $0<x<\infty$. 
            \item $Y=X^2$ and $f_X(x)=30x^2(1-x)^2$, $0<x<1$.
        \end{enumerate}
    \end{exercise}

    \begin{solution}
        \begin{enumerate}[(a)]
            \item $Y=g(X)=X^3$, $X\in(0,1)$, $Y\in(0,1)$. $X=g^{-1}(Y)=Y^{1/3}$. And $g(X)$ is monotone on $(0, 1)$, so, 
            \begin{align*}
                f_Y(y)&=f_X(g^{-1}(y))|(g^{-1}(y))'|\\
                &=42y^{5/3}\left(1-y^{1/3}\right)|1/3y^{-2/3}|\\
                &=14y^{5/3}\left(1-y^{1/3}\right)|y^{-2/3}|\\
                &=14y\left(1-y^{1/3}\right)\\
                &=14y-14y^{4/3}. 
            \end{align*}
            \[\int_0^1f_Y(y)d y=(7y^2-6y^{7/3})\big|_{0}^1=7-6-0=1. \]
            \item $Y=g(X)=4X+3$, $X\in(0,\infty)$, $Y\in(3,\infty)$. $X=g^{-1}(Y)=Y/4-3/4$. And $g(X)$ is monotone on $(0, \infty)$, so, 
            \begin{align*}
                f_Y(y)&=f_X(g^{-1}(y))|(g^{-1}(y))'|\\
                &=7e^{-7(y-3)/4}\frac{1}{4}\\
                &=\frac{7}{4}e^{-7y/4+21/4}. 
            \end{align*}
            \[\int_3^\infty f_Y(y)d y=(-e^{-7y/4+21/4})\big|_{3}^\infty=0+1=1. \]
            \item $Y=g(X)=X^2$, $X\in(0,1)$, $Y\in(0,1)$. $X=g^{-1}(Y)=Y^{1/2}$. And $g(X)$ is monotone on $(0, 1)$, so, 
            \begin{align*}
                f_Y(y)&=f_X(g^{-1}(y))|(g^{-1}(y))'|\\
                &=30y(1-y^{1/2})^2\frac{1}{2}y^{-1/2}\\
                &=15y^{1/2}(1-2y^{1/2}+y)\\
                &=15(y^{1/2}-2y+y^{3/2}).
            \end{align*}
            \[\int_0^1 f_Y(y)d y=\left.15\left(\frac{2}{3}y^{3/2}-y^2+\frac{2}{5}y^{5/2}\right)\right|_{0}^1=15\cdot\frac{1}{15}-0=1. \]
        \end{enumerate}
    \end{solution}

    \setcounter{exer}{3}
    \begin{exercise}
        Let $\lambda$ be a fixed positive constant, and define the function $f(x)$ by $f(x)=\frac{1}{2}\lambda e^{-\lambda x}$ if $x\geq 0$ and $f(x)=\frac{1}{2}\lambda e^{\lambda x}$ if $x<0$. 
        \begin{enumerate}[(a)]
            \item Verify that $f(x)$ is a p.d.f. 
            \item If $X$ is a random variable with p.d.f. given by $f(x)$, find $P(X<t)$ for all $t$. Evaluate all integrals. 
            \item Find $P(|X|<t)$ for all $t$. Evaluate all integrals. 
        \end{enumerate}
    \end{exercise}

    \begin{solution}
        \begin{enumerate}[(a)]
            \item $\lambda$ is positive, so $f(x)\geq 0$ for all $x$. And for $x\geq 0$, 
            \[\int_{0}^\infty\frac{1}{2}\lambda e^{-\lambda x} dx=\left.-\frac{1}{2}e^{-\lambda x}\right|_0^\infty=\frac{1}{2}. \]
            For $x< 0$, 
            \[\int_{-\infty}^0\frac{1}{2}\lambda e^{\lambda x} dx=\left.\frac{1}{2}e^{\lambda x}\right|_{-\infty}^0=\frac{1}{2}. \]
            So, $\int_{-\infty}^\infty f(x) dx=1/2+1/2=1$. Hence, $f(x)$ is p.d.f. 
            \item \[P(X<t)=\int_{-\infty}^t f(x) dx, \]
            When $t<0$, 
            \[
                P(X<t)=\int_{-\infty}^t \frac{1}{2}\lambda e^{\lambda x} dx = \left.\frac{1}{2}e^{\lambda x}\right|_{-\infty}^t=\frac{1}{2}e^{\lambda t}.
            \]
            And When $t\geq 0$, 
            \[
                P(X<t)=\int_{-\infty}^t \frac{1}{2}\lambda e^{-\lambda x} dx = \left.-\frac{1}{2}e^{-\lambda x}\right|_{0}^t+\frac{1}{2}=1-\frac{1}{2}e^{-\lambda t}.
            \]
            \item \begin{align*}
                P(|X|<t)&=P(-t<X<t)\\
                &=\int_{-t}^0\frac{1}{2}\lambda e^{\lambda x} dx+\int_0^t\frac{1}{2}\lambda e^{-\lambda x} dx\\
                &=\left.\frac{1}{2}e^{\lambda x}\right|_{-t}^0+\left(\left.-\frac{1}{2}e^{-\lambda x}\right|_0^t\right)\\
                &=\frac{1}{2}-\frac{1}{2}e^{-\lambda t}-\frac{1}{2}e^{-\lambda t}+\frac{1}{2}\\
                &=1-e^{-\lambda t}.
            \end{align*}
        \end{enumerate}
    \end{solution}

    \setcounter{exer}{6}
    \begin{exercise}
        Let $X$ have p.d.f. $f_X(x)=\frac{2}{9}(x+1)$, $-1\leq x\leq 2$. 
        \begin{enumerate}[(a)]
            \item Find the p.d.f. of $Y=X^2$. Note that Theorem 2.1.8 is not directly applicable in this problem. 
            \item Show that Theorem 2.1.8 remains valid if the sets $A_0, A_1, \cdots, A_k$ contain $\mathcal{X}$, and apply the extension to solve part (a) using $A_0=\varnothing, A_1=(-2,0)$, and $A_2=(0,2)$. 
        \end{enumerate}
    \end{exercise}

    \begin{solution}
        \begin{enumerate}[(a)]
            \item Because the third condition of Theorem 2.1.8 is not satisfied, we cannot apply the theorem directly. 
            Using Example 2.1.7, we know the p.d.f. of $\ Y$ on $[-1,1]$ is: 
            \[f_Y(y)=\frac{1}{2\sqrt{y}}(f_X(\sqrt{y})+f_X(-\sqrt{y}))=\frac{2}{9\sqrt{y}},\quad y\in (0, 1]. \]
            And for $x\in (1,2]$: 
            \[F_Y(y)=P(Y<y)=P(1<X\leq y^{1/2})=\int_{1}^{y^{1/2}}f_X(x)dx\]
            \[f_Y(y)=F_Y'(y)=\frac{1}{9}y^{1/2}+\frac{1}{9}, \quad y\in(1,4]. \]
            \item 
        \end{enumerate}
    \end{solution}

    \begin{exercise}
        
    \end{exercise}

    \begin{solution}
        
    \end{solution}

    \setcounter{exer}{10}
    \begin{exercise}
        
    \end{exercise}

    \begin{solution}
        
    \end{solution}

    \setcounter{exer}{22}
    \begin{exercise}
        
    \end{exercise}

    \begin{solution}
        
    \end{solution}

    \setcounter{exer}{24}
    \begin{exercise}
        
    \end{exercise}

    \begin{solution}
        
    \end{solution}

    \setcounter{exer}{31}
    \begin{exercise}
        
    \end{exercise}

    \begin{solution}
        
    \end{solution}

    \begin{exercise}
        
    \end{exercise}

    \begin{solution}
        
    \end{solution}
\end{document}