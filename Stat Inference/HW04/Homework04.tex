\documentclass[14pt]{elegantbook}

\newcommand{\CN}{BIOS 7040\\[0.5cm] Statistical Inference I}
\newcommand{\Ti}{Homework 4}
\newcommand{\Pf}{Dr. Srivastav}
\newcommand{\FN}{Zehao}
\newcommand{\LN}{Wang}
\usepackage[fontsize=14pt]{fontsize}

\usepackage{enumitem}
\renewcommand{\chaptername}{Homework}
\begin{document}
\begin{titlepage}
	\begin{center}
		
		\includegraphics[width=0.6\textwidth]{Tulane.png}\\[1cm]

		\textsc{\Huge \CN}\\[0.5cm]
		\textsc{\large \Pf}\\[1.0cm]

		\textsc{\LARGE \Ti}\\[0.5cm]
		\textsc{\large \LN, \FN}\\
		{Master student in Statistics of Math Dept.}

		\vfill

		{\Large \emph{Update: \today}}

	\end{center}
\end{titlepage}

\tableofcontents
\setcounter{page}{0}
\setcounter{chapter}{3}
\chapter{}
    \setcounter{chapter}{2}
    

    \setcounter{exer}{0}
    \begin{exercise}
        In each of the following find the p.d.f. of $Y$. Show that the p.d.f. integrates to $1$.
        \begin{enumerate}[(a)]
            \item $Y=X^3$, and $f_X(x)=42x^5(1-x)$, $0<x<1$. 
            \item $Y=4X+3$ and $f_X(x)=7e^{-7x}$, $0<x<\infty$. 
            \item $Y=X^2$ and $f_X(x)=30x^2(1-x)^2$, $0<x<1$.
        \end{enumerate}
    \end{exercise}

    \begin{solution}
        \begin{enumerate}[(a)]
            \item $Y=g(X)=X^3$, $X\in(0,1)$, $Y\in(0,1)$. $X=g^{-1}(Y)=Y^{1/3}$. And $g(X)$ is monotone on $(0, 1)$, so, 
            \begin{align*}
                f_Y(y)&=f_X(g^{-1}(y))|(g^{-1}(y))'|\\
                &=42y^{5/3}\left(1-y^{1/3}\right)|1/3y^{-2/3}|\\
                &=14y^{5/3}\left(1-y^{1/3}\right)|y^{-2/3}|\\
                &=14y\left(1-y^{1/3}\right)\\
                &=14y-14y^{4/3}. 
            \end{align*}
            \[\int_0^1f_Y(y)d y=(7y^2-6y^{7/3})\big|_{0}^1=7-6-0=1. \]
            \item $Y=g(X)=4X+3$, $X\in(0,\infty)$, $Y\in(3,\infty)$. $X=g^{-1}(Y)=Y/4-3/4$. And $g(X)$ is monotone on $(0, \infty)$, so, 
            \begin{align*}
                f_Y(y)&=f_X(g^{-1}(y))|(g^{-1}(y))'|\\
                &=7e^{-7(y-3)/4}\frac{1}{4}\\
                &=\frac{7}{4}e^{-7y/4+21/4}. 
            \end{align*}
            \[\int_3^\infty f_Y(y)d y=(-e^{-7y/4+21/4})\big|_{3}^\infty=0+1=1. \]
            \item $Y=g(X)=X^2$, $X\in(0,1)$, $Y\in(0,1)$. $X=g^{-1}(Y)=Y^{1/2}$. And $g(X)$ is monotone on $(0, 1)$, so, 
            \begin{align*}
                f_Y(y)&=f_X(g^{-1}(y))|(g^{-1}(y))'|\\
                &=30y(1-y^{1/2})^2\frac{1}{2}y^{-1/2}\\
                &=15y^{1/2}(1-2y^{1/2}+y)\\
                &=15(y^{1/2}-2y+y^{3/2}).
            \end{align*}
            \[\int_0^1 f_Y(y)d y=\left.15\left(\frac{2}{3}y^{3/2}-y^2+\frac{2}{5}y^{5/2}\right)\right|_{0}^1=15\cdot\frac{1}{15}-0=1. \]
        \end{enumerate}
    \end{solution}

    \setcounter{exer}{3}
    \begin{exercise}
        Let $\lambda$ be a fixed positive constant, and define the function $f(x)$ by $f(x)=\frac{1}{2}\lambda e^{-\lambda x}$ if $x\geq 0$ and $f(x)=\frac{1}{2}\lambda e^{\lambda x}$ if $x<0$. 
        \begin{enumerate}[(a)]
            \item Verify that $f(x)$ is a p.d.f. 
            \item If $X$ is a random variable with p.d.f. given by $f(x)$, find $P(X<t)$ for all $t$. Evaluate all integrals. 
            \item Find $P(|X|<t)$ for all $t$. Evaluate all integrals. 
        \end{enumerate}
    \end{exercise}

    \begin{solution}
        \begin{enumerate}[(a)]
            \item $\lambda$ is positive, so $f(x)\geq 0$ for all $x$. And for $x\geq 0$, 
            \[\int_{0}^\infty\frac{1}{2}\lambda e^{-\lambda x} dx=\left.-\frac{1}{2}e^{-\lambda x}\right|_0^\infty=\frac{1}{2}. \]
            For $x< 0$, 
            \[\int_{-\infty}^0\frac{1}{2}\lambda e^{\lambda x} dx=\left.\frac{1}{2}e^{\lambda x}\right|_{-\infty}^0=\frac{1}{2}. \]
            So, $\int_{-\infty}^\infty f(x) dx=1/2+1/2=1$. Hence, $f(x)$ is p.d.f. 
            \item \[P(X<t)=\int_{-\infty}^t f(x) dx, \]
            When $t<0$, 
            \[
                P(X<t)=\int_{-\infty}^t \frac{1}{2}\lambda e^{\lambda x} dx = \left.\frac{1}{2}e^{\lambda x}\right|_{-\infty}^t=\frac{1}{2}e^{\lambda t}.
            \]
            And when $t\geq 0$, 
            \begin{align*}
                P(X<t)&=\int_{-\infty}^t f(x) dx\\
                &=\int_{-\infty}^0 \frac{1}{2}\lambda e^{\lambda x} dx+\int_0^t \frac{1}{2}\lambda e^{-\lambda x} dx\\
                &=\left.\frac{1}{2}e^{\lambda x}\right|_{-\infty}^0+\left.\left(-\frac{1}{2}e^{-\lambda x}\right)\right|_0^t\\
                &=\frac{1}{2}-\frac{1}{2}e^{-\lambda t}+\frac{1}{2}\\
                &=1-\frac{1}{2}e^{-\lambda t}.
            \end{align*}
            \item \begin{align*}
                P(|X|<t)&=P(-t<X<t)\\
                &=\int_{-t}^0\frac{1}{2}\lambda e^{\lambda x} dx+\int_0^t\frac{1}{2}\lambda e^{-\lambda x} dx\\
                &=\left.\frac{1}{2}e^{\lambda x}\right|_{-t}^0+\left(\left.-\frac{1}{2}e^{-\lambda x}\right|_0^t\right)\\
                &=\frac{1}{2}-\frac{1}{2}e^{-\lambda t}-\frac{1}{2}e^{-\lambda t}+\frac{1}{2}\\
                &=1-e^{-\lambda t}.
            \end{align*}
        \end{enumerate}
    \end{solution}

    \setcounter{exer}{6}
    \begin{exercise}
        Let $X$ have p.d.f. $f_X(x)=\frac{2}{9}(x+1)$, $-1\leq x\leq 2$. 
        \begin{enumerate}[(a)]
            \item Find the p.d.f. of $Y=X^2$. Note that Theorem 2.1.8 is not directly applicable in this problem. 
            \item Show that Theorem 2.1.8 remains valid if the sets $A_0, A_1, \cdots, A_k$ contain $\mathcal{X}$, and apply the extension to solve part (a) using $A_0=\varnothing, A_1=(-2,0)$, and $A_2=(0,2)$. 
        \end{enumerate}
    \end{exercise}

    \begin{solution}
        \begin{enumerate}[(a)]
            \item Because the third condition of Theorem 2.1.8 is not satisfied, we cannot apply the theorem directly. 
            Using Example 2.1.7, we know the p.d.f. of $\ X$ on $[-1,1]$ is: 
            \[P(Y<y)=P(X^2<y)=P(-\sqrt{y}<X<\sqrt{y})=F_X(\sqrt{y})-F_X(-\sqrt{y})\]
            \[f_Y(y)=\frac{1}{2\sqrt{y}}(f_X(\sqrt{y})+f_X(-\sqrt{y}))=\frac{1}{9\sqrt{y}}(1+\sqrt{y}+1-\sqrt{y})=\frac{2}{9\sqrt{y}},y\in (0, 1]. \]
            And for $x\in (1,2]$: 
            \begin{align*}
                F_Y(y)=P(Y<y)&=P(1<X\leq y^{1/2})=\int_{1}^{y^{1/2}}f_X(x)dx\\
                &=\frac{1}{9}(x+1)^2|_1^{y^{1/2}}=\frac{1}{9}(y^{1/2}+1)^2-\frac{4}{9}
            \end{align*}
            \[f_Y(y)=F_Y'(y)=\frac{2}{9}(y^{1/2}+1)\frac{1}{2}y^{-1/2}=\frac{1}{9}+\frac{1}{9}y^{-1/2}, \quad y\in(1,4]. \]
            \item 
            %\textcolor{red}{I found two version of this problem, the other one is: $A_1=(-1,1)$, $A_2=(1,2)$. But I think the first one is more reasonable, because $A_1, A_2$ contain $\mathcal{X}$. }
            We define $f_X(x)$ as following: 
            \[
                f_X(x)=\frac{2}{9}(x+1), \quad x\in [-1, 2], 
            \]
            \[
                f_X(x)=0, \quad x\notin [-1, 2].
            \]
            Now, $Y=g(X)=X^2$, $g(A_1)\in(0,4)$, $g(A_2)=(0,4)$. The condition iii is satisfied. So, we use Theorem 2.1.8: 
            \begin{align*}
                y\in(0,1]: \quad f_Y(y)&=f_X(g_1^{-1}(y))\left|\left(g_1^{-1}(y)\right)'\right|+f_X(g_2^{-1}(y))\left|\left(g_2^{-1}(y)\right)'\right|\\
                &=\frac{2}{9}(1-\sqrt{y})\frac{1}{2}y^{-1/2}+\frac{2}{9}(1+\sqrt{y})\frac{1}{2}y^{-1/2}\\
                &=\frac{2}{9\sqrt{y}}.\\
                y\in(1,4): \quad f_Y(y)&=f_X(g_1^{-1}(y))\left|\left(g_1^{-1}(y)\right)'\right|+f_X(g_2^{-1}(y))\left|\left(g_2^{-1}(y)\right)'\right|\\
                &=0+\frac{2}{9}(1+\sqrt{y})\frac{1}{2}y^{-1/2}\\
                &=\frac{1}{9}+\frac{1}{9}y^{-1/2}.
            \end{align*}
            Same as part (a). 
        \end{enumerate}
    \end{solution}

    \begin{exercise}
        In each of the following show that the given function is a c.d.f. and find $F_X^{-1}(y)$. \begin{enumerate}[(a)]
            \item \[F_X(x)=\left\{\begin{matrix}
                0&\text{if }x<0\\
                1-e^{-x}&\text{if }x\geq 0. 
            \end{matrix}\right.\]
            \item \[F_X(x)=\left\{\begin{matrix}
                e^x/2&\text{if }x<0\\
                1/2&\text{if }0\leq x<1\\
                1-(e^{1-x}/2)&\text{if }1\leq x.)
            \end{matrix}\right.\]
            \item \[F_X(x)=\left\{\begin{matrix}
                e^x/4&\text{if }x<0\\
                1-e^{-x}/4&\text{if }x\geq 0. 
            \end{matrix}\right.\]
        \end{enumerate}
        Note that, in part (c), $F_X(x)$ is discontinuous but (2.1.13) is still the appropriate definition of $F_X^{-1}(y)$. 
    \end{exercise}

    \begin{solution}
        \begin{enumerate}[(a)]
            \item \begin{enumerate}[i)]
                \item $\lim_{x\to-\infty}F_X(x)=0$, and $\lim_{x\to\infty}F_X(x)=1-e^{-\infty}=1$. 
                \item $F_X'(x)=-(-e^{-x})=e^{-x}>0$. So, $F_X(x)$ is increasing. 
                \item $\lim_{x\downarrow 0}=1-1=0=F_X(0)$. Hence, $F_X(x)$ is right continuous. 
              \end{enumerate}
              So, $F_X(x)$ is a c.d.f. 
              \begin{align*}
                y&=1-e^{-x}\\
                -x&=\ln(1-y)\\
                x&=-\ln(1-y)\Rightarrow F_X^{-1}(y)=-\ln(1-y), y>0. 
              \end{align*}
            \item \begin{enumerate}[i)]
                \item $\lim_{x\to-\infty}F_X(x)=0/2=0$, and $\lim_{x\to\infty}F_X(x)=1-e^{-\infty}/2=1-0=1$. 
                \item $x<0$, $F_X'(x)=e^x/2>0$; $0\leq x<1$, $F_X'(x)=0$; $x\geq 1$, $F_X'(x)=e^{1-x}/2>0$. So, $F_X(x)$ is nondecreasing. 
                \item $\lim_{x\downarrow 0}=1/2=F_X(0)$, $\lim_{x\downarrow 1}=1-1/2=1/2=F_X(1)$. Hence, $F_X(x)$ is right continuous. 
              \end{enumerate}
              So, $F_X(x)$ is a c.d.f. 
              \begin{align*}
                y&=e^x/2\\
                x&=\ln(2y)\Rightarrow F_X^{-1}(y)=\ln(2y), y\in(0,1/2). \\
                y&=1-(e^{1-x}/2)\\
                e^{1-x}&=2(1-y)\\
                1-x&=\ln(2(1-y))\\
                x&=1-\ln(2(1-y))\Rightarrow F_X^{-1}(y)=1-\ln(2(1-y)), y\in(1/2,1).
              \end{align*}
            \item \begin{enumerate}[i)]
                \item $\lim_{x\to-\infty}F_X(x)=e^{-\infty}/4=0$, and $\lim_{x\to\infty}F_X(x)=1-e^{-\infty}/4=1-0=1$. 
                \item $x<0$, $F_X'(x)=e^x/4>0$; $ x\geq 0$, $F_X'(x)=e^{-x}/4>0$. So, $F_X(x)$ is increasing. 
                \item $\lim_{x\downarrow 0}=1-1/4=3/4=F_X(0)$. Hence, $F_X(x)$ is right continuous. 
              \end{enumerate}
              So, $F_X(x)$ is a c.d.f. 
              \begin{align*}
                y&=e^x/4\\
                x&=\ln(4y)\Rightarrow F_X^{-1}(y)=\ln(4y), y\in(0,1/4). \\
                y&=1-(e^{-x}/4)\\
                e^{-x}&=4(1-y)\\
                x&=-\ln(4(1-y))\Rightarrow F_X^{-1}(y)=-\ln(4(1-y)), y\in[3/4,1).
              \end{align*}
        \end{enumerate}
    \end{solution}

    \setcounter{exer}{10}
    \begin{exercise}
        Let $X$ have the standard normal p.d.f., $f_X(x)=(1/\sqrt{2\pi})e^{-x^2/2}$. 
        \begin{enumerate}[(a)]
            \item Find $EX^2$ directly, and then by using the p.d.f. of $Y=X^2$ from Example 2.1.7 and calculating $EY$. 
            \item Find the p.d.f. of $Y=|X|$, and find its mean and variance. 
        \end{enumerate}
    \end{exercise}

    \begin{solution}
        \begin{enumerate}[(a)]
            \item \begin{align*}
                EX^2&=\int_{-\infty}^\infty x^2f_X(x)dx=\int_{-\infty}^\infty x^2(1/\sqrt{2\pi})e^{-x^2/2}dx\\
                &=\frac{1}{\sqrt{2\pi}}\int_{-\infty}^\infty x^2e^{-x^2/2}dx\\
                &=\frac{1}{\sqrt{2\pi}}\int_{-\infty}^\infty-xd(e^{-x^2/2})dx\\
                &=\frac{1}{\sqrt{2\pi}}\left(\left.-xe^{-x^2/2}\right|_{-\infty}^\infty+\int_{-\infty}^\infty e^{-x^2/2}dx\right)\\
                &=\frac{1}{\sqrt{2\pi}}\int_{-\infty}^\infty e^{-x^2/2}dx=1. 
            \end{align*}
            From 2.1.7, we know 
            \[f_Y(y)=\frac{1}{2\sqrt{y}}(f_X(\sqrt{y})+f_X(-\sqrt{y}))=\frac{1}{2\sqrt{y}}\frac{2}{\sqrt{2\pi}}e^{-y/2}=\frac{1}{\sqrt{2\pi}}\frac{e^{-y/2}}{\sqrt{y}}. \]
            \[EY=\int_{0}^\infty yf_Y(y)dy=\frac{1}{\sqrt{2\pi}}\int_0^\infty \sqrt{y}e^{-y/2}dy, \]
            let $u=\sqrt{y}$, $y=u^2$, $dy=2udu$. Then 
            \[EY=\frac{1}{\sqrt{2\pi}}\int_0^\infty ue^{-u^2/2}\cdot 2u du=2\frac{1}{\sqrt{2\pi}}\int_0^\infty u^2e^{-u^2/2}du\]
            \[=2\frac{1}{\sqrt{2\pi}}\int_0^\infty x^2e^{-x^2/2}dx=2\cdot \frac{1}{2}EX^2=1. \]
            \item $y\geq0$, 
            \[F(Y<y)=P(-y<X<y)=F_X(y)-F_X(-y). \]
            \[f_Y(y)=F_Y'(y)=F_X(y)+F_X(-y)=\frac{2}{\sqrt{2\pi}}e^{-y^2/2}. \]
            \[EY=\sqrt{\frac{2}{\pi}}\int_0^\infty ye^{-y^2/2}dy=\sqrt{\frac{2}{\pi}}(-e^{-y^2/2})|_0^\infty=\sqrt{\frac{2}{\pi}}(0+1)=\sqrt{\frac{2}{\pi}}. \]
            \[
                EY^2=\sqrt{\frac{2}{\pi}}\int_{0}^\infty y^2e^{-y^2/2}dy=\sqrt{\frac{2}{\pi}}\sqrt{2\pi}EX^2/2=1. \]
            \[
                Var(Y)=EY^2-(EY)^2=1-\sqrt{\frac{2}{\pi}}^2=1-\frac{2}{\pi}. 
            \]
        \end{enumerate}
    \end{solution}

    \setcounter{exer}{22}
    \begin{exercise}
        Let $X$ have the p.d.f. 
        \[f(x)=\frac{1}{2}(1+x), \quad -1<x<1 .\]
        \begin{enumerate}[(a)]
            \item Find the p.d.f. of $Y=X^2$. 
            \item Find $EY$ and $Var(Y)$. 
        \end{enumerate}
    \end{exercise}

    \begin{solution}
        \begin{enumerate}[(a)]
            \item \[F_Y(y)=P(Y<y)=P(-\sqrt{y}<X<\sqrt{y})=F_X(\sqrt{y})-F_X(-\sqrt{y}). \]
            \[f_Y(y)=\frac{1}{2\sqrt{y}}\left(\frac{1}{2}(1+\sqrt{y}+1-\sqrt{y})\right)=\frac{1}{2\sqrt{y}}, y\in(0,1). \]
            \item \[EY=\int_0^1 y/(2\sqrt{y})dy=\frac{1}{2}\int_0^1y^{1/2}dy=\frac{1}{2}\frac{2}{3}y^{3/2}|_0^1=\frac{1}{3}. \]
            \[EY^2=\int_0^1 y^2/(2\sqrt{y})dy=\frac{1}{2}\int_0^1y^{3/2}dy=\frac{1}{2}\frac{2}{5}y^{5/2}|_0^1=\frac{1}{5}, \]
            \[Var(Y)=EX^2-(EX)^2=\frac{1}{5}-\frac{1}{9}=\frac{4}{45}. \]
        \end{enumerate}
    \end{solution}

    \setcounter{exer}{24}
    \begin{exercise}
        Suppose the p.d.f. $f_X(x)$ of a random variable $X$ is an even function. Show that 
        \begin{enumerate}[(a)]
            \item $X$ and $-X$ are identically distributed. 
            \item $M_X(t)$ is symmetric about $0$.
        \end{enumerate}
    \end{exercise}

    \begin{solution}
        \begin{enumerate}[(a)]
            \item For $Y=-X$, 
            \[f_Y(y)=f_X(-y)|(-y)'|=f_X(y)|-1|=f_X(y). \]
            So, they are identically distributed.
            \item $\forall\ t$, we need to prove that $M_X(t)=M_X(-t)$. 
            \begin{align*}
                M_X(t)&=\int_{-\infty}^\infty e^{tX}f_X(x)dx\\
                &=\int_{-\infty}^0 e^{tX}f_X(x)dx+\int_0^\infty e^{tX}f_X(x)dx\\
                &=\int_{-\infty}^0 e^{t(-X)}f_X(-x)dx+\int_0^\infty e^{t(-X)}f_X(-x)dx\\
                &=\int_{-\infty}^0 e^{-tX}f_X(x)dx+\int_0^\infty e^{-tX}f_X(x)dx\\
                &=\int_{-\infty}^\infty e^{-tX}f_X(x)dx=M_X(-t). 
            \end{align*}
        \end{enumerate}
    \end{solution}

    \setcounter{exer}{31}
    \begin{exercise}
        Let $M_X(t)$ be the moment generating function of a random variable $X$. And define $S_X(t)=\log M_X(t)$. Show that
        $$\frac{d}{dt}S(t)|_{t=0}=EX\text{  and  }\frac{d^2}{dt^2}S(t)|_{t=0}=Var(X). $$
    \end{exercise}

    \begin{solution}
        \begin{align*}
            \frac{d}{dt}S(t)&=\frac{d}{dt}\log M_X(t)=\frac{M_X'(t)}{M_X(t)}. 
        \end{align*}
        With $t=0$, $M_X(0)=\int_{-\infty}^\infty e^{0X}f_X(x)dx=\int_{-\infty}^\infty f_X(x)dx=1$, we have
        \[\frac{d}{dt}S(t)=\frac{EX}{1}=EX. \]
        \begin{align*}
            \frac{d^2}{dt^2}S(t)&=\frac{M_X''(t)M_X(t)-(M_X'(t))^2}{M_X^2(t)}. 
        \end{align*}
        With $t=0$, 
        \[\frac{d^2}{dt^2}S(t)=\frac{EX^2-(EX)^2}{1}=Var(X). \]
    \end{solution}

    \begin{exercise}
        In each of the following cases verify the expression for the moment generating function, and in each case use the moment generating function to calculate $EX$ and $Var(X)$. 
        \begin{enumerate}[(a)]
            \item \[P(X=x)=\frac{e^{-\lambda}\lambda^x}{x!}, M_X(t)=e^{\lambda (e^t-1)}, x=0,1,\cdots, \lambda>0. \]
            \item \[P(X=x)=p(1-p)^x, M_X(t)=\frac{p}{1-(1-p)e^t}, x=0,1,\cdots, 0<p<1. \]
            \item \[f_X(x)=\frac{e^{-(x-\mu)^2/(2\sigma^2)}}{\sqrt{2\pi}\sigma}, M_X(t)=e^{\mu t+\sigma^2t^2/2}, -\infty<x<\infty, \sigma>0. \]
        \end{enumerate}
    \end{exercise}

    \begin{solution}
        \begin{enumerate}[(a)]
            \item \begin{align*}
                M_X(t)&=\sum_{x=0}^\infty \frac{e^{-\lambda}\lambda^x}{x!}e^{tx}\\
                &=e^{-\lambda}\sum_{x=0}^\infty \frac{{(\lambda e^t)}^x}{x!}\\
                &=e^{-\lambda}e^{\lambda e^t}=e^{\lambda (e^t-1)}.
            \end{align*}
            \[EX=M_X'(0)=\lambda e^t e^{\lambda (e^t-1)}|_{t=0}=\lambda. \]
            \[EX^2=M_X''(0)=\lambda e^t e^{\lambda (e^t-1)}+\lambda e^t \lambda e^t e^{\lambda(e^t-1)}|_{t=0}=\lambda^2+\lambda. \]
            \[Var(X)=EX^2-(EX)^2=\lambda. \]
            \item \begin{align*}
                M_X(t)&=\sum_{x=0}^\infty p(1-p)^xe^{tx}\\
                &=p\sum_{x=0}^\infty ((1-p)e^t)^x
            \end{align*}
            Only when $(1-p)e^t<1$, i.e. $e^{-t}>1-p$, $t<-\ln(1-p)$, $M_X(t)$ exists. So, 
            \[M_X(t)=\frac{p}{1-(1-p)e^t}. \]

            \[EX=M_X'(0)=\frac{-p(-(1-p)e^t)}{(1-(1-p)e^t)^2}|_{t=0}=\frac{p-p^2}{p^2}=\frac{1-p}{p}. \]
            \begin{align*}
                EX^2&=M_X''(0)\\
                &=\frac{-p(-(1-p)e^t)(1-(1-p)e^t)+2p(-(1-p)e^t)(-(1-p)e^t)}{(1-(1-p)e^t)^3}|_{t=0}\\
                &=\frac{(1-p)p^2+2p(p-1)^2}{p^3}=\frac{(1-p)p+2(p-1)^2}{p^2}. 
            \end{align*}
            \[Var(X)=EX^2-(EX)^2=\frac{p-p^2+p^2-2p+1}{p^2}=\frac{1-p}{p^2}. \]
            \item \begin{align*}
                M_X(t)&=\int_{-\infty}^\infty e^{tx}f_X(x) dx\\
                &=\int_{-\infty}^\infty e^{tx}\frac{e^{-(x-\mu)^2/(2\sigma^2)}}{\sqrt{2\pi}\sigma}dx\\
                &=\frac{1}{\sqrt{2\pi}\sigma}\int_{-\infty}^\infty e^{tx-(x-\mu)^2/(2\sigma^2)}dx. 
            \end{align*}
            \begin{align*}
                tx-(x-\mu)^2/(2\sigma^2)&=tx-\frac{1}{2\sigma^2}(x^2-2\mu x+\mu^2)\\
                &=\frac{1}{2\sigma^2}(2t\sigma^2x-x^2+2\mu x-\mu^2)\\
                &=-\frac{1}{2\sigma^2}(x^2-2(\mu+t\sigma^2)x+\mu^2)\\
                &=-\frac{1}{2\sigma^2}((x-\mu-t\sigma^2)^2-(\mu+t\sigma^2)^2+\mu^2)\\
                &=-\frac{1}{2\sigma^2}((x-\mu-t\sigma^2)^2-(2\mu t\sigma^2+t^2\sigma^4))
            \end{align*}
            \begin{align*}
                M_X(t)&=\frac{1}{\sqrt{2\pi}\sigma}\int_{-\infty}^\infty e^{tx-(x-\mu)^2/(2\sigma^2)}dx\\&=e^{(2\mu t\sigma^2+t^2\sigma^4)/(2\sigma^2)}\frac{1}{\sqrt{2\pi}\sigma}\int_{-\infty}^\infty e^{-(x-\mu-t\sigma^2)^2/(2\sigma^2)}dx\\
                &=e^{(2\mu t\sigma^2+t^2\sigma^4)/(2\sigma^2)}\\
                &=e^{\mu t+\sigma^2t^2/2}.
            \end{align*}
            \[EX=M_X'(0)=e^{\mu t+\sigma^2t^2/2}(\mu+\sigma^2t)|_{t=0}=\mu. \]
            \[EX^2=M_X''(0)=\left(e^{\mu t+\sigma^2t^2/2}(\mu+\sigma^2t)^2+e^{\mu t+\sigma^2t^2/2}\sigma^2\right)|_{t=0}=\mu^2+\sigma^2. \]
            \[Var(X)=EX^2-(EX)^2=\sigma^2. \]
        \end{enumerate}
    \end{solution}
\end{document}