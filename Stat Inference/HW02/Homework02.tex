\documentclass[14pt]{elegantbook}

\newcommand{\CN}{BIOS 7040\\[0.5cm] Statistical Inference I}
\newcommand{\Ti}{Homework 2}
\newcommand{\Pf}{Dr. Srivastav}
\newcommand{\FN}{Zehao}
\newcommand{\LN}{Wang}
\usepackage[fontsize=14pt]{fontsize}

\usepackage{enumitem}
\renewcommand{\chaptername}{Homework}
\begin{document}
\begin{titlepage}
	\begin{center}
		
		\includegraphics[width=0.6\textwidth]{Tulane.png}\\[1cm]

		\textsc{\Huge \CN}\\[0.5cm]
		\textsc{\large \Pf}\\[1.0cm]

		\textsc{\LARGE \Ti}\\[0.5cm]
		\textsc{\large \LN, \FN}\\
		{Master student in Statistics of Math Dept.}

		\vfill

		{\Large \emph{Update: \today}}

	\end{center}
\end{titlepage}

\tableofcontents
\setcounter{page}{0}
\setcounter{chapter}{1}
\chapter{}

    \begin{exercise*}[1]
        Consider an experiment whose sample space consists of a countably infinite number of points. Show that not all points can be equally likely. Can all points have a positive probability of occurring? 
    \end{exercise*}

    \begin{solution}
        Let $S=\{x_1, x_2, \cdots\}$ be a countably infinite sample space (e.g. rational number) and $\Pr(x_i)=a\neq0$. Then 
        \[\Pr(S)=\sum_{i=1}^\infty\Pr(x_i)=\sum_{i=1}^\infty a=\infty\neq 1.\]
        Here we get the contradiction. So not all points can be equally likely. 

        From the previous result, we know that if the infinite sum, i.e., 
        \[\Pr(S)=\sum_{i=1}^\infty \Pr(x_i)\] 
        is finite, then all points can have a positive probability of occurring, (Maybe we need to divide some constant to the original probability to make sure the sum is $1$). Apparently, there are many infinite series whose sum is finite. For example, $\sum_{i=1}^\infty \frac{1}{i^2}=1$. So, let $\Pr(x_i)=1/i^2>0$, then $\Pr(S)=1$. 
    \end{solution}

    \begin{exercise*}[2]
        This problem introduces a simple meteorological model, more complicated versions of which have been proposed in the meteorological literature. Consider a sequence of days and let $R_i$ denote the event that it rains on day $i$. Suppose that $P(R_i|R_{i-1})=\alpha$ and $P(R_i^c|R_{i-1}^c)=\beta$. Suppose further that only today's weather is relevant to predicting tomorrow's weather. That is, $P(R_i|R_{i-1}\cap R_{i-2}\cap \cdots\cap R_0)=P(R_i|R_{i-1})$. 
        \begin{enumerate}[(a)]
            \item If the probability of rain today is $p$, what is the probability of rain tomorrow?
            \item What is the probability of rain the day after tomorrow? 
        \end{enumerate}
    \end{exercise*}

    \begin{solution}
        \begin{enumerate}[(a)]
            \item 
            \begin{align*}
                \Pr(R_{i+1})=&\Pr(R_{i+1}\cap R_i)+\Pr(R_{i+1}\cap R_i^c)\\
                =&\Pr(R_{i+1}|R_i)\Pr(R_i)+\Pr(R_{i+1}|R_i^c)\Pr(R_i^c)\\
                =&\alpha p+(1-\beta)(1-p)
            \end{align*}
            \begin{align*}
                \Pr(R_{i+1}^c)=&\Pr(R_{i+1}^c\cap R_i)+\Pr(R_{i+1}^c\cap R_i^c)\\
                =&\Pr(R_{i+1}^c|R_i)\Pr(R_i)+\Pr(R_{i+1}^c|R_i^c)\Pr(R_i^c)\\
                =&(1-\alpha) p+\beta(1-p)
            \end{align*}
            So, the transition matrix is: 
            \[\begin{bmatrix}
                \Pr(R_{i+1}|R_i)&\Pr(R_{i+1}|R_i^c)\\
                \Pr(R_{i+1}^c|R_i)&\Pr(R_{i+1}^c|R_i^c)
            \end{bmatrix}=\begin{bmatrix}
                \alpha&1-\beta\\
                1-\alpha&\beta
            \end{bmatrix}\]
            Then, \[\left[
                \begin{matrix}
                    \Pr(R_{i+1})\\
                    \Pr(R_{i+1}^c)
                \end{matrix}\right]=\begin{bmatrix}
                    \alpha&1-\beta\\
                    1-\alpha&\beta
                \end{bmatrix}\begin{bmatrix}
                    \Pr(R_i)\\
                    \Pr(R_i^c)
                \end{bmatrix}
            \]
            \item \begin{align*}
                \left[\begin{matrix}
                    \Pr(R_{i+2})\\
                    \Pr(R_{i+2}^c)
                \end{matrix}\right]=&\begin{bmatrix}
                    \alpha&1-\beta\\
                    1-\alpha&\beta
                \end{bmatrix}\begin{bmatrix}
                    \alpha&1-\beta\\
                    1-\alpha&\beta
                \end{bmatrix}\begin{bmatrix}
                    \Pr(R_i)\\
                    \Pr(R_i^c)
                \end{bmatrix}\\
                =&\begin{bmatrix}
                    \alpha&1-\beta\\
                    1-\alpha&\beta
                \end{bmatrix}^2\begin{bmatrix}
                    \Pr(R_i)\\
                    \Pr(R_i^c)
                \end{bmatrix}\\
                =&\begin{bmatrix}
                    \alpha^2+(1-\alpha)(1-\beta)&\alpha(1-\beta)+(1-\beta)\beta\\
                    \alpha(1-\alpha)+(1-\alpha)\beta&\beta^2+(1-\beta)(1-\alpha)
                \end{bmatrix}\begin{bmatrix}
                    p\\1-p
                \end{bmatrix}\\
                =&\begin{bmatrix}
                    \alpha^2p+(1-\alpha)(1-\beta)p+\alpha(1-\beta)(1-p)+(1-\beta)\beta(1-p)\\
                    \alpha(1-\alpha)p+(1-\alpha)\beta p+(1-\alpha)(1-\beta)(1-p)+\beta^2(1-p)
                \end{bmatrix}
            \end{align*}
            \item We first calculate (I use Wolfram Alpha to calculate it): 
            \begin{align*}
                &\begin{bmatrix}
                    a&1-\beta\\
                    1-\alpha&\beta
                \end{bmatrix}^n\\=&\frac{1}{\alpha + \beta - 2}\begin{bmatrix}
                    \alpha (\alpha + \beta - 1)^n - (\alpha + \beta - 1)^n & -\beta (\alpha + \beta - 1)^n + (\alpha + \beta - 1)^n \\{} + \beta - 1 &{} + \beta - 1\\
                    -\alpha (\alpha + \beta - 1)^n + (\alpha + \beta - 1)^n &{} \beta (\alpha + \beta - 1)^n - (\alpha + \beta - 1)^n \\{} + \alpha - 1 &{} + \alpha - 1
                \end{bmatrix}
            \end{align*}
            So, if $n\to\infty$, we have:
            \[\begin{bmatrix}
                a&1-\beta\\
                1-\alpha&\beta
            \end{bmatrix}^n\\=\begin{bmatrix}
                \frac{\beta-1}{\alpha+\beta-2}&\frac{\beta-1}{\alpha+\beta-2}\\
                \frac{\alpha-1}{\alpha+\beta-2}&\frac{\alpha-1}{\alpha+\beta-2}
            \end{bmatrix}\]
            Hence, \[
                \Pr(R_{i+\infty})=\frac{\beta-1}{\alpha+\beta-2}. 
            \]
        \end{enumerate}
    \end{solution}
    
    \setcounter{exer}{46}
    \setcounter{chapter}{1}

    \begin{exercise}
        Prove that the following functions are c.d.f. 
        \begin{enumerate}[(a)]
            \item \[\frac{1}{2}+\frac{1}{\pi}\tan^{-1}(x), \quad x\in(-\infty,\infty). \]
            \item \[(1+e^{-x})^{-1}, \quad x\in(-\infty,\infty). \]
            \item \[e^{-e^{-x}}, \quad x\in(-\infty, \infty). \]
            \item \[1-e^{-x}, \quad x\in(0, \infty). \]
            \item \[F_Y(y)=\left\{\begin{matrix}
                \frac{1-\varepsilon}{1+e^{-y}}&y<0\\
                \varepsilon+\frac{1-\varepsilon}{1+e^{-y}}&y\geq0
            \end{matrix}\right.\]
        \end{enumerate} 
    \end{exercise}

    \begin{solution}
        \begin{enumerate}[(a)]
            \item \[
                \lim_{x\to-\infty}\tan^{-1}(x)\to-\pi/2, \quad \lim_{x\to\infty}\tan^{-1}(x)\to\pi/2. 
                \]
                So, \[
                    \lim_{x\to-\infty}\frac{1}{2}+\frac{1}{\pi}\tan^{-1}(x)\to0, \quad \lim_{x\to\infty}\frac{1}{2}+\frac{1}{\pi}\tan^{-1}(x)\to1. 
                \] and it is nondecreasing and continuous. So, it is a c.d.f. 
            \item $e^{-x}$ is decreasing and continuous. So, $(1+e^{-x})^-1$ is nondecreasing and continuous. In addition, \[
                \lim_{x\to-\infty}(1+e^{-x})^{-1}\to0, \quad \lim_{x\to\infty}(1+e^{-x})^{-1}\to1.
            \] So, it is a c.d.f. 
            \item $e^{-x}$ is decreasing and continuous. So, $e^{-e^{-x}}$ is nondecreasing and continuous. In addition, \[
                \lim_{x\to-\infty}e^{-e^{-x}}\to0, \quad \lim_{x\to\infty}e^{-e^{-x}}\to1.
            \] So, it is a c.d.f.
            \item $e^{-x}$ is decreasing and continuous. So, $1-e^{-x}$ is nondecreasing and continuous. In addition, \[
                \lim_{x\to0}(1-e^{-x})\to0, \quad \lim_{x\to\infty}(1-e^{-x})\to1.
            \] So, it is a c.d.f.
            \item $e^{-y}$ is decreasing, and \[
                \frac{1-\varepsilon}{1+e^{-y}}<\varepsilon+\frac{1-\varepsilon}{1+e^{-y}}. 
            \]
            Therefore, $F_Y(y)$ is nondecreasing. In addition, $F_Y(y)$ is continuous in $(-\infty, 0)$ or $[0, \infty)$. And it is right continuous in $0$, i.e. 
            \[\lim_{x\to0^+}\varepsilon+\frac{1-\varepsilon}{1+e^{-y}}=\varepsilon+\frac{1-\varepsilon}{2}=F_Y(0). \]
            And \[lim_{y\to -\infty}F_Y(y)=0,\quad \lim_{y\to\infty}F_Y(y)=1. \]
            So, $F_Y(y)$ is a c.d.f.
        \end{enumerate}
    \end{solution}

    \begin{exercise}
        Prove the necessary part of Theorem 1.5.3. 
    \end{exercise}

    \begin{solution}
        Let $F(x)$ be a c.d.f., i.e. $F(x)=\Pr(X<x)$. Then, if $x_1\leq x_2$, 
        \[F(x_1)=\Pr(X<x_1)=\int_{-\infty}^{x_1}f(x)d x\]
        \begin{align*}
            F(x_2)=\Pr(X<x_2)&=\int_{-\infty}^{x_2}f(x)d x=\int_{-\infty}^{x_1}f(x)d x+\int_{x_1}^{x_2}f(x)d x\\
            &=F(x_1)+\int_{x_1}^{x_2}f(x)d x \geq F(x_1), \quad \text{$f$ is positive}.
        \end{align*}
        So, $F(x)$ is nondecreasing. Apparently, $F_Y(-\infty)=0$, and $F_Y(\infty)=1$. 

        Let $A_n=\{X<x+x_n\}$, $x_n>0$, and $x_n\to 0$. We need to prove $\lim_{n\to\infty}\Pr(A_n)=\Pr(X<x)$. Because $A_1\supset A_2\supset A_3\supset \cdots$, we have
        \[
            \Pr(\lim _{n\to\infty}A_n)=\lim_{n\to\infty}\Pr(A_n). 
        \]
        i.e., 
        \[
            \Pr(\lim_{n\to\infty}X<x+x_n)=\Pr(X<x)=\lim_{n\to\infty}\Pr(A_n). 
        \]
        So, $F(x)$ is also right-continuous. 
    \end{solution}

    \setcounter{exer}{51}
    \begin{exercise}
        Let $X$ be a continuous random variable with p.d.f. $f(x)$ and c.d.f. $F(x)$. For a fixed number $x_0$, define the function
        \[
            g(x)=\left\{\begin{matrix}
                f(x)/(1-F(x_0))&x\geq x_0\\
                0&x<x_0. 
            \end{matrix}\right.
        \]
        Prove that $g$ is a p.d.f. (Assume that $F(x_0)<1$.)
    \end{exercise}

    \begin{solution}
        $1-F(x_0)>0$, $f(x)\geq 0$. So, $g(x)\geq 0$. And \[
            \int_{x_0}^\infty g(x) dx=\frac{1}{1-F(x_0)}\int _{x_0}^\infty f(x) dx=\frac{1}{1-F(x_0)}(1-F(x_0))=1. 
        \]
        So, it is a p.d.f.
    \end{solution}

    \begin{exercise}
        A certain river floods every year. Suppose that the low-water mark is set at $1$ and the high-water mark $Y$ has distribution function 
        \[
            F_Y(y)=P(Y\leq y)=1-1/y^2, \quad 1\leq y <\infty. 
        \]
        \begin{enumerate}[(a)]
            \item Verify that $Y$ is a c.d.f.
            \item Find $f_Y(y)$, the p.d.f. of $Y$.
            \item If the low-water mark is reset at $0$ and we use a unit of measurement that is $1/10$ of that given previously, the high-water mark becomes $Z=10(Y-1)$. Find $F_Z(z)$. 
        \end{enumerate}
    \end{exercise}

    \begin{solution}
        \begin{enumerate}[(a)]
            \item $F_Y(y)$ is continuous. And $F_Y(1)=0$, $F_Y(\infty)=1$. $1/y^2$ is decreasing, so $1-1/y^2$ is nondecreasing. So, $F_Y(y)$ is a c.d.f.
            \item Because $F_Y(y)$ is continuous, $f_Y(y)=F_Y'(y)=2/y^3$, $1\leq y<\infty$. 
            \item \begin{align*}
                F_Z(z)&=P(Z\leq z)=P(10(Y-1)\leq z)=P(Y\leq z/10+1)\\
                &=1-1/(z/10+1)^2, \quad 0\leq z<\infty. 
            \end{align*}
        \end{enumerate}
    \end{solution}


\end{document}