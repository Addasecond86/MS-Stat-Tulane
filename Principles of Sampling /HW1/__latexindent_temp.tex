\documentclass[12pt]{article}

\newcommand{\CN}{Principles of Sampling}
\newcommand{\Ti}{Homework 1}
\newcommand{\Pf}{Dr. Li}
\newcommand{\FN}{Zehao}
\newcommand{\LN}{Wang}

\usepackage[left=2cm, right=2cm, top=2cm, bottom=2cm]{geometry}

\title{\textsc{\Ti}}
\author{\textsc{\LN, \FN}}
\date{\emph{\today}}


\usepackage{tcolorbox}
\tcbuselibrary{breakable}
\usepackage{amsfonts} 
\usepackage{amsmath}
\usepackage{amssymb}
\usepackage{newtxmath}
\usepackage{enumerate}
\usepackage{minted}
\usepackage{fancyhdr}


\pagestyle{fancy}
\fancyhf{}
\rhead{\textsc{\FN}}
\lhead{\textsc{\CN}}
\chead{\textsc{\Ti}}
\rfoot{\thepage}


\linespread{1.3}
\setlength{\parskip}{3mm}
\setlength{\parindent}{2em}

% \usepackage[backend=biber]{biblatex}
% \addbibresource{my.bib}

\numberwithin{equation}{section}

% Problem 

\newcounter{exercise}[section]

\newenvironment{exercise}[1][\thesection.\refstepcounter{exercise}\theexercise]{\begin{tcolorbox}[colback=black!15, colframe=black!80, breakable, title=#1]}{\end{tcolorbox}}

% Proof

\newenvironment{proof}{\begin{tcolorbox}[colback=white, colframe=black!50, breakable, title=Proof. ]\setlength{\parskip}{0.8em}}{\hfill $\blacksquare$ \end{tcolorbox}}

\newenvironment{solution}{\begin{tcolorbox}[colback=white, colframe=black!50, breakable, title=Solution. ]\setlength{\parskip}{0.8em}}{\end{tcolorbox}}

\newcommand{\pder}{\partial\,}

\newcommand{\der}{\,\mathbf{d}\,}


\newcommand\define[1]{\emph{\bfseries Definition \the\value{section}.#1}} 
\newcommand\thm[1]{\emph{\bfseries Theorem \the\value{section}.#1}}
\newcommand\rmk[1]{\emph{\bfseries Remark \the\value{section}.#1}}
\newcommand\prop[1]{\emph{\bfseries Propositions \the\value{section}.#1}}

\usepackage{titlesec}

\titleformat{\section}{\bfseries\Large}{}{0em}{}

\AddToHook{cmd/section/before}{\clearpage}

\usepackage{hyperref}



\usepackage{longtable}

\begin{document}
% \maketitle
\begin{titlepage}
	\begin{center}
		
		\includegraphics[width=0.6\textwidth]{Tulane.png}\\[1cm]

		\textsc{\Huge \CN}\\[0.5cm]
		\textsc{\large \Pf}\\[1.0cm]

		\textsc{\LARGE \Ti}\\[0.5cm]
		\textsc{\large \LN, \FN}\\
		{Master student in Statistics of Math Dept.}

		\vfill

		{\Large \emph{Update: \today}}

	\end{center}
\end{titlepage}

\tableofcontents
\setcounter{page}{0}
    
    \section{Homework 1}

    \begin{exercise}[Lohr-2.1]
        Let $N = 6$ and $n = 3$. For purposes of studying sampling distributions, assume that all population values are known. 
        \[
            \begin{array}{cll}
            y_{1}=98 & y_{2}=102 & y_{3}=154 \\
            y_{4}=133 & y_{5}=190 & y_{6}=175
            \end{array}
        \]
        We are interested in \(\bar{y}_u\), the population mean. Two sampling plans are proposed. 
        \begin{itemize}
            \item Plan 1. Eight possible samples may be chosen. 
            \begin{center}
                \begin{tabular}{ccc} 
                    Sample Number & Sample, \(\mathcal{S}\) & \(P(\mathcal{S})\) \\
                    \hline 1 & \(\{1,3,5\}\) & \(1 / 8\) \\
                    2 & \(\{1,3,6\}\) & \(1 / 8\) \\
                    3 & \(\{1,4,5\}\) & \(1 / 8\) \\
                    4 & \(\{1,4,6\}\) & \(1 / 8\) \\
                    5 & \(\{2,3,5\}\) & \(1 / 8\) \\
                    6 & \(\{2,3,6\}\) & \(1 / 8\) \\
                    7 & \(\{2,4,5\}\) & \(1 / 8\) \\
                    8 & \(\{2,4,6\}\) & \(1 / 8\)
                \end{tabular}
            \end{center}
            \item Plan 2. Three possible samples may be chosen.
            \begin{center}
                \begin{tabular}{ccc} 
                    Sample Number & Sample, \(\mathcal{S}\) & \(P(\mathcal{S})\) \\
                    \hline 1 & \(\{1,4,6\}\) & \(1 / 4\) \\
                    2 & \(\{2,3,6\}\) & \(1 / 2\) \\
                    3 & \(\{1,3,5\}\) & \(1 / 4\)
                \end{tabular}
            \end{center}
        \end{itemize}
        \begin{enumerate}[(a)]
            \item What is the value of \(\bar{y}_u\)? 
            \item Let \(\bar{y}\) be the mean of the sample values. For each sampling plan, find
            
            (i) \(E[\bar{y}]\);\ 
            (ii) \(V[\bar{y}]\);\ 
            (iii) \(\operatorname{Bias}(\bar{y})\)\ 
            (iv) \(\operatorname{MSE}(\bar{y})\). 
            \item Which sampling plan do you think is better? Why?
        \end{enumerate}
    \end{exercise}

    \begin{solution}
        \begin{enumerate}[(a)]
            \item $\bar{y}_u=\frac{1}{n}\sum_{i=1}^n y_i=852/6=142. $
            \item 
            \begin{itemize}
                \item Plan 1. 
                \begin{enumerate}[(i)]
                    \item $\bar{y}_1=147.33$, $\bar{y}_2=142.33$, $\bar{y}_3=140.33$, $\bar{y}_4=135.33$, \\
                    $\bar{y}_5=148.67$, $\bar{y}_6=143.67$, $\bar{y}_7=141.67$, $\bar{y}_8=136.67$.\\
                    $E[\bar{y}]=\frac{1}{8}\sum_{i=1}^8\bar{y}_i=142$. 
                    \item $V[\bar{y}]=\frac{1}{8}\sum_{i=1}^8(\bar{y}_i-E(\bar{y}_i))^2=18.94$. 
                    \item $Bias(\bar{y})=E(\bar{y})-\bar{y}_u=0$. 
                    \item $MSE(\bar{y})=\frac{1}{8}\sum_{i=1}^8(\bar{y}_i-\bar{y}_u)^2=18.94$. 
                \end{enumerate}
                \item Plan 2.  
                \begin{enumerate}[(i)]
                    \item $\bar{y}_1=135.33$, $\bar{y}_2=143.67$, $\bar{y}_3=147.33$.\\
                    $E[\bar{y}]=\sum_{i=1}^3p_i\bar{y}_i=142.5$. 
                    \item $V[\bar{y}]=\sum_{i=1}^3p_i(\bar{y}_i-E(\bar{y}_i))^2=19.36$. 
                    \item $Bias(\bar{y})=E(\bar{y})-\bar{y}_u=0.5$. 
                    \item $MSE(\bar{y})=\sum_{i=1}^3p_i(\bar{y}_i-\bar{y}_u)^2=19.61$. 
                \end{enumerate}
            \end{itemize}
            \item In general, Plan 1 is better because its bias, variance, and mean square error are smaller. However, in terms of convenience, Plan 2 has fewer kinds of samples. 
        \end{enumerate}
        
    \end{solution}


    
    \begin{exercise}[Lohr-2.2]
        For the population in Example \(2.2\), consider the following sampling scheme:

        \begin{center}
            \begin{tabular}{cc}
                \(\mathcal{S}\) & \(P(\mathcal{S})\) \\
                \hline\(\{1,3,5,6\}\) & \(1 / 8\) \\
                \(\{2,3,7,8\}\) & \(1 / 4\) \\
                \(\{1,4,6,8\}\) & \(1 / 8\) \\
                \(\{2,4,6,8\}\) & \(3 / 8\) \\
                \(\{4,5,7,8\}\) & \(1 / 8\)
            \end{tabular}
            \begin{enumerate}
                \item Find the probability of selection \(\pi_{i}\) for each unit \(i\). 
                \item What is the sampling distribution of \(\hat{t}=8 \bar{y}\) ?
            \end{enumerate}
        \end{center}
        Population in Example 2.2: 
        \[
            y_i = \{1,2,4,4,7,7,7,8\}. 
        \]
    \end{exercise}

    \begin{solution}
        \begin{enumerate}
            \item $\pi_1=P(S_1)+P(S_3)=1/4$, \\
            $\pi_2=P(S_2)+P(S_4)=5/8$, \\
            $\pi_3=P(S_1)+P(S_2)=3/8$, \\
            $\pi_4=P(S_3)+P(S_4)+P(S_5)=5/8$, \\
            $\pi_5=P(S_1)+P(S_5)=1/4$, \\
            $\pi_6=P(S_1)+P(S_3)+P(S_4)=5/8$, \\
            $\pi_7=P(S_2)+P(S_5)=3/8$, \\
            $\pi_8=P(S_2)+P(S_3)+P(S_4)+P(S_5)=1-1/8=7/8$. 
            \item $\hat{t}_{S_1}=38$, $\hat{t}_{S_2}=42$, $\hat{t}_{S_3}=40$, $\hat{t}_{S_4}=42$, $\hat{t}_{S_5}=52$. 
            \[
                \left\{\begin{array}{l}
                    P(\hat{t}=38)=1/8 \\
                    P(\hat{t}=40)=1/8 \\
                    P(\hat{t}=42)=1/4+3/8=5/8 \\
                    P(\hat{t}=52)=1/8
                \end{array}\right.
            \]
        \end{enumerate}
    \end{solution}

    \begin{exercise}[Levy-2.4]
        As part of an AIDS education program, 120 intravenous drug users seronegative for HIV (Human Immunodeficiency Virus) at a first screening were given instructions on sterilizing their needles with bleach and practicing ``safe sex." One year after the program's inception, a sample of 30 of these subjects was taken by numbering the participants from 1 to 120 and by taking all subjects whose numbers are divisible by 4 (e.g., \(4,8,12\), etc.). 
        \begin{enumerate}[a.]
            \item What is the chance of each individual being chosen in the sample?
            \item If subjects \(1,3,4,8,29\), and 65 are seropositive for HIV, what is the proportion of seroconverted subjects for this population?
            \item What is the proportion of seroconverted subjects in the sample? Is this an unbiased estimate of the population proportion?
        \end{enumerate}
    \end{exercise}

    \begin{solution}
        \begin{enumerate}[a.]
            \item $p=30/120=0.25$. 
            \item $p=6/120=0.05$.
            \item $\hat{p}=2/30=1/15$. It is not unbiased. 
        \end{enumerate}
    \end{solution}

    \begin{exercise}[Levy-2.8]
        The following table shows the total number of laboratory determinations for the 10 sample patients described in Exercise 2.6 as well as the number of invalid determinations among the 10 determinations sampled: \\[0.5em]
        \begin{tabular}{ccc}
            \hline Subject & Total Laboratory Values & Total Invalid Laboratory Values in 10 Sampled \\
            \hline 1 & 103 & 1 \\
            2 & 123 & 0 \\
            3 & 93 & 2 \\
            4 & 200 & 0 \\
            5 & 128 & 1 \\
            6 & 165 & 0 \\
            7 & 132 & 0 \\
            8 & 189 & 0 \\
            9 & 176 & 0 \\
            10 & 180 & 1 \\
            \hline
        \end{tabular}
        \\[0.5em]
        Based on these data, estimate the proportion of laboratory values that are invalid using all of the data in the above table. 
    \end{exercise}
    \begin{exercise}[Exercise 2.6]
        It is desired to perform a quality control audit of laboratory data from a large clinical trial for purposes of estimating the proportion of laboratory values in the database that are invalid. There are 394 patients in the clinical trial, and each person has had from 60 to 200 laboratory determinations during the course of the trial. The sampling plan chosen was to select a random sample of 10 patients and for each sample person take a random sample of 10 laboratory determinations, which would then be checked against the patient medical record for accuracy. 
    \end{exercise}

    \begin{exercise}[Levy-3.6]
        A survey of workers is to be taken in a large plant that makes products similar to those made in the plant from which the data in Table 3.8 are obtained. The purposes of the survey are to estimate 
        \begin{enumerate}[(a)]
            \item the proportion of all workers having an fvc below $70\%$. 
            \item the mean fvc among all workers. 
        \end{enumerate}
        Estimates are needed within $5\%$ of the true value of the parameter being estimated. How large a sample of workers is required? The plant employs 5000 workers.
    \end{exercise}

    \begin{longtable}[c]{lllllllllll}
        \hline
        Worker & Exposure & Fvc & Popsize & Wtl &  & Worker & Exposure & Fvc & Popsize & Wtl \\ \hline
        \endfirsthead
        %
        \endhead
        %
        \hline
        \endfoot
        %
        \endlastfoot
        %
        1      & 3        & 81  & 1200    & 30  &  & 21     & 2        & 70  & 1200    & 30  \\
        2      & 3        & 64  & 1200    & 30  &  & 22     & 1        & 64  & 1200    & 30  \\
        3      & 2        & 85  & 1200    & 30  &  & 23     & 3        & 72  & 1200    & 30  \\
        4      & 2        & 91  & 1200    & 30  &  & 24     & 2        & 72  & 1200    & 30  \\
        5      & 3        & 60  & 1200    & 30  &  & 25     & 3        & 95  & 1200    & 30  \\
        6      & 1        & 97  & 1200    & 30  &  & 26     & 3        & 96  & 1200    & 30  \\
        7      & 1        & 82  & 1200    & 30  &  & 27     & 3        & 62  & 1200    & 30  \\
        8      & 1        & 99  & 1200    & 30  &  & 28     & 3        & 67  & 1200    & 30  \\
        9      & 3        & 96  & 1200    & 30  &  & 29     & 3        & 95  & 1200    & 30  \\
        10     & 3        & 91  & 1200    & 30  &  & 30     & 1        & 87  & 1200    & 30  \\
        11     & 1        & 71  & 1200    & 30  &  & 31     & 3        & 84  & 1200    & 30  \\
        12     & 3        & 88  & 1200    & 30  &  & 32     & 3        & 89  & 1200    & 30  \\
        13     & 2        & 84  & 1200    & 30  &  & 33     & 3        & 89  & 1200    & 30  \\
        14     & 3        & 85  & 1200    & 30  &  & 34     & 3        & 65  & 1200    & 30  \\
        15     & 3        & 77  & 1200    & 30  &  & 35     & 3        & 67  & 1200    & 30  \\
        16     & 3        & 76  & 1200    & 30  &  & 36     & 3        & 69  & 1200    & 30  \\
        17     & 3        & 62  & 1200    & 30  &  & 37     & 3        & 80  & 1200    & 30  \\
        18     & 3        & 67  & 1200    & 30  &  & 38     & 3        & 98  & 1200    & 30  \\
        19     & 3        & 91  & 1200    & 30  &  & 39     & 3        & 65  & 1200    & 30  \\
        20     & 2        & 99  & 1200    & 30  &  & 40     & 3        & 84  & 1200    & 30  \\ \hline
    \end{longtable}

    \begin{exercise}[Levy-4.2(a-c)]
        From the 120 homes of Exercise 1, suppose a one-in-five systematic sample was taken and suppose that the initial random number was 5 .
        \begin{enumerate}[a.]
            \item Estimate the proportion of homes with lead hazards from this sample.
            \item Obtain a \(95 \%\) confidence interval for the proportion of homes with lead hazards.
            \item What is the true variance of the distribution of the estimated proportion of lead hazardous homes? Compare this result with the estimated variance you used in part (b).
        \end{enumerate}
    \end{exercise}

    \begin{exercise}[Exercise 4.1]
        Suppose the local Childhood Lead Poisoning Prevention Council in a metropolitan area in western Tennessee undertakes the responsibility of determining the proportion of homes in a certain development of 120 homes with unsafe lead levels. Because of the great expense involved in performing spectrometric testing of interior walls, ceilings, floors, baseboards, cabinets, and other obvious lead hazards such as crib bars, as well as of exterior sidings, porches and porch rails, it was decided to select a sample of the homes under study. A good up-to-date frame exists for sampling purposes. This frame is a street listing containing the address and owner of each home for each of the streets in the target area. It was decided to select a one-in-three sample of homes. Let us assume that the only houses with serious lead hazard problems are the 26th, 27th, 28th, and 29th on the list. 
    \end{exercise}
    % \printbibliography
\end{document}
