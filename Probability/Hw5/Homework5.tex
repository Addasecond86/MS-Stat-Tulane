\documentclass[en, normal, 11pt, black]{elegantnote}

\usepackage{tcolorbox}
\tcbuselibrary{breakable}
\usepackage{amsfonts} 
\usepackage{newtxtext}
\usepackage{ulem}
\usepackage{amssymb}

\newenvironment{exercise}[1]{\begin{tcolorbox}[colback=black!15, colframe=black!80, breakable, title=#1]}{\end{tcolorbox}}

\renewenvironment{proof}{\begin{tcolorbox}[colback=white, colframe=black!50, breakable, title=Proof. ]\setlength{\parskip}{0.8em}}{\,\\\rightline{$\square$}\end{tcolorbox}}

\newenvironment{solution}{\begin{tcolorbox}[colback=white, colframe=black!50, breakable, title=Solution. ]\setlength{\parskip}{0.8em}}{\end{tcolorbox}}

\newcommand{\pder}{\partial\,}

\newcommand{\der}{\,\mathbf{d}}

\title{\textsc{Probability: Problem Set 5}}
\author{\textsc{Zehao Wang}}
\date{\emph{November 9, 2021}}

% \vspace{-30pt}

\begin{document}
    \maketitle
    \begin{exercise}{2.2.1}
        Let $X_{1}, X_{2}, \cdots$ be uncorrelated with $E (X_{i})=\mu_{i}$ and $\frac{Var\left(X_{i}\right) }{i} \rightarrow 0$ as $i \rightarrow \infty$. Let $S_{n}=X_{1}+\cdots+X_{n}$ and $v_{n}=E\left(\frac{S_n}{n}\right)$ then, prove that as $n \rightarrow \infty$, 
        \[
            \frac{S_{n}}{n}-v_{n} \rightarrow 0
        \]
        in $L^{2}$ and in probability. 
    \end{exercise}

    \begin{proof}
        \[
            E\left(\frac{S_n}{n}-v_n\right)^2=Var\left(\frac{S_n}{n}\right)=\frac{1}{n^2}Var(S_n)=\frac{1}{n^2}\sum_{k=1}^nVar(X_k), 
        \]
        Because $\frac{Var(X_i)}{i}\to0$, for any $\varepsilon>0$, there exist $A$, such that $Var(X_i)<A+i\varepsilon$. So, 
        \[
            \frac{1}{n^2}\sum_{k=1}^nVar(X_k)\leqslant A/n+\varepsilon. 
        \]
        With $n\to\infty$, $\frac{S_n}{n}-v_n\to0$. 
    \end{proof}

    \begin{exercise}{2.2.2}
        The $L^{2}$ weak law generalizes immediately to certain dependent sequences. Suppose 
        \[
            E (X_{n})=0, \qquad E (X_{n} X_{m}) \leqslant r(n-m)
        \]
        for $m \leqslant n$ (no absolute value on the left-hand side!) with $r(k) \rightarrow 0$ as $k \rightarrow \infty$. Show that 
        \[
            \frac{X_{1}+\cdots+X_{n}}{ n} \rightarrow 0
        \]
        in probability. 
    \end{exercise}

    \begin{proof}
        Let $S_n=X_1+\cdots+X_n$. And what we want to prove is 
        \[
            P\left(\left|\frac{S_n}{n}\right|>\varepsilon\right)=0. 
        \]

        From Chebyshev's inequality, we know that 
        \[
            P\left(\left|\frac{S_n}{n}\right|>\varepsilon\right)\leqslant\frac{Var(S_n/n)}{\varepsilon^2}. 
        \]
        And
        \[
            Var(S_n)=E(S_n)^2
        \]
        Because $r(k)\to0$, for any $\varepsilon>0$, there exist $K\in\mathbb{Z}$, such that when $k>K$, $r(k)<\varepsilon$. 
        \[
            \begin{aligned}
                E(S_n)^2=\sum_{1\leqslant i,j\leqslant n}E(X_iX_j)&=\sum_{|i-j|\leqslant K}E(X_iX_j)+\sum_{|i-j|>K}E(X_iX_j)\\
                &\leqslant Kc+\sum_{|i-j|>K}\varepsilon\\
                &\leqslant Kc+n\varepsilon. 
            \end{aligned}
        \]
        Where $c=\max\{r(0), r(1), \cdots, r(K)\}$. 
        So, 
        \[
            Var(S_n/n)=\frac{E(S_n)^2}{n^2}\leqslant\frac{Kc}{n^2}+\frac{\varepsilon}{n}\to0,\quad\text{with $n\to0$}. 
        \]
    \end{proof}

    \begin{exercise}{2.2.3. Monte Carlo integration. }
        (i) Let $f$ be a measurable function on $[0,1]$ with 
        \[
            \int_{0}^{1}|f(x)| \der x<\infty. 
        \]
        Let $U_{1}, U_{2}, \cdots$ be independent and uniformly distributed on $[0,1]$, and let
        \[
            I_{n}=n^{-1}\left(f\left(U_{1}\right)+\cdots+f\left(U_{n}\right)\right), 
        \]
        Show that 
        \[
            I_{n} \rightarrow I \equiv \int_{0}^{1} f \der x
        \]
        in probability. 
        
        (ii) Suppose 
        \[
            \int_{0}^{1}|f(x)|^{2} \der x<\infty. 
        \]
        Use Chebyshev's inequality to estimate 
        \[
            P\left(\left|I_{n}-I\right|>a / n^{1 / 2}\right). 
        \]
    \end{exercise}

    \begin{proof}
        {\bfseries{(i). }}
        From Weak Law of Large Number, 
        \[
            E\left(\frac{\sum_{i=1}^nf(U_i)}{n}\right)\to \int_0^1f\der x. 
        \]

        {\bfseries{(ii). }}
        \[
            P\left(|I_n-I|>\frac{a}{n^{1/2}}\right)\leqslant\frac{n}{a^2}Var(I_n)=\frac{n\cdot E(I_n-I)^2}{a^2}=\frac{n}{a^2}\left(\int_0^1f^2\der x-\left(\int_0^1f\der x\right)^2\right). 
        \]
        \vspace*{-20pt}
    \end{proof}

    \begin{exercise}{2.2.5}
        Let $X_{1}, X_{2}, \cdots$ be i.i.d. with 
        \[
            P\left(X_{i}>x\right)=\frac{e}{x \log x}
        \]
        for $x \geqslant e$. Show that 
        \[
            E\left|X_{i}\right|=\infty, 
        \]
        but there is a sequence of constants $\mu_{n} \rightarrow \infty$ so that $\frac{S_{n}}{n}-\mu_{n} \rightarrow 0$ in probability. 
    \end{exercise}

    \begin{proof}
        \[
            E(X_i)=\int_e^\infty P(X_i>x)\der x_i=e\log(\log x)\big|_e^\infty=\infty. 
        \]
        Define $\mu_n^{(i)}=\int_e^nP(X_i>x)\der x_i$. So, 
        \[
            \mu_n^{(i)}=e\log(\log x)\big|_e^n=e\log(\log n)\to\infty. 
        \]
        So, $S_n/n-\mu_n\to0$, with $n\to\infty$. 
    \end{proof}

    \begin{exercise}{2.2.6}
        (i) Show that if $X \geqslant 0$ is integer valued, 
        \[
            E(X)=\sum_{n \geqslant 1} P(X \geqslant n). 
        \]
        
        (ii) Find a similar expression for $E (X^{2})$. 
    \end{exercise}

    \begin{proof}
        \bfseries{(i). }
        \[
            \begin{aligned}
                E(X)=&\sum_{k=0}^\infty k\cdot P(X=k)\\
                =&1\cdot P(X=1)+2\cdot P(X=2)+\cdots\\
                =&\quad P(X=1)\\
                &+P(X=2)+P(X=2)\\
                &+P(X=3)+P(X=3)+P(X=3)\\
                &+\cdots\\
                =&\sum_{k=1}^\infty P(X=k)+\sum_{k=2}^\infty P(X=k)+\sum_{k=3}^\infty P(X=k)+\cdots\\
                =&P(X>0)+P(X>1)+P(X>2)+\cdots\\
                =&\sum_{k=0}^\infty P(X>k). 
            \end{aligned}
        \]
        \bfseries{(ii). }
        \[
            \begin{aligned}
                E(X^2)=&\sum_{k=0}^\infty k^2\cdot P(X^2=k^2)\\
                =&1^2\cdot P(X^2=1^2)+2^2\cdot P(X^2=2^2)+\cdots\\
                =&\quad1\cdot P(X^2=1^2)\\
                &+2\cdot P(X^2=2^2)+2\cdot P(X^2=2^2)\\
                &+3\cdot P(X^2=3^2)+3\cdot P(X^2=3^2)+3\cdot P(X^2=3^2)\\
                &+4\cdot P(X^2=4^2)+4\cdot P(X^2=4^2)+4\cdot P(X^2=4^2)+4\cdot P(X^2=4^2)\\
                &+\cdots\\
                =&\sum_{k=1}^\infty k\cdot P(X^2=k^2)+\sum_{k=2}^\infty k\cdot P(X^2=k^2)+\sum_{k=3}^\infty k\cdot P(X^2=k^2)+\cdots\\
                =&\sum_{n=1}^\infty\sum_{k=n}^\infty k\cdot P(X^2=k^2). 
            \end{aligned}
        \]
    \end{proof}
\end{document}
