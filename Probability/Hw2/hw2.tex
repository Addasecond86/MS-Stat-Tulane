\documentclass[en, normal, 11pt, black]{elegantnote}

\usepackage{tcolorbox}
\tcbuselibrary{breakable}
\usepackage{amsfonts}
\usepackage{newtxtext}
\usepackage{ulem}
\usepackage{amssymb}

\newenvironment{exercise}[1]{\begin{tcolorbox}[colback=black!15, colframe=black!80, breakable, title=#1]}{\end{tcolorbox}}

\renewenvironment{proof}{\begin{tcolorbox}[colback=white, colframe=black!50, breakable, title=Proof. ]\setlength{\parskip}{0.8em}}{\\\rightline{$\square$}\end{tcolorbox}}

\newenvironment{solution}{\begin{tcolorbox}[colback=white, colframe=black!50, breakable, title=Solution. ]\setlength{\parskip}{0.8em}}{\end{tcolorbox}}

\newcommand{\pder}{\partial\,}

\newcommand{\der}{\,\mathbf{d}}

\title{\textsc{Probability: Problem Set 2}}
\author{\textsc{Zehao Wang}}
\date{\today}

% \vspace{-30pt}

\begin{document}
\maketitle
    \begin{exercise}{1.4.1}
        Show that if $f\geqslant0$ and $\int f\der \mu=0$, the $f=0$ a.e. 
    \end{exercise}
    \begin{proof}
        We need to prove $\mu\left(\left\{x:f(x)>0\right\}\right)=0$. And let $A_\varepsilon=\left\{x:f(x)>\varepsilon\right\}$. So, when $\varepsilon\to0$, $A_\varepsilon\to A_0$. 
        And from the definition of measure, we can know: 
        \begin{align*}
            0\leqslant\varepsilon\mu(A_\varepsilon)=\int_{A_\varepsilon}\varepsilon\der\mu<\int f\der\mu=0, 
        \end{align*}
        So, $\mu(A_\varepsilon)=0$, and considering the arbitrary of $\varepsilon$, we can know that: 
        \[\mu(A_0)=\lim_{\varepsilon\to0}\mu(A_\varepsilon)=0. \]\vspace{-30pt}
    \end{proof}

    \begin{exercise}{1.4.2}
        Let $f \geq 0$ and $E_{n, m}=\left\{x: m / 2^{n} \leqslant f(x)<(m+1) / 2^{n}\right\}$. Show that as $n \uparrow \infty$, 
        \[
            \sum_{m=1}^{\infty} \frac{m}{2^{n}} \mu\left(E_{n, m}\right) \uparrow \int f d \mu. 
        \]
    \end{exercise}
    \begin{proof}
        Let $g=\sum_{m=1}^\infty\frac{m}{2^n}\mathbf{1}_{E_{n,m}}$, $\forall\,x\in E_{n,m}$, $\frac{m}{2^n}<f(x)$. So, we have 
        \[
            \sum_{m=1}^\infty\frac{m}{2^n}\mu(E_{n,m})=\int g\der \mu\leqslant\int f\der \mu. 
        \]
        Hence as $n\to\infty$, $\sup \sum_{m=1}^\infty\frac{m}{2^n}\mu(E_{n,m})\leqslant \int f\der \mu$. 

        For the other inequation, 
        \[
            g+\sum_{m=1}^\infty \frac{1}{2^n}\mathbf{1}_{E_{n,m}}\geqslant f\mathbf{1}_{n,m}
        \]
        \[
            \frac{1}{2^n}\sum_{m=1}^\infty\mu(E_{n,m})+\sum_{m=1}^\infty\frac{m}{2^n}\mu(E_{n,m})\geqslant\int f\der\mu, 
        \]
        As $n\to 0$, we can write above inequation as: 
        \[
            \sum_{m=1}^\infty\frac{m}{2^n}\mu(E_{n,m})\geqslant\int f\der\mu, 
        \]
        \[
            \inf \sum_{m=1}^\infty\frac{m}{2^n}\mu(E_{n,m})\geqslant\int f\der\mu\geqslant\sup \sum_{m=1}^\infty\frac{m}{2^n}\mu(E_{n,m}). 
        \]
        So, $\sum_{m=1}^{\infty} \frac{m}{2^{n}} \mu\left(E_{n, m}\right) \uparrow \int f d \mu$. 
    \end{proof}

    % \begin{exercise}{A.2.1}
    %     Let $B$ be the nonmeasurable set constructed in Theorem A.2.4. 
        
    %     (i) Let $B_{q}=$ $q+^{\prime} B$ and show that if $D_{q} \subset B_{q}$ is measurable, then $\lambda\left(D_{q}\right)=0$. 
        
    %     (ii) Use (i) to conclude that if $A \subset \mathbf{R}$ has $\lambda(A)>0$, there is a nonmeasurable $S \subset A$. 
    % \end{exercise}

    \begin{exercise}{1.4.3}
        Let $g$ be an integrable function on $\mathbb{R}$ and $\varepsilon>0$. 
        
        (i), Use the definition of the integral to conclude there is a simple function 
        \[
            \varphi=\sum_{k} b_{k} 1_{A_{k}},\,\text{with}\int|g-\varphi| d x<\varepsilon. 
        \]
        
        (ii), Use Exercise A.2.1 to approximate the $A_{k}$ by finite unions of intervals to get a step function 
        \[
            q=\sum_{j=1}^{k} c_{j} 1_{\left(a_{j-1}, a_{j}\right)}
        \]
        with $a_{0}<a_{1}<\cdots<a_{k}$, so that $\int|\varphi-q|<\varepsilon$. 
        
        (iii), Round the corners of $q$ to get a continuous function $r$ so that $\int|q-r| d x<\varepsilon$. 

        (iv), To make a continuous function replace each $c_{j} \mathbf{1}_{\left(a_{j-1}, a_{j}\right)}$ by a function that is $0$ on $\left(a_{j-1}, a_{j}\right)^{c}$, $c_{j}$ on $\left[a_{j-1}+\delta-j, a_{j}-\delta_{j}\right]$, and linear otherwise. If the $\delta_{j}$ are small enough and we let $r(x)=\sum_{j=1}^{k} r_{j}(x)$, then
        \[
            \int|q(x)-r(x)| d \mu=\sum_{j=1}^{k} \delta_{j} c_{j}<\varepsilon. 
        \]
    \end{exercise}
    
    \begin{exercise}{1.4.4}
        Prove the Riemann-Lebesgue lemma. If $g$ is integrable, then 
        \[
            \lim _{n \rightarrow \infty} \int g(x) \cos n x d x=0. 
        \]
        Hint: If $g$ is a step function, this is easy. Now use the previous exercise. 
    \end{exercise}
    
\end{document}

