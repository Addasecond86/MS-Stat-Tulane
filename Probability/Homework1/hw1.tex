\documentclass[en, normal, 11pt, black]{elegantnote}

\usepackage{tcolorbox}



\newenvironment{exercise}[1]{\begin{tcolorbox}[title=#1]}{\end{tcolorbox}}

\newcommand{\solution}{\textbf{Solution. }}
\renewcommand{\proof}{\textbf{Proof. }}


%1.1.1, 1.1.3, 1.1.4, 1.1.5, 1.2.1, 1.2.4, 1.2.7, 1.3.1, 1.3.5

\begin{document}
% \maketitle
    \begin{exercise}{1.1.1}
        Let $\Omega=\mathbf{R}$, $\mathcal{F}=$ all subsets so that $A$ or $A^c$ is countable, $P(A) = 0$ in the first case and $= 1$ in the second. Show that $(\Omega,\mathcal{F},P)$ is a probability space.ffdcfcx
    \end{exercise}
    \proof

    \begin{exercise}{1.1.3}
        A $\sigma$-field $\mathcal{F}$ is said to be countably generated if there is a countable collection $\mathcal{C} \subset \mathcal{F}$ so that $\sigma(\mathcal{C}) = \mathcal{F}$. Show that $\mathcal{R}^d$ is countably generated. 
    \end{exercise}

    \begin{exercise}{1.1.4}
        (i) Show that if $\mathcal{F}_1\subset\mathcal{F}_2\subset$ are $\sigma$-algebras, then $\cup_i\mathcal{F}_i$ is an algebra. 
        
        (ii) Give an example to show that $\cup_i\mathcal{F}_i$ need not be a $\sigma$-algebra. 
    \end{exercise}

    \begin{exercise}{1.1.5}
        A set $A \subset \{1,2, \cdots\}$ is said to have asymptotic density $\theta$ if
        \[
        \lim _{n \rightarrow \infty}|A \cap\{1,2, \ldots, n\}| / n=\theta
        \]
        Let $\mathcal{A}$ be the collection of sets for which the asymptotic density exists. Is $\mathcal{A}$ a $\sigma$-algebra? an algebra?
    \end{exercise}

    \begin{exercise}{1.2.1}
        Suppose $X$ and $Y$ are random variables on $(\Omega, \mathcal{F}, P)$ and let $A \in \mathcal{F}$. Show that if we let $Z(\omega)=X(\omega)$ for $\omega \in A$ and $Z(\omega)=Y(\omega)$ for $\omega \in A^{c}$, then $Z$ is a random variable.
    \end{exercise}

    \begin{exercise}{1.2.4}
        Show that if $F(x)=P(X \leq x)$ is continuous, then $Y=F(X)$ has a uniform distribution on $(0,1)$, that is, if $y \in[0,1], P(Y \leq y)=y$.
    \end{exercise}

    \begin{exercise}{1.2.7}
        (i) Suppose $X$ has density function $f$. Compute the distribution function of $X^{2}$ and then differentiate to find its density function. 
        
        (ii) Work out the answer when $X$ has a standard normal distribution to find the density of the chi-square distribution. 
    \end{exercise}

    \begin{exercise}{1.3.1}
        Show that if $\mathcal{A}$ generates $\mathcal{S}$, then $X^{-1}(\mathcal{A}) \equiv\{\{X \in A\}: A \in \mathcal{A}\}$ generates $\sigma(X)=$ $\{\{X \in B\}: B \in \mathcal{S}\} .$
    \end{exercise}

    \begin{exercise}{1.3.5}
        A function $f$ is said to be lower semicontinuous or l.s.c. if
        $$
        \liminf _{y \rightarrow x} f(y) \geq f(x)
        $$
        and upper semicontinuous (u.s.c.) if $-f$ is l.s.c. Show that $f$ is l.s.c. if and only if $\{x: f(x) \leq a\}$ is closed for each $a \in \mathbf{R}$ and conclude that semicontinuous functions are measurable.
    \end{exercise}
    
\end{document}

% \title{Probability: Problem Set 1}
% \author{Zehao Wang}
% \date{\today}