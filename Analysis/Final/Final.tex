\documentclass[11pt]{article}

\usepackage[left=2cm, right=2cm, top=2cm, bottom=2cm]{geometry}

\usepackage{tcolorbox}
\tcbuselibrary{breakable}
\usepackage{amsfonts} 
\usepackage{amsmath}
\usepackage{amssymb}
\usepackage{newtxmath}
\usepackage{hyperref}

\linespread{1.3}
\setlength{\parskip}{3mm}
\setlength{\parindent}{2em}

\usepackage[backend=biber]{biblatex}
\addbibresource{my.bib}

\numberwithin{equation}{section}

% Problem 

\newcounter{exercise}[section]

\newenvironment{exercise}[1][\textsc{Exercise }\thesection.\refstepcounter{exercise}\theexercise]{\begin{tcolorbox}[colback=black!15, colframe=black!80, breakable, title=\textsc{Exercise }#1]}{\end{tcolorbox}}


% Example

\newcounter{example}[section]

\newenvironment{example}[1][\textsc{Example }\thesection.\refstepcounter{example}\theexample]{\begin{tcolorbox}[colback=black!15, colframe=black!80, breakable, title=#1]}{\end{tcolorbox}}

% Theorem

\newcounter{theorem}[section]

\newenvironment{theorem}[1][\textsc{Theorem }\thesection.\refstepcounter{theorem}\thetheorem]{\begin{tcolorbox}[colback=black!15, colframe=red!60, breakable, title=#1]}{\end{tcolorbox}}


\newenvironment{proof}{\begin{tcolorbox}[colback=white, colframe=black!50, breakable, title=Proof. ]\setlength{\parskip}{0.8em}}{\end{tcolorbox}}

\newenvironment{solution}{\begin{tcolorbox}[colback=white, colframe=black!50, breakable, title=Solution. ]\setlength{\parskip}{0.8em}}{\end{tcolorbox}}

\newcommand{\pder}{\partial\,}

\newcommand{\der}{\,\mathbf{d}\,}

\title{\textsc{Analysis: Final Exam}}
\author{\textsc{Zehao Wang}}
\date{\emph{\today}}

\begin{document}
    \maketitle

        \begin{exercise}[(1)]
            Let $\mathcal{A}$ be an infinite $\sigma$-algebra. Is it true that $\mathcal{A}$ contains an infinite sequence of disjoint sets? 
        \end{exercise}

        \begin{solution}
            \footfullcite{folland1999real}
            For an infinite $\sigma$-algebra $\mathcal{A}$, it's true that
            \vspace{-15pt}\begin{center}
                \emph{There is $ E\in\mathcal{A}$, such that $\{ E^c\cap F,  F\in\mathcal{A}\}$ is infinite. }
            \end{center}
            \vspace{-15pt}If not, pick any $\varnothing\neq E\in\mathcal{A}$, then $\{ E^c\cap F,  F\in\mathcal{A}\}$ is finite. And then $\{ E\cap F,  F\in\mathcal{A}\}$ must be finite, too. Because $\mathcal{A}$ can be generated by these two finite disjoint set $\{ E,  E^c\}$, $\mathcal{A}$ must be finite, too. 

            We say the restriction of $ E$ on $\mathcal{A}$ is $\{E\cap F, F\in \mathcal{A}\}:=E_{\mathcal{A}}$. We know if $E_\mathcal{A}$ is infinite,  $E_\mathcal{A}^c$ is infinite, too. So, we can build: $E_0 = \{E: E_\mathcal{A}\text{ is infinite}\}$; $E_1 = \{E: E_{\mathcal{A}\backslash E_0}\text{ is infinite}\}$; $ E_2= \{E: E_{\mathcal{A}\backslash E_0\backslash E_1}\text{ is infinite}\}$, $E_3=\cdots$. Because $\mathcal{A}$ is infinite, we can make sure that this sequence is infinite and every $E_i$, $E_j$ are disjoint. 
        \end{solution}
    
        \begin{exercise}[(2)]
            Let $(X, \mathcal{A}, \mu)$ be a measure space. Define $\rho(E, F)=\mu(E \Delta F)$. Prove that $\rho$ is a metric. 
        \end{exercise}

        \begin{proof}
            \begin{itemize}
                \item If $\rho(E,F)=\mu(E\Delta F)=0$, $\mu(E\Delta F)=0$, $E\Delta F=0$, $E=F$. 
                \item $E\Delta F=F\Delta E$. So, $\rho(E,F)=\rho(F,E)$. 
                \item $\rho(X,Z)+\rho(Z,Y)=\mu(X\Delta Z)+\mu(Z\Delta Y)=\mu((X\backslash Z)\cup (Z\backslash X))+\mu((Z\backslash Y)\cup (Y\backslash Z))\geqslant\mu(X\cup Z)-\mu(X\cap Z)+\mu(Y\cup Z)-\mu(Z\cup Y)\geqslant\mu((X\backslash Y)\cup(Y\backslash X))=\rho(X,Y)$. 
            \end{itemize}
            \vspace{-15pt}So, $\rho$ is a metric. 
        \end{proof}

        \begin{exercise}[(4)]
            Let $f: \mathbb{R}\to\mathbb{R}$ be monotone. Prove that $f$ is measurable. 
        \end{exercise}

        \begin{solution}
            If $f:\mathbb{R}\to\mathbb{R}$ is monotone, $f$ is continuous \emph{a.e.} Let $D$ be the set of all discontinuous point of $f$. Then $D$ is countable, hence the measure of $D$  is 0. So, $f$ on $D$ is measurable because every subset of $D$ is measurable. And $f$ on $\mathbb{R}\backslash D$ is measurable because it is continuous (continuous function is measurable). So, $f$ is measurable on $\mathbb{R}$. 
        \end{solution}
    
        \begin{exercise}[(5)] 
            Assume $f_{n}, g_{n}, f, g \in L^{1}$ and that $f_{n} \rightarrow f, g_{n} \rightarrow g$ \emph{a.e.} with $\left|f_{n}\right| \leq g_{n}$. Show that if $\int g_{n} \rightarrow \int g$, then $\int f_{n} \rightarrow \int f$. 
        \end{exercise}

        \begin{solution}
            From \emph{Fatou's lemma}, we know 
            \begin{align*}
                    \int(g-f)
                    &=\int \liminf _{n}\left(g_{n}-f_{n}\right) \\
                    &\leqslant \liminf _{n} \int\left(g_{n}-f_{n}\right)\\
                    &=\liminf _{n}\left(\int g_{n}-\int f_{n}\right) \\
                    &=\int g+\liminf _{n} \left(-\int f_{n}\right)\\
                    &=\int g-\limsup _{n} \left(\int f_{n}\right). 
            \end{align*}
            So, $\int f\geqslant\limsup_n\left(\int f_n\right)$. Similarly, on the other hand, 
            \begin{align*}
                \int(g+f)
                &=\int \liminf _{n}\left(g_{n}+f_{n}\right) \\
                &\leqslant \liminf _{n} \int\left(g_{n}+f_{n}\right)\\
                &=\int g+\liminf _{n} \left(\int f_{n}\right). 
            \end{align*}
            So, $\int f\leqslant\liminf _{n} \left(\int f_{n}\right)$. Because $\limsup\geqslant\liminf$, $\lim_{n\to\infty}\int f_n=\int f$. 
        \end{solution}
        
        \begin{exercise}[(16)]
            For $p \neq 2$, prove that the norm in $L^p$ does not come from an inner product in $L^p$. 
        \end{exercise}

        \begin{solution}
            Let $x=(1,1,0,\cdots)^T$, $y=(1,-1,0,\cdots)^T$. It is clear that 
            \[
                \|x+y\|_{p}=\|x-y\|_{p}=2, \quad\|x\|_{p}=\|y\|_{p}=2^{1 / p} .
            \]
            From the parallelogram law: 
            \[
            \|x+y\|^{2}+\|x-y\|^{2}=2\left(\|x\|^{2}+\|y\|^{2}\right). 
            \]
            Left side is $8$, and right side is $4(2^{2 / p})$. So, if $p \neq 2$, the parallelogram law fails, i.e. the norm $\|\cdot\|_{p}$ is not induced by an inner product. 
        \end{solution}
    

    \printbibliography
\end{document}
