\documentclass[12pt]{article}

\usepackage[left=2cm, right=2cm, top=2cm, bottom=2cm]{geometry}

% \linespread{1.5}

\usepackage{tcolorbox}
\tcbuselibrary{breakable}
\usepackage{amsfonts} 
\usepackage{amsmath}
% \usepackage{newtxtext}
% \usepackage{ulem}
\usepackage{amssymb}
% \usepackage[backend=biber]{biblatex}
% \addbibresource{my.bib}
\usepackage{newtxmath}
\usepackage{hyperref}

\linespread{1.3}
\setlength{\parskip}{3mm}
\setlength{\parindent}{2em}


\newenvironment{exercise}[1]{\begin{tcolorbox}[colback=black!15, colframe=black!80, breakable, title=#1]}{\end{tcolorbox}}

\newenvironment{proof}{\begin{tcolorbox}[colback=white, colframe=black!50, breakable, title=Proof. ]\setlength{\parskip}{0.8em}}{\,\\\rightline{$\square$}\end{tcolorbox}}

\newenvironment{solution}{\begin{tcolorbox}[colback=white, colframe=black!50, breakable, title=Solution. ]\setlength{\parskip}{0.8em}}{\end{tcolorbox}}

\newcommand{\pder}{\partial\,}

\newcommand{\der}{\,\mathbf{d}}

\title{\textsc{Analysis: Homework 4}}
\author{\textsc{Zehao Wang}}
\date{\emph{\today}}

% \vspace{-30pt}

\begin{document}
    \maketitle

    \begin{exercise}{1}
        Evaluate the limit
        \[
            \lim_{n\to \infty}\int_0^\infty\frac{\sin\left(\frac{x}{n}\right)}{x+x^2}\der x. 
        \]
    \end{exercise}

    \begin{solution}
        At first, I want to find a dominated function. However, it doesn't exist on $(0,\infty)$. So, I divide $(0,\infty)$ in two parts: $(0,1]$ and $(1,\infty)$. 

        For $x\in (0,1]$, 
        \[
            \begin{aligned}
                \int_0^1\frac{\sin\left(\frac{x}{n}\right)}{x+x^2}\der x & \leqslant\int_0^1\left|\frac{\sin\left(\frac{x}{n}\right)}{x+x^2}\right|\der x\\
                & \leqslant\int_0^1\frac{x/n}{x+x^2}\der x\\
                & \leqslant\frac{1}{n}\int_0^1\frac{1}{1+x}\der x\\
                & =\frac{1}{n}\ln(2). 
            \end{aligned}
        \]
        Let $n\to\infty$, 
        \[
            \begin{aligned}
                \lim_{n\to\infty}\int_0^1\frac{\sin\left(\frac{x}{n}\right)}{x+x^2}\der x \leqslant\lim_{n\to\infty}\frac{1}{n}\ln(2)=0. 
            \end{aligned}
        \]
        For $x\in(1,\infty)$, $\left|\frac{\sin\left(\frac{x}{n}\right)}{x+x^2}\right|\leqslant\frac{1}{x^2}$. And we have
        \[
            \int_1^\infty\frac{1}{x^2}=-x^{-1}\big|_1^\infty=1<\infty. 
        \]
        So, 
        \[
            \lim_{n\to\infty}\int_1^\infty\frac{\sin\left(\frac{x}{n}\right)}{x+x^2}\der x =\int_1^\infty\lim_{n\to\infty}\frac{\sin\left(\frac{x}{n}\right)}{x+x^2}\der x=\int_1^\infty\frac{0}{x+x^2}\der x=0. 
        \]
        Then, we can get
        \[
            \lim_{n\to \infty}\int_0^\infty\frac{\sin\left(\frac{x}{n}\right)}{x+x^2}\der x=\lim_{n\to \infty}\int_0^1\frac{\sin\left(\frac{x}{n}\right)}{x+x^2}\der x+\lim_{n\to \infty}\int_1^\infty\frac{\sin\left(\frac{x}{n}\right)}{x+x^2}\der x=0. 
        \]
    \end{solution}

    \begin{exercise}{2}
        Let $f\in L^1(X,\mathscr{S},\mu)$. Prove that
        \[
            \lim_{n\to\infty}n\mu(\{x\in X: |f(x)|\geqslant n\})=0. 
        \]
        
    \end{exercise}

    \begin{proof}
        Define
        \[
            A_n=\{x\in X: n\leqslant|f(x)|< n+1\}. 
        \]
        Then, 
        \[
            \sum_{n=0}^\infty\int_{A_n}|f|\der \mu=\int_X|f|\der \mu<\infty. 
        \]
        So, for any $\varepsilon>0$, there is some large enough $n$, such that $\sum_{k=n}^\infty\int_{A_k}|f|\der \mu<\varepsilon$. In other words, 
        \[
            \lim_{n\to\infty}\sum_{k=n}^\infty\int_{A_k}|f|\der \mu=0. 
        \]
        Because $|f|\geqslant k$ on $A_k$, we can get 
        \[
            \begin{aligned}
                \lim_{n\to\infty}n\mu(\{x\in X: n\leqslant|f(x)|\})
                \leqslant&\lim_{n\to\infty}n\sum_{k=n}^\infty\mu(A_k)\\
                \leqslant&\lim_{n\to\infty}\sum_{k=n}^\infty k\mu(A_k)\\
                \leqslant&\lim_{n\to\infty}\sum_{k=n}^\infty\int_{A_k}|f|\der \mu=0. 
            \end{aligned}
        \]
        \vspace*{-30pt}
    \end{proof}

    \begin{exercise}{3}
        Define $f : \mathbb{R} \to \mathbb{R}$ by
        \[
            f(x)=\begin{cases}
                x^{1/2} & 0<x<1, \\
                0 & otherwise. 
            \end{cases}
        \]
        Is this function in $L^1(\mathbb{R})$? Try to use monotone convergence theorem to decide. 
        
        Let $\{r_n\}$ a list of the rational numbers. Define 
        \[
            g(x)=\sum_{n=1}^\infty\frac{1}{2^n}f(x-r_n). 
        \]
        Prove that $g \in L^1(\mathbb{R})$ even though is unbounded on any interval of $\mathbb{R}$. 
    \end{exercise}

    \begin{solution}
        Because
        \[
            \int_{\mathbb{R}}x^{1/2}\der x=\frac{2}{3}, 
        \]
        $f\in L^1$. \textcolor{blue}{use monotone convergence theorem for what? }
        
        For $g$, 
        \[
            \begin{aligned}
                \int_\mathbb{R}g(x)\der x&=\int_\mathbb{R}\sum_{n=1}^\infty\frac{1}{2^n}f(x-r_n) \der x \\
                &=\sum_{n=1}^\infty\frac{1}{2^n}\int_\mathbb{R}f(x-r_n)\der x \\
                &=\sum_{n=1}^\infty\frac{1}{2^n}\int_{r_n}^{r_n + 1}f(x-r_n)\der x \\
                &=\sum_{n=1}^\infty\frac{1}{2^n}\frac{2}{3} \\
                &=\frac{2}{3}<\infty. 
            \end{aligned}
        \]
        \textcolor{blue}{When will $g$ be unbounded? }
    \end{solution}

    \begin{exercise}{4}
        Let $f\in L^1(\mathbb{R})$. Prove that 
        \[
            \lim_{n\to\infty} \int_{-n}^{n} f\der \mu = 0, 
        \]
        where $\mu$ is Lebesgue measure. 
    \end{exercise}

    \begin{solution}
        There seems to be a mistake in this problem? 

        \textcolor{blue}{
            Counterexample, if $X\sim N(0,1)$, i.e. $f(x)=\frac {1}{ \sqrt {2\pi }}e^{-\frac {x^{2}}{2}}$. Then \[
                \lim_{n\to\infty} \int_{-n}^{n} f\der \mu = \int_{-\infty}^{\infty}f(x)\der x=1. 
            \]
        }
    \end{solution}

    \begin{exercise}{5}
        Examine the behavior of the integral of
        \[
            f_n(x)=\frac{1}{2n}\chi_{(n^2-n,n^2+n)}(x). 
        \]
    \end{exercise}

    \begin{solution}
        For $n=1$, 
        \[f_1(x)=\frac{1}{2}\chi_{(0,2)}(x). \]
        So, 
        \[
            \int_\mathbb{R}\frac{1}{2}\chi_{(0,2)}(x)\der x = 1. 
        \]
        And we can find that for any fixed $n>0$, 
        \[
            \int_\mathbb{R}\frac{1}{2n}\chi_{(n^2-n,n^2+n)}(x)\der x = 1. 
        \]
        Then, 
        \[
            \lim_{n\to\infty}\int_\mathbb{R}\frac{1}{2n}\chi_{(n^2-n,n^2+n)}(x)\der x = 1\not=0=\int_\mathbb{R}\lim_{n\to\infty}\frac{1}{2n}\chi_{(n^2-n,n^2+n)}(x)\der x . 
        \]
    \end{solution}

    \begin{exercise}{6}
        Evaluate 
        \[
            \lim_{n\to\infty}\int_1^\infty\frac{e^{-nx}}{x}\der x. 
        \]
    \end{exercise}

    \begin{solution}
        Assume $n\in\mathbb{Z}^+$. And because $x\in(1,\infty)$, 
        \[
            \frac{e^{-nx}}{x}\leqslant e^{-nx}\leqslant e^{-x}, 
        \]
        Considering $\int_1^\infty e^{-x}=\frac{1}{e}$, then 
        \[
            \lim_{n\to\infty}\int_1^\infty\frac{e^{-nx}}{x}\der x=\int_1^\infty\lim_{n\to\infty}\frac{e^{-nx}}{x}\der x=0. 
        \]
    \end{solution}

    \begin{exercise}{7}
        Let $f : \mathbb{R} \to [0, \infty)$ be a measurable function. Is
        \[
            \lim_{n\to\infty}\int_{-n}^nf\der \mu=\int_\mathbb{R}f\der \mu
        \]
        valid? 
    \end{exercise}

    \begin{solution}
        % From exercise 2, we know 
        % \[
        %     \lim_{k\to\infty}k\mu(\{x\in \mathbb{R}: |f(x)|\geqslant k\})=0. 
        % \]
        Define
        \[
            A_k=\{x\in \mathbb{R}: k - 1 \leqslant |f(x)| < k\}. 
        \]
        Then, 
        \[
            \int_\mathbb{R}f\der \mu=\sum_{k = 1}^\infty\int_{A_k}f\der \mu. 
        \]
        Let $B_k=[-n, n]\cap A_k$. Then, we have
        \[
            \begin{aligned}
                \int_{-n}^{n} f\der \mu &= \sum_{k=1}^\infty\int_{B_k}f\der \mu
                % \\
                % &\leqslant\sum_{k=1}^\infty\int_{B_k}k\der \mu\\
                % &\leqslant\sum_{k=1}^\infty k \mu(B_k). 
            \end{aligned}
        \]
        % Similarly, 
        % \[
        %     \int_{-n}^{n} f\der \mu \geqslant \sum_{k=1}^\infty(k-1)\mu(B_k). 
        % \]
        Let $n\to\infty$, $[-n,n]=\mathbb{R}$, $B_k=\mathbb{R}\cap A_k=A_k$. So, 
        \[
            \lim_{n\to\infty}\int_{-n}^{n} f\der \mu = \sum_{k=1}^\infty\int_{B_k}f\der \mu=\sum_{k = 1}^\infty\int_{A_k}f\der \mu=\int_\mathbb{R}f\der \mu. 
        \]
    \end{solution}

    \begin{exercise}{8}
        Let $f\geqslant0$, $f(0)=0$, $f'(0)<\infty$. Assume $f\in L^1([0,\infty))$. Prove
        \[
            \int_0^\infty\frac{f(x)}{x}\der x<\infty. 
        \]
    \end{exercise}

    \begin{solution}
        For any $0<\varepsilon<1$, because $f\in L^1([0,\infty))$, 
        \[
            \int_\varepsilon^\infty\frac{f(x)}{x}\der x<\frac{1}{\varepsilon}\int_\varepsilon^\infty f(x)\der x<\infty, 
        \]
        We only need to prove 
        \[
            \int_0^\varepsilon\frac{f(x)}{x}\der x<\infty. 
        \]
        Rewrite $f(x)$ at $0$ using Taylor formula, 
        \[
            f(x)=f(0)+f'(0)x+o(x^2). 
        \]
        Then
        \[
            \frac{f(x)}{x}=f'(0)+\frac{o(x^2)}{x}. 
        \]
        Let $\varepsilon\to 0$, then $\frac{o(x^2)}{x}=0$. So, 
        \[
            \frac{f(x)}{x}=f'(0)<\infty. 
        \]
        Finally, we can get
        \[
            \int_0^\infty\frac{f(x)}{x}\der x<=\int_0^\varepsilon\frac{f(x)}{x}\der x+\int_\varepsilon^\infty\frac{f(x)}{x}\der x<\infty. 
        \]
    \end{solution}

    \begin{exercise}{9}
        Prove that 
        \[
            \lim_{n\to \infty}\int_0^1 n\ln(1+\frac{|f(x)|^2}{n^2})\der x=0. 
        \]
    \end{exercise}

    \begin{solution}
        We know $|f(x)|^2/n^2$ is nonnegative, thus $\ln(1+|f(x)|^2/n^2)\leqslant |f(x)|^2/n^2$. SO, we have
        \[
            \int_0^1n\ln(1+\frac{|f(x)|^2}{n^2})\der x\leqslant \int_0^1n\frac{|f(x)|^2}{n^2}\der x=\int_0^1\frac{|f(x)|^2}{n}\der x
        \]
        We assume $f(x)\in L^1$, then $\int_0^1|f(x)|\der x<\infty$. And $|f(x)|^2$ is bounded on $(0,1)$. So, 
        \[
            \lim_{n\to \infty}\int_0^1 n\ln(1+\frac{|f(x)|^2}{n^2})\der x=\int_0^1\lim_{n\to \infty}\frac{|f(x)|^2}{n}\der x=0. 
        \]
    \end{solution}

    \begin{exercise}{10}
        Let $f \in L^1(\mathbb{R})$ and $p > 0$ a fixed number. Prove that
        \[
            \lim_{n\to\infty}\frac{f(nx)}{n^p}=0
        \]
        almost everywhere in $x \in \mathbb{R}$. This means that the set where this does not happen has measure 0. 
    \end{exercise}

    \begin{solution}
        Because $f\in L^1(\mathbb{R})$, $f$ is bounded $a.e$. That means there may be a set $E$ with $\mu(E)=0$, on which $f$ can be unbounded. Let 
        \[
            M=\max_{\mathbb{R}\backslash E}\{f\}<\infty, 
        \]
        then
        \[
            \lim_{n\to\infty}\frac{f(nx)}{n^p}\leqslant\lim_{n\to\infty}\frac{M}{n^p}=0, 
        \]
        So, 
        \[
            \lim_{n\to\infty}\frac{f(nx)}{n^p}=0, \qquad a.e. 
        \]
    \end{solution}

    \begin{exercise}{11}
        Let $f \in L^1(\mathbb{R})$. Prove that if the integral of $f$ over every interval with rational endpoints, then $f = 0$ almost everywhere. 
    \end{exercise}

    \begin{solution}
        \textcolor{blue}{I think the problem should be like this? }
        
        Let $f \in L^1(\mathbb{R})$. Prove that if the integral of $f$ over every interval with rational endpoints \textcolor{red}{is $0$}, then $f = 0$ almost everywhere. 
        
        Suppose there is a set $E\subset \mathbb{R}$ satisfy $f|_E\not=0$. $E$ contains two types of elements: 1) lots of single points, denoted by $P=\{p_i\}$ and 2) some intervals, denoted by $I=\{[a_i,b_i]\}$. 

        1) single points. Because $\mathbb{Q}$ is dense in $\mathbb{R}$, for any point $p_i$, there must be an rational interval contains it. So, $\sum_i\mu(p_i)$ has to be $0$. 

        2) some intervals. Also because $\mathbb{Q}$ is dense in $\mathbb{R}$, there must be an rational interval contains $\cup_i[a_i,b_i]$. Hence, $\mu(\cup_i[a_i,b_i])$ has to be $0$, as well. 

        Finally, we know that \[\mu(E)=\mu(P)+\mu(I)\leqslant0, \]
        i.e. $f=0,\quad a.e$. 
    \end{solution}

    \begin{exercise}{12}
        Let $(X, \mathscr{S}, \mu)$ be a measure space and $\rho : X \to [0, \infty]$ be a measurable function. For $A \in \mathscr{S}$, define
        \[
            v(A)=\int_A\rho\der \mu. 
        \]
        Prove that $v$ is a measure and that for every $f: X\to [0;\infty]$ measurable the formula
        \[
            \int_Xf\der v=\int_Xf\rho\der\mu. 
        \]
    \end{exercise}

    \begin{solution}
        \textsc{1. Non-negativity: }

        Because $\rho\geqslant 0$ and $\mu(A)\geqslant 0$, $v(A)$ is also not less than $0$. 

        \textsc{2. Null empty set: }

        Because $\mu(\varnothing)=0$, $v(\varnothing)=0$. 

        \textsc{3. Countable additivity: }

        Let $\{A_i\}\subset \mathscr{S}$, and is pairwise disjoint, then
        \[
            v(\cup_{i=1}^\infty A_i)=\int_{\cup_{i=1}^\infty A_i}\rho\der \mu=\sum_{i=1}^\infty\int_{A_i}\rho\der \mu=\sum_{i=1}^\infty v(A_i). 
        \]
        So, $v$ is a measure. 

        For $A\in\mathscr{S}$, $x\in A$, 
        \[
            \int_Af(x)\der(v(x))=\int_Af(x)\der(v(x))=\int_Af(x)\der(\int_A\rho(x)\der \mu(x))=\int_Af(x)\rho(x)\der \mu(x). 
        \]
    \end{solution}

    % \printbibliography
\end{document}
