\documentclass[en, normal, 11pt, black]{elegantnote}

\usepackage{tcolorbox}
\tcbuselibrary{breakable}
\usepackage{amsfonts}
\usepackage{newtxtext}
\usepackage{ulem}
\usepackage{amssymb}

\newenvironment{exercise}[1]{\begin{tcolorbox}[colback=black!15, colframe=black!80, breakable, title=#1]}{\end{tcolorbox}}

\renewenvironment{proof}{\begin{tcolorbox}[colback=white, colframe=black!50, breakable, title=Proof. ]\setlength{\parskip}{0.8em}}{\,\\\rightline{$\square$}\end{tcolorbox}}

\newenvironment{solution}{\begin{tcolorbox}[colback=white, colframe=black!50, breakable, title=Solution. ]\setlength{\parskip}{0.8em}}{\end{tcolorbox}}

\newcommand{\pder}{\partial\,}

\newcommand{\der}{\,\mathbf{d}}

\title{\textsc{Analysis: Homework 1}}
\author{\textsc{Zehao Wang}}
\date{Sep 24, 2021}

\begin{document}
\maketitle
    \begin{exercise}{Picard's Theorem}
        For the Initial Value Problem: 
        \begin{equation*}
            \left\{
            \begin{aligned}
                y^\prime(x)&=F(x,y) \\
                y(x_0)&=y_0. 
            \end{aligned}
            \right.
        \end{equation*}
        Prove the existence and uniqueness of the solution on \[R=\{(x,y): |x-x_0|\leqslant a, |y-y_0|\leqslant b\}. \] 
        Assume: 
        \begin{itemize}
            \item $F:D\subset \mathbb{R}^2\to\mathbb{R}$ is Lipschitz continuous for $y$ with constant $\delta$, which means for $y_1$, $y_2$: 
            \[\left|F(x,y_1)-F(x,y_2)\right|\leqslant\delta\left|y_1-y_2\right|. \]
            \item $F$ is bounded: 
            \[\forall\,(x,y) \in D,\,\exists\,M\in \mathbb{R},\,s.t.\,\left|F(x,y)\right|\leqslant M. \]
            \item $\delta a<1$. 
        \end{itemize}
    \end{exercise}
    \begin{proof}
        \textsc{Existence}, we construct the iteration as following: 
        \begin{equation*}
            \left\{
                \begin{aligned}
                    y_0(x)&=y_0, \\
                    y_{n+1}(x)&=y_0+\int_{x_0}^xF(s, y_{n}(s))\der s. 
                \end{aligned}
            \right.
        \end{equation*}
        We can know that $\{y_n\}$ is differentiable. Then, we need to prove the limit exists. 
        \begin{align*}
            |y_1(x)-y_0(x)|\leqslant & \left|\int_{x_0}^x F(s, y_0(s))\der s\right|\\
            \leqslant & M|x-x_0|\\
            \leqslant & Ma
        \end{align*}
        \begin{align*}
            |y_2(x)-y_1(x)|\leqslant & \left|\int_{x_0}^x F(s, y_1(s))-F(s, y_0(s))\der s\right|\\
            \leqslant & \left|\delta\int_{x_0}^x |y_1(s)-y_0(s)|\der s\right|\\
            \leqslant & \left|\delta\int_{x_0}^x \int_{x_0}^{s_0} |F(s_1,y_0(s_1)|\der s_1\right|\\
            \leqslant & \delta M\frac{|x-x_0|^2}{2!}\\
            \leqslant & \frac{a^2}{2!}\delta M. 
        \end{align*}
        So, for $|y_n(x)-y_{n-1}(x)|$, we have: 
        \begin{align*}
            |y_n(x)-y_{n-1}(x)| \leqslant & \left|\int_{x_0}^x F(s, y_{n-1}(s))-F(s, y_{n-2}(s))\der s\right|\\
            \leqslant & \left|\delta^{n-1}\int_{x_0}^x\int_{x_0}^{s_0}\cdots\int_{x_0}^{s_{n-2}} |F(s_{n-1}, y_0(s_{n-1}))|\der s_{n-1}\right|\\
            \leqslant & \delta^{n-1} M\frac{|x-x_0|^n}{n!}\\
            \leqslant & \frac{a^n}{n!}\delta^{n-1} M. 
        \end{align*}
        Since $\delta a<1$, then, $\forall\,m>n>0$, we can get: 
        \begin{align*}
            |y_m(x)-y_{n}(x)| \leqslant & |y_m(x)-y_{m-1}(x)|+\cdots+|y_{n+1}(x)-y_{n}(x)|\\
            \leqslant & \delta^{n-1} M \sum_{i=n}^m \frac{a^n}{n!}. 
        \end{align*}
        When $n\to \infty$, $|y_m(x)-y_{n}(x)|\to 0$. So, $\{y_n\}$ is a cauchy sequence, and it is uniformly convergent. 
        
        Let $\lim_{n\to\infty}y_n=y$, then: 
        \[y(x)=y_0+\int_{x_0}^xF(s,y(s))\der s, \]
        And 
        \begin{equation*}
            \left\{
                \begin{aligned}
                    y(x)&=y_0, \\
                    y^\prime(x)&=F(x, y(x)). 
                \end{aligned}
            \right.
        \end{equation*}
        Thus, $y$ is the solution of that Initial Value Problem. 
    \end{proof}
    \begin{proof}
        \textsc{Uniqueness}, if $y_1$, $y_2$ are both the solution of the problem, then, we have: 
        \[y_1(x)=y_0+\int_{x_0}^xF(s,y_1(s))\der s, \]
        \[y_2(x)=y_0+\int_{x_0}^xF(s,y_2(s))\der s, \]
        Then, 
        \begin{align*}
            |y_1(x)-y_2(x)|&=\left|\int_{x_0}^xF(s,y_1(s))-F(s,y_2(s))\der s\right|\\
            &\leqslant\left|\delta\int_{x_0}^x|y_1(s)-y_2(s)|\der s\right|\\
            &\leqslant\delta |x-x_0| \max_{s\in [x_0,x_0+h]}|y_1(s)-y_2(s)|\\
            &\leqslant \delta a \max_{s\in [x_0,x_0+a]}|y_1(s)-y_2(s)|. 
        \end{align*}
        We can write above inequation as: 
        \[\max_{x\in [x_0, x_0+a]}|y_1(x)-y_2(x)|\leqslant \delta a \max_{s\in [x_0,x_0+a]}|y_1(s)-y_2(s)|, \]
        So, $\max_{x\in [x_0, x_0+a]}|y_1(x)-y_2(x)|$ can only be $0$, which means $y_1=y_2$. 
    \end{proof}
    \begin{exercise}{\textsc{Exercise. }1}
        \[y^\prime=1+y^2.\]
    \end{exercise}
    \begin{solution}
        1. 
        \begin{align*}
            y_0(x)&=y_0, \\
            y_1(x)&=y_0+\int_{x_0}^x F(s, y_0(s))\der s\\
            &=y_0+\int_{x_0}^x(1+y_0^2)\der s\\
            &=y_0+(1+y_0^2)(x-x_0), \\
            y_2(x)&=y_0+\int_{x_0}^x F(s, y_1(s))\der s\\
            &=y_0+\int_{x_0}^x1+y_1(s)^2\der s\\
            &=y_0+\int_{x_0}^x1+(y_0+(1+y_0^2)(s-x_0))^2\der s\\
            &=y_0+(1+y_0^2)(x-x_0)+y_0(1+y_0^2)(x-x_0)^2+1/3(1+y_0^2)^2(x-x_0)^3, \\
            \cdots\cdots&\\
        \end{align*}
        This integral calculation is more tedious for larger $n$. However, it can still be seen that the first few terms fit the Taylor expansion of $\tan(x)$ at $x_0$. 
        
        2. 
        \[\frac{\der y}{1+y^2}=\der x\]
        \[\arctan(y)=x+c\]
        For generality, we can write $y$ as: 
        \[y(x)=\tan(x+c), \]
        If we know the initial value $y(x_0)=y_0$, then
        \[c=\arctan(y_0)-x_0. \]
    \end{solution}
    \begin{exercise}{\textsc{Exercise. }2}
        \[y^\prime=\sqrt{|y|}. \]
    \end{exercise}
    \begin{solution}
        If $y>0$, 
        \[y^{-1/2}\der y=\der x\]
        \[2y^{1/2}=x+c\]
        \[y=\frac{1}{4}(x+c)^2\]
        If we know $y(x_0)=y_0$, then
        \[c=2\sqrt{y_0}-x_0. \]
        If $y<0$, 
        \[(-y)^{-1/2}\der y=\der x\]
        \[2(-y)^{1/2}=x+c\]
        \[y=-\frac{1}{4}(x+c)^2\]
        If we know $y(x_0)=y_0$, then
        \[c=2\sqrt{-y_0}-x_0. \]
    \end{solution}
\end{document}

