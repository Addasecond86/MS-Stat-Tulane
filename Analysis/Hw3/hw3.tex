\documentclass[12pt]{article}

\usepackage[left=2cm, right=2cm, top=2cm, bottom=2cm]{geometry}

% \linespread{1.5}

\usepackage{tcolorbox}
\tcbuselibrary{breakable}
\usepackage{amsfonts} 
\usepackage{amsmath}
\usepackage{newtxtext}
% \usepackage{ulem}
\usepackage{amssymb}
\usepackage[backend=biber]{biblatex}
\addbibresource{my.bib}
\usepackage{newtxmath}
\usepackage{hyperref}

\linespread{1.3}
\setlength{\parskip}{3mm}
\setlength{\parindent}{2em}


\newenvironment{exercise}[1]{\begin{tcolorbox}[colback=black!15, colframe=black!80, breakable, title=#1]}{\end{tcolorbox}}

\newenvironment{proof}{\begin{tcolorbox}[colback=white, colframe=black!50, breakable, title=Proof. ]\setlength{\parskip}{0.8em}}{\,\\\rightline{$\square$}\end{tcolorbox}}

\newenvironment{solution}{\begin{tcolorbox}[colback=white, colframe=black!50, breakable, title=Solution. ]\setlength{\parskip}{0.8em}}{\end{tcolorbox}}

\newcommand{\pder}{\partial\,}

\newcommand{\der}{\,\mathbf{d}}

\title{\textsc{Analysis: Homework 3}}
\author{\textsc{Zehao Wang}}
\date{Oct 20, 2021}

% \vspace{-30pt}

\begin{document}
    \maketitle

    \begin{exercise}{1}
        Let $X$ be a finite set with $n$ elements. Find the number of $\sigma$-algebras in $X$. Give a combinatorial description of this problem. 
    \end{exercise}

    \begin{solution}
        For a $n$-element set $X$, its $\sigma$-algebra can be generated from partitions of $X$. So, we need to know the number of different ways to partition the set $X$. And the number of different ways to partition a set is called \emph{Bell Number}\footfullcite{bellnumber}. Bell number of $n$-element set $X$ is denoted $B_n$. For an example, if $X=\{a,b,c\}$, then $B_3=5$. Because
        \[\{(a), (b,c)\},\qquad\{(b), (a,c)\},\qquad\{(c), (a,b)\}, \qquad\{(a,b,c)\},\qquad\{(a), (b), (c)\}\]
        And the union of any above set with $\{\varnothing, X\}$ would be a $\sigma$-algebra. 
        Hence, the number of $\sigma$-algebras in $n$-element set $X$ is $B_n$, which can be calculate by $B_n=\sum_{i=0}^{n-1}B_i\binom{n-1}{k}$. 
    \end{solution}

    \begin{exercise}{2}
        Let $E\in\mathbb{R}$ be a measurable set. Prove that 
        \[E=\left(\bigcup_{k=1}^\infty F_k\right)\cup G\]
        where each $F_k$ is compact, and $\mu(G)=0$. Here $\mu$ is Lebesgue measure. 
    \end{exercise}

    \begin{proof}
        If $E$ is Lebesgue measurable, for any $\varepsilon>0$, there exists a set $N$, such that
        \[
            \mu(E-N)<\varepsilon, 
        \]
        Let $B_n$ be a closed ball with radius $n$ and center at point $0$. So, $F_k=N\cap B_k$ is closed, and it is compact. And $F_k\uparrow N$, $E-F_k\downarrow E-N$. Hence, for a large $N$, if $k>N$, then $\mu(E-F_k)<\varepsilon$. In this case, let $G=E-\bigcup_{k=1}^\infty F_k=E-N$, $\mu(G)=0$. 
    \end{proof}

    \begin{exercise}{3}
        Suppose $f$ is a measurable function defined on $E\subset\mathbb{R}$ with $\mu(E) < \infty$ and suppose $f$ is finite except on a subset $F \subset E$ with $\mu(F) = 0$. Show that for any $\varepsilon > 0$, there exists a bounded function $g$, defined on $E$, such that
        \[
            \mu(\{x\in E:|f(x)-g(x)|>0\})<\varepsilon. 
        \]
    \end{exercise}

    \begin{proof}
        $f$ is measurable, for $\{-\infty, -n\}\cup\{n,\infty\}\in\mathbb{R}$, define 
        \[A_n=f^{-1}(\{-\infty, -n\}\cup\{n,\infty\})\in E. \] 
        We can know $A_1\supset A_2\supset A_3\supset\cdots$. Because $\mu(E)<\infty$, $\mu(A_1)<\infty$. So, for any $n\in\mathbb{N}$, $E-A_n$ is bounded, and in this case, let $g=f$. 

        For the set $A_n$, $f$ is not bounded, but $\lim_{n\to\infty}\cap_nA_n$ is actually the empty set. So, \[\lim_{n\to\infty}\mu(\cap_nA_n)=0. \]
        Let $A=\lim_{n\to\infty}\cap_nA_n$, $\mu(A)=0$. So, for any $\varepsilon>0$, $\mu(\{x\in A:|f(x)-g(x)|>0\})=0<\varepsilon$. 

        Got some hints from this website\footfullcite{2304794}. 
    \end{proof}

    \begin{exercise}{4}
        Let $(X; \mathscr{S}; \mu)$ be a measure space and $f : (X; \mathscr{S}) \to (Y; \mathscr{T} )$ be a measurable function (define what this is). Define a function on subsets of $Y$ by using a reasonable rule (Think about how you would bring the measure from $X$ to $Y$. This process is called the measure induced by $f$). Discuss the relation of this problem with the notion of the distribution of a random variable. 
    \end{exercise}

    \begin{solution}
        We want to transfer the measure $\mu$ on measurable space $(X,\mathscr{S})$ to another measurable space $(Y,\mathscr{T})$ by using a measurable function $f$. Assume $m: \mathscr{T}\to [0,\infty]$ is the measure on $(Y,\mathscr{T})$. It must satisfy: 
        \[m(X)=\mu(f^{-1}(X)),\quad \forall\, X\in\mathscr{T}. \]
        For a random variable, it is a pushforward measure (Prof. Nathan talked about it in probability class). It first maps the probability space into a space, and then give it a probability measure defined by this called pushforward. 
    \end{solution}

    \begin{exercise}{5}
        A \texttt{dyadic interval} is one of the form
        \[\left(\frac{k}{2^n}, \frac{k+1}{2^n}\right], \]
        where $k, n$ are integers. Prove that every open subset of $\mathbb{R}$ is a countable disjoint union of dyadic intervals. 
    \end{exercise}

    \begin{solution}
        How can the disjoint union of intervals like this $(a,b]$ be an open set like this $(c,d)$? I mean because of the disjoint union, the right side of the interval will always be closed, and it will never be an open set, because the right endpoints is not inner point. 
    \end{solution}

    \begin{exercise}{6}
        A measure space $(X; \mathscr{S}; \mu)$ is called complete if it contains every subset of measure $0$; that is, if $F \subset X$ and there exists a set $E \in \mathscr{S}$ with $E \subset F$ and $\mu(E) = 0$, then $F \in \mathscr{S}$. Now suppose $(X; \mathscr{S}; \mu)$ is a measure space and let
        \[\mathscr{N}=\{N\in\mathscr{S}:\mu(N)=0\}. \]
        Define
        \[\bar{\mathscr{S}}=\{E\cup F: E\in\mathscr{S}, F\subset N \text{ for some } N\subset \mathscr{N}\}. \]
        Define $\bar{\mu}: \bar{\mathscr{S}}\to[0,\infty]$ by
        \[\bar{\mu}(E\cup F)=\mu(E). \]
        Prove that $\bar{S}$ is a $\sigma$-algebra and $\bar{\mu}$ extends $\mu$ and $(\bar{\mathscr{S}}, \bar{\mu})$ is a complete measure space. 
    \end{exercise}

    \begin{solution}
        $\sigma$-algebra: 

        Let $X\in \bar{\mathscr{S}}$, Then $X=E\cup F$, where $E\in \mathscr{S}$ and $F\subset N$ for some $N\in\mathscr{N}$ with $\mu(N)=0$. Then we have: 
        \begin{align*} 
            X^c &= (E \cup F)^c \\ 
            &= (E \cup (N \setminus (N \setminus F)))^c \\ 
            &= (E \cup (N \cap (N \setminus F)^c))^c \\ 
            &= E^c \cap (N^c \cup (N \setminus F)) \\ 
            &= (E^c \cap N^c) \cup (E^c \cap (N \setminus F)) 
        \end{align*}
        Considering $E, N, F\in\mathscr{S}$, $E^c, N^c\in\mathscr{S}$. So, $E^c\cap N^c\in\mathscr{S}$, $(N\backslash F)\subset N$ with $\mu(N)=0$. Hence, $X^c\in \bar{\mathscr{S}}$. 

        If $\{S_i\}_i^\infty\in\bar{\mathscr{S}}$, then:
        \begin{align*} 
            \bigcup_{n=1}^{\infty} S_n &= \bigcup_{n=1}^{\infty} (E_n \cup F_n) \\ 
            &= \left ( \bigcup_{n=1}^{\infty} E_n \right ) \cup \left ( \bigcup_{n=1}^{\infty} F_n \right ) 
        \end{align*}
        And $\bigcup_{n=1}^{\infty} E_n\in\mathscr{S}$, $\bigcup_{n=1}^{\infty} F_n\subset N$. So, $ \bigcup_{n=1}^{\infty} S_n\in \mathscr{S}$. Plus $\{\varnothing, X\}\in\bar{\mathscr{S}}$, $\bar{\mathscr{S}}$ is $\sigma$-algebra. 

        $\bar{\mu}$ is a measure: 

        We only need to prove countable additivity. Let $\{S_n\}_{n=1}^\infty\in\bar{\mathscr{S}}$. Then, $S_n=E_n\cup F_n$. 
        \begin{align*} 
            \bar{\mu} \left ( \bigcup_{n=1}^{\infty} S_n \right ) = \mu \left ( \bigcup_{n=1}^{\infty} (E_n\cup F_n) \right ) = \mu \left ( \bigcup_{n=1}^{\infty} E_n \right ) =\sum_{n=1}^{\infty} \mu (E_n) = \sum_{n=1}^{\infty} \bar{\mu} (E_n). 
        \end{align*}
        So, $(X, \bar{\mathscr{S}}, \bar{\mu})$ is a measure space. 

        Complete: 

        For $N'\in\bar{\mathscr{S}}$, $\bar{\mu}(N')=0$, let $F'\subset N'$, we need to prove that $F'\in \bar{\mathscr{S}}$. Since $N'\in \bar{\mathscr{S}}$, we know $N'=E\cup F$ where $E\subset{\mathscr{S}}$, 
        $F\subset N$ with $\mu(N)=0$. And because $\bar{\mu}(N')=0$, $\mu(E)=0$. $F'=\varnothing\cup F'$, where $\varnothing\in\mathscr{S}$, $F'\subset N$ with $\mu(N)=0$. So, $F'\in\bar{\mathscr{S}}$. 
        
        I learn this from this website, it used an alternative definition of complete. \footfullcite{ababab}
        
    \end{solution}

    \begin{exercise}{7}
        Present a discussion of \texttt{the Caratheodory procedure} to define a measure from an \texttt{outer measure}. Compare your statements from the results pre-sented in class. 
    \end{exercise}

    \begin{solution}
        \emph{Given an outer measure $\mu$ on a set $X$, the collection $\mathscr{S}$ of measurable subsets of $X$ forms a $\sigma$-algebra. Moreover, $\mu$ restricted to $\mathscr{S}$ is a measure.}

        Because $\mathscr{S}$ contains all measurable subsets of $X$, for $A\in\mathscr{S}$, $A$ is measurable. So, for every set $E\subset X$, $\mu(E)=\mu(E\cap A)+\mu(E^c\cap A)$. Hence, $A^c$ is measurable and $A^c\in\mathscr{S}$. $\{\varnothing, X\}\in\mathscr{S}$. 

        Then, we want to prove that $\mathscr{S}$ is closed under countable disjoint union. We first prove a weaker result that $\mathscr{S}$ is closed under finite disjoint union. For $E_1,E_2\in\mathscr{S}$, take a set $A\subset X$, then using sub-additivity, we have: 
        \[
            \begin{aligned}
                \mu(A)&=\mu(E_1\cap X)+\mu(E_1^c\cap X)\\
                &=\mu(E_2\cap E_1\cap X)+\mu(E_2^c\cap E_1\cap X)+\mu(E_2\cap E_1^c\cap X)+\mu(E_2^c\cap E_1^c\cap X)\\
                &\geqslant\mu((E_2\cup E_1)\cap X)+\mu((E_2\cup E_1)^c\cap X), 
            \end{aligned}
        \]
        And clearly, $\mu(A)\leqslant\mu((E_2\cup E_1)\cap X)+\mu((E_2\cup E_1)^c\cap X)$, which means $E_2\cup E_1$ is also measurable. So, $E_2\cup E_1\in\mathscr{S}$. Then we have: 
        \[Y_n=\bigcup_{i=1}^nE_i\in\mathscr{S},\qquad Y=\bigcup_{i=1}^\infty E_i, \]
        We should prove $E\in\mathscr{S}$. For any set $A\subset X$: 
        \[
            \begin{aligned}
                \mu\left(Y_{n} \cap A\right) &=\mu\left(E_{n} \cap\left(Y_{n} \cap A\right)\right)+\mu\left(E_{n}^{c} \cap\left(Y_{n} \cap A\right)\right) \\
                &=\mu\left(E_{n} \cap A\right)+\mu\left(Y_{n-1} \cap A\right) \text{  (because $\{E_i\}$ are disjoint)}\\
                &=\mu\left(E_{n} \cap A\right)+\mu\left(E_{n-1} \cap A\right)+\mu\left(Y_{n-2} \cap A\right)\\
                &=\sum_{i=1}^{n} \mu\left(E_{i} \cap A\right). 
            \end{aligned}
        \]
        Because $Y_{n+1}^c\subset Y_{n}^c$, $Y^c\subset Y_{n}^c$, $\mu(Y^c)\leqslant\mu(Y_n^c)$, 
        \[
            \begin{aligned}
                \mu(A)&=\mu\left(Y_{n} \cap A\right)+\mu\left(Y_{n}^{c} \cap A\right) \\
                &=\sum_{i=1}^{n} \mu\left(E_{i} \cap A\right)+\mu\left(Y_{n}^{c} \cap A\right) \\
                &\geqslant \sum_{i=1}^{n} \mu\left(E_{i} \cap A\right)+\mu\left(Y^{c} \cap A\right)
            \end{aligned}
        \]
        Make $n$ tend to $\infty$, 
        \[
            \begin{aligned}
                \mu(A) &\geqslant \sum_{i=1}^{\infty} \mu\left(E_{i} \cap A\right)+\mu\left(Y^{c} \cap A\right)\\
                 & \geqslant \mu(Y \cap A)+\mu\left(Y^{c} \cap A\right) \\
                & \geqslant \mu(A), 
            \end{aligned}
        \]
        So, $Y\in\mathscr{S}$. And $\mu$ on $\mathscr{S}$ is a measure. 
    \end{solution}

    

    \begin{exercise}{8}
        Let $(X; \mathscr{A}; \mu)$ be a measure space. Assume $\mu(X) = 1$, so that this is a \texttt{probability space}. Take a collection of measurable sets $\{A_n\}$. Give a definition of the set of points $x \in X$ with the condition $x$ is \texttt{infinitely often} in $\{A_n\}$. Prove that this set is measurable. This is denoted by $A_n\ i.o.$ Prove that
        \[\text{if } \sum_k \mu(A_k)<\infty, \text{ then } \mu(A_n\ i.o.)=0. \]
        (This goes under the name \emph{the easy part of Borel-Cantelli.}) Read about the
        notion of \texttt{independence of sets} and discuss the converse of the previous
        statement.
    \end{exercise}

    \begin{solution}
        If $x$ is infinitely often, then the set is $\{x: x\in \cap_{j=1}^\infty \cup_{i=j}^\infty A_i\}$. This is the limit supremum of set, which means there will be infinite events. The intersection and union are countable, it is measurable. 

        Proof: 

        Let $\{E_n\}$ be a events sequence, and $\{\cup_i^\infty E_n\}_{i=1}^\infty$ is non-increasing. So, 
        \[
            \begin{aligned}
                \Pr(\cap_{j=1}^\infty\cup_{i=j}^\infty A_i)&=\lim_{j\to\infty} \Pr(\cup_{i=j}^\infty A_i)\\
                &\leqslant \sum_{i=j}^\infty \Pr(A_i)\\
                &\leqslant \sum_{i=1}^\infty \Pr(A_i)\leqslant \infty. 
            \end{aligned}
        \]
        So, $\sum_{i=1}^\infty \Pr(A_i)$ is bounded, and it converges. Then, 
        \[
            \lim_{j\to\infty} \Pr(\cup_{i=j}^\infty A_i)\leqslant \sum_{i=j}^\infty \Pr(A_i)\to 0. 
        \]

        Converse: 

        If events $\{E_{n}\}_{n=1}^{\infty }$ are independent, and $\sum _{n=1}^{\infty }\Pr(E_{n})=\infty$, then 
        \[
            \Pr(\limsup _{n\to \infty }E_{n})=1. 
        \]

        Proof: 

        We need to prove: 
        \[
            1 - \Pr (\limsup_{i \to \infty} E_i)=1-\Pr(\cap_{j=1}^\infty\cup_{i=j}^\infty E_i) = 0. 
        \]
        Noting that: 
        \[
            \begin{aligned}
                1-\Pr\left(\bigcap_{j=1}^\infty\bigcup_{i=j}^\infty E_i\right)
                &=\Pr \left(\left(\bigcap _{j=1}^{\infty }\bigcup _{i=j}^{\infty }E_{i}\right)^{c}\right)\\
                &=\Pr \left(\bigcup _{j=1}^{\infty }\bigcap _{i=j}^{\infty }E_{i}^{c}\right)\\
                &=\lim _{j\to \infty }\Pr \left(\bigcap _{i=j}^{\infty }E_{i}^{c}\right), 
            \end{aligned}
        \]
        Because $E_i$ are independent, $E_i^c$ are independent, as well. So, 
        \[
            \begin{aligned}
                \lim _{j\to \infty }\Pr \left(\bigcap _{i=j}^{\infty }E_{i}^{c}\right)=\lim_{j\to \infty }\prod_{i=j}^{\infty }(1-\Pr(E_i))\to 0. 
            \end{aligned}
        \]
        
    \end{solution}

    \begin{exercise}{9}
        Prove \texttt{Kolmogorov 0-1 law}: if $\{A_n\}$ is a sequence of independent sets. Then
        \[
            \mu(A_n\ i.o)=\begin{cases}
                0, & \sum_{n}\mu(A_n)<\infty, \\
                1, & \sum_{n}\mu(A_n)=\infty. 
            \end{cases}
        \]
    \end{exercise}
    \begin{solution}
        This is actually exercise 8. 
    \end{solution}

    \begin{exercise}{10}
        Give a construction of a measure space based on the idea of flipping a coin many times. 
    \end{exercise}

    \begin{solution}
        Suppose we toss a coin $n$ times, and denote HEAD, TAIL with $0,1$. So, the sample space is $\Omega=(0,1)^n$. 

        For the $\sigma$-algebra $\mathscr{F}$, any result of tossing a coin $n$ times belongs to $\mathscr{F}$. i.e. $\mathscr{F}=2^\Omega$. 

        For the measure $P$, let 
        \[P_i(w_i=0)=p,\ p\in(0,1),\ P_i(w_i=1)=1-p\]
        \[P(\omega=(\omega_1,\omega_2,\cdots,\omega_n))=\prod_{i=1}^nP_i(\omega_i)\]
        $P(\cdot)$ is a measure of $\omega\in\Omega$, and $(\Omega, \mathscr{F}, P)$ is a measure space. Futhermore, $P(\Omega)=1$, $(\Omega, \mathscr{F}, P)$ is a probability space. 
    \end{solution}

    % \printbibliography
\end{document}
