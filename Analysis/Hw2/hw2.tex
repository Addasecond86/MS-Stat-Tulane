\documentclass[en, normal, 11pt, black]{elegantnote}

\usepackage{tcolorbox}
\tcbuselibrary{breakable}
\usepackage{amsfonts} 
\usepackage{newtxtext}
\usepackage{ulem}
\usepackage{amssymb}

\newenvironment{exercise}[1]{\begin{tcolorbox}[colback=black!15, colframe=black!80, breakable, title=#1]}{\end{tcolorbox}}

\renewenvironment{proof}{\begin{tcolorbox}[colback=white, colframe=black!50, breakable, title=Proof. ]\setlength{\parskip}{0.8em}}{\,\\\rightline{$\square$}\end{tcolorbox}}

\newenvironment{solution}{\begin{tcolorbox}[colback=white, colframe=black!50, breakable, title=Solution. ]\setlength{\parskip}{0.8em}}{\end{tcolorbox}}

\newcommand{\pder}{\partial\,}

\newcommand{\der}{\,\mathbf{d}}

\title{\textsc{Analysis: Homework 2}}
\author{\textsc{Zehao Wang}}
\date{\today}

% \vspace{-30pt}

\begin{document}
\maketitle

\begin{exercise}{(1), }
    Let $f, g:[a, b] \rightarrow \mathbb{R}$ be Riemann integrable. Prove that $f+g$ is also Riemann integrable and
    \[
        \int_{a}^{b}(f(x)+g(x))\der x=\int_{a}^{b} f(x)\der x+\int_{a}^{b} g(x)\der x. 
    \]
\end{exercise}
\begin{proof}
    Because $f$ and $g$ are integrable, there exist $I(f), I(g)\in\mathbb{R}$, such that
    \[
        \int_{a}^{b} f(x)\der x=I(f)=\lim_{n\to\infty}\sum_{i=1}^{n}f(\varepsilon_i)(x_{i-1}, x_i)
    \]
    \[
        \int_{a}^{b} g(x)\der x=I(g)=\lim_{n\to\infty}\sum_{i=1}^{n}g(\varepsilon_i)(x_{i-1}, x_i)
    \]
    where $\varepsilon_i\in(x_{i-1}, x_i)$. So, we need to evaluate 
    \[
        \lim_{n\to\infty}\left(\sum_{i=1}^{n}\left(f(\varepsilon_i)+g(\varepsilon_i)\right)(x_{i-1}, x_i)-\sum_{i=1}^{n}f(\varepsilon_i)(x_{i-1}, x_i)-\sum_{i=1}^{n}g(\varepsilon_i)(x_{i-1}, x_i)\right). 
    \]
    \begin{align*}
        &\left|\sum_{i=1}^{n}\left(f(\varepsilon_i)+g(\varepsilon_i)\right)(x_{i-1}, x_i)-\sum_{i=1}^{n}f(\varepsilon_i)(x_{i-1}, x_i)-\sum_{i=1}^{n}g(\varepsilon_i)(x_{i-1}, x_i)\right|\\
        \leqslant&\left|\sum_{i=1}^{n}f(\varepsilon_i)(x_{i-1}, x_i)-f(\varepsilon_i)(x_{i-1}, x_i)\right|+\left|\sum_{i=1}^{n}g(\varepsilon_i)(x_{i-1}, x_i)-g(\varepsilon_i)(x_{i-1}, x_i)\right|\\
        =&0. 
    \end{align*}
    So, $f+g$ is integrable, and
    \[
        \int_{a}^{b}(f(x)+g(x))\der x=I(f)+I(g)=\int_{a}^{b} f(x)\der x+\int_{a}^{b} g(x)\der x. 
    \]
    \vspace{-30pt}
\end{proof}

\begin{exercise}{(2), }
    Suppose $f:[a, b] \rightarrow \mathbb{R}$ is Riemann integrable. Define 
    \[
        F(t)= \begin{cases}0, & t=a, \\ \int_{a}^{t} f, & t \in(a, b]. \end{cases}
    \]
    Prove that $F$ is a continuous function. 
\end{exercise}
\begin{proof}
    Because $f$ is integrable, $f$ is continuous on $(a,b]$. And
    \[\lim_{x\to a}F(x)=\lim_{x\to a}\int_{a}^{x}f(t)\der t=\int_a^af(x)\der x=0=F(a). \]
    So, $F$ is continuous. 
\end{proof}

\begin{exercise}{(3), }
    Let $\mathbb{Q}=\left\{r_{j}\right\} .$ Define a function 
    \[
        f_{n}(x)= \begin{cases}\left(x-r_{n}\right)^{-1 / 2}, & \text { if } x>r_{n}, \\ 0, & \text { if } x \leq r_{n}. \end{cases}
    \]
    Then define $f:[0,1] \rightarrow[0, \infty]$ by
    \[
        f(x)=\sum_{n=1}^{\infty} \frac{f_{n}(x)}{2^{n}}. 
    \]
    Prove that $f$ is not Riemann integrable. Argue that the area under the curve of $f$ is at most $2 .$ Discuss existence of this area.
\end{exercise}

\begin{proof}
    % WLOG, let $\{r_j\}$ be monotonical, i.e. $r_i\leqslant r_j$, for $i<j$. Because the domain of $f$ is $[0,1]$, $f_n(x)=0$, for all $r_n\geqslant 1$, and in this case, $f(x)=0$. 

    % Let $\mathbb{Q}^+:=\{r_i:r_i>0\}$
    If $r_n<0$ or $r_n>1$, then $f$ is integrable because it will be bounded on $[0,1]$. So, in the case $r_n\in[0,1]$, we have $f_n(x)=0$, $x\leqslant r_n$, and $f_n(x)=(x-r_n)^{-1/2}=\frac{1}{\sqrt{x-r_n}}\leqslant\frac{1}{\sqrt{x}}$, $x>r_n$. So, we can know that 
    \[
        f(x)\leqslant\sum_{n=1}^{\infty} \frac{1/\sqrt{x}}{2^{n}}=\frac{1}{\sqrt{x}}, 
    \]
    The equality holds when $r_n=0$. And because $\frac{1}{\sqrt{x}}$ is unbounded and $[0,1]$ is a closed interval, $f(x)$ is not integrable. 

    However, 
    \[
        \lim_{t\to0}\int_t^1\frac{1}{\sqrt{x}}\der x=2. 
    \]
    So, the area under the curve is at most $2$. \vspace{-10pt}
\end{proof}

\begin{exercise}{(4), }
    \textcolor{blue}{I modified $n$ to $2n$. }

    Compute
    \[
        h(x)=\lim _{m \rightarrow \infty} \lim _{n \rightarrow \infty} \cos (m ! \pi x)^{2n}, 
    \]
    and discuss this function in terms of Riemann integrability.
\end{exercise}
\begin{solution}
    Because $|\cos(m!\pi x)|\leqslant1$, 
    \[
        \lim_{n\to\infty}\cos(m!\pi x)^{2n}=
        \begin{cases}
            1, & |\cos(m!\pi x)|=1, \\
            0, & \text{otherwise}. 
        \end{cases}
    \]
    So, for the case $|\cos(m!\pi x)|=1$, $m!\pi x=k\pi$, $k\in\mathbb{Z}$. 
    
    Hence we can get that $x=\frac{k}{m!}$, $k\in\mathbb{Z}$. And for any rational number denoted by $\frac{p}{q}$, $p,q\in\mathbf{Z}$, $q\not=0$, $\frac{p}{q}=\frac{(q-1)!p}{q!}$, which means $x$ can be any rational number. 
    
    When $x$ is an irrational number, $h(x)=0$.  And when $x$ is a rational number, $h(x)=1$. 

    Because $\mu(\mathbb{Q})=0$, 
    \[
        \int_\mathbb{R}h(x)\der x=\int_{\mathbb{R}\backslash\mathbb{Q}}h(x)\der x+\int_\mathbb{Q}h(x)\der x=0. 
    \]
    % \[
    %     h(x)=\begin{cases}
    %         -1, & x \text{ is rational, $m!x$ and $n$ are both odd. }\\
    %          1, & x \text{ is rational, $m!x$ and $n$ are other cases. }\\
    %          0, & 
    %     \end{cases}
    % \]
\end{solution}

\begin{exercise}{(5), }
    Let $A, B \subset \mathbb{R}$ and $|B|=0 .$ Prove that $|A \cup B|=|A|$. 
\end{exercise}

\begin{proof}
    $\leqslant$: 
    \begin{align*}
        |A\cup B|\leqslant|A|+|B|=|A|, 
    \end{align*}
    $\geqslant$:
    \[
        |A|\leqslant|A\cup B|
    \]
    So, $|A\cup B|=|A|$. 
\end{proof}

\begin{exercise}{(6), }
    Let $A \subset \mathbb{R}$. Evaluate $\lim _{n \rightarrow \infty}|A \cup[-n, n]|$. 
\end{exercise}
\begin{proof}
    \textcolor{blue}{This problem is kind of strange. }

    Suppose $[-n,n]\in\mathbb{Z}$, 
    \[|A\cup[-n,n]|\leqslant|A|+|[-n,n]|\leqslant|A|+2n, \]
    So, 
    \[\lim _{n \rightarrow \infty}|A \cup[-n, n]|=|A|+\infty=\infty. \]
    \vspace{-30pt}
\end{proof}

\begin{exercise}{(7), }
    Let $\mathbb{Q}=\left\{r_{n}\right\}$. Define
    \[
        A=\mathbb{R} \backslash \bigcup_{n=1}^{\infty}\left(r_{n}-\frac{1}{2^{n}}, r_{n}+\frac{1}{2^{n}}\right). 
    \]
    Prove that $A$ is closed, that any interval $I \subset A$ contains more than one element and that $|A|=\infty$. 
\end{exercise}
\begin{proof}
    $U:=\bigcup_{n=1}^{\infty}\left(r_{n}-\frac{1}{2^{n}}, r_{n}+\frac{1}{2^{n}}\right)$. 
    \begin{itemize}
        \item Because $U$ is countable unions of open set, it is a open set. And 
        \[
            A=\mathbb{R}\backslash U=\mathbb{R}\cap U^c=U^c, 
        \]  
        So, $A$ is a closed set. 
        \item $|U|\leqslant\sum_{n=1}^{\infty}|\left(r_{n}-\frac{1}{2^{n}}, r_{n}+\frac{1}{2^{n}}\right)|\leqslant2$. And $\infty=|R|=|U|+|U^c|\leqslant2+|U^c|$. Hence $|U^c|=\infty$, i.e. $|A|=\infty$. 
        \item \textcolor{blue}{Can we get an interval $I\subset A$? I mean because $\mathbb{Q}\subset U$, $A$ is nowhere dense. Hence $A$ should only contain infinite single irrational points. One can not even take an interval in it. }
    \end{itemize}
    \vspace{-30pt}
\end{proof}

\end{document}

