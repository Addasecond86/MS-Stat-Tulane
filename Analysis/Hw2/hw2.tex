\documentclass[en, normal, 11pt, black]{elegantnote}

\usepackage{tcolorbox}
\tcbuselibrary{breakable}
\usepackage{amsfonts} 
\usepackage{newtxtext}
\usepackage{ulem}
\usepackage{amssymb}

\newenvironment{exercise}[1]{\begin{tcolorbox}[colback=black!15, colframe=black!80, breakable, title=#1]}{\end{tcolorbox}}

\renewenvironment{proof}{\begin{tcolorbox}[colback=white, colframe=black!50, breakable, title=Proof. ]\setlength{\parskip}{0.8em}}{\,\\\rightline{$\square$}\end{tcolorbox}}

\newenvironment{solution}{\begin{tcolorbox}[colback=white, colframe=black!50, breakable, title=Solution. ]\setlength{\parskip}{0.8em}}{\end{tcolorbox}}

\newcommand{\pder}{\partial\,}

\newcommand{\der}{\,\mathbf{d}}

\title{\textsc{Analysis: Homework 2}}
\author{\textsc{Zehao Wang}}
\date{\today}

% \vspace{-30pt}

\begin{document}
\maketitle

\begin{exercise}{(1), }
    Let $f, g:[a, b] \rightarrow \mathbb{R}$ be Riemann integrable. Prove that $f+g$ is also Riemann integrable and
    \[
        \int_{a}^{b}(f+g)=\int_{a}^{b} f+\int_{a}^{b} g. 
    \]
\end{exercise}

\begin{exercise}{(2), }
    Suppose $f:[a, b] \rightarrow \mathbb{R}$ is Riemann integrable. Define 
    \[
        F(t)= \begin{cases}0, & t=a, \\ \int_{a}^{t} f, & t \in(a, b]. \end{cases}
    \]
    Prove that $F$ is a continuous function. 
\end{exercise}

\begin{exercise}{(3), }
    Let $\mathbb{Q}=\left\{r_{j}\right\} .$ Define a function 
    \[
        f_{n}(x)= \begin{cases}\left(x-r_{n}\right)^{-1 / 2}, & \text { if } x>r_{n}, \\ 0, & \text { if } x \leq r_{n}. \end{cases}
    \]
    Then define $f:[0,1] \rightarrow[0, \infty]$ by
    \[
        f(x)=\sum_{n=1}^{\infty} \frac{f_{n}(x)}{2^{n}}. 
    \]
    Prove that $f$ is not Riemann integrable. Argue that the area under the curve of $f$ is at most $2 .$ Discuss existence of this area.
\end{exercise}

\begin{exercise}{(4), }
    Compute
    \[
        h(x)=\lim _{m \rightarrow \infty} \lim _{n \rightarrow \infty} \cos (m ! \pi x)^{n}, 
    \]
    and discuss this function in terms of Riemann integrability.
\end{exercise}
\begin{solution}
    Because $|\cos(m!\pi x)|\leqslant1$, 
    \[
        \lim_{n\to\infty}\cos(m!\pi x)^n=
        \begin{cases}
            \pm 1, & |\cos(m!\pi x)|=1, \\
            0, & \text{otherwise}. 
        \end{cases}
    \]
    So, for the case $|\cos(m!\pi x)|=1$, $m!\pi x=k\pi$, $k\in\mathbb{Z}$. 
    
    Hence we can get that $x=\frac{k}{m!}$, $k\in\mathbb{Z}$. And for any rational number denoted by $\frac{p}{q}$, $p,q\in\mathbf{Z}$, $q\not=0$, $\frac{p}{q}=\frac{(q-1)!p}{q!}$, which means $x$ can be any rational number. 
    
    When $x$ is an irrational number, $h(x)=0$.  But when $x$ is a rational number, the limit value would oscillate between $-1$ and $1$. In this case, the limit doesn't exist. So, 
    \[
        h(x)=0, \, x\in\mathbb{R}\backslash\mathbb{Q}. 
    \]
    Because $\mu(\mathbb{Q})=0$, 
    \[
        \int_\mathbb{R}h(x)\der x=\int_{\mathbb{R}\backslash\mathbb{Q}}h(x)\der x+\int_\mathbb{Q}h(x)\der x=0. 
    \]
    % \[
    %     h(x)=\begin{cases}
    %         -1, & x \text{ is rational, $m!x$ and $n$ are both odd. }\\
    %          1, & x \text{ is rational, $m!x$ and $n$ are other cases. }\\
    %          0, & 
    %     \end{cases}
    % \]
\end{solution}

\begin{exercise}{(5), }
    Let $A, B \subset \mathbb{R}$ and $|B|=0 .$ Prove that $|A \cup B|=|A|$. 
\end{exercise}

\begin{exercise}{(6), }
    Let $A \subset \mathbb{R}$. Evaluate $\lim _{n \rightarrow \infty}|A \cup[-n, n]|$. 
\end{exercise}

\begin{exercise}{(7), }
    Let $\mathbb{Q}=\left\{r_{n}\right\}$. Define
    \[
        A=\mathbb{R} \backslash \bigcup_{n=1}^{\infty}\left(r_{n}-\frac{1}{2^{n}}, r_{n}+\frac{1}{2^{n}}\right). 
    \]
    Prove that $A$ is closed; that any interval $I \subset A$ contains more than one element and that $|A|=\infty$. 
\end{exercise}

\end{document}

