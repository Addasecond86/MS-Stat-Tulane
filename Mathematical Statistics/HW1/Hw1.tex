\documentclass[12pt]{article}

\newcommand{\CN}{Mathematical Statistics}
\newcommand{\Ti}{Homework 1}
\newcommand{\Pf}{Dr. Didier}
\newcommand{\FN}{Zehao}
\newcommand{\LN}{Wang}



\usepackage[left=2cm, right=2cm, top=2cm, bottom=2cm]{geometry}

\title{\textsc{\Ti}}
\author{\textsc{\LN, \FN}}
\date{\emph{\today}}


\usepackage{tcolorbox}
\tcbuselibrary{breakable}
\usepackage{amsfonts} 
\usepackage{amsmath}
\usepackage{amssymb}
\usepackage{newtxmath}
\usepackage{enumerate}
\usepackage{minted}
\usepackage{fancyhdr}


\pagestyle{fancy}
\fancyhf{}
\rhead{\textsc{\FN}}
\lhead{\textsc{\CN}}
\chead{\textsc{\Ti}}
\rfoot{\thepage}


\linespread{1.3}
\setlength{\parskip}{3mm}
\setlength{\parindent}{2em}

% \usepackage[backend=biber]{biblatex}
% \addbibresource{my.bib}

\numberwithin{equation}{section}

% Problem 

\newcounter{exercise}[section]

\newenvironment{exercise}[1][\thesection.\refstepcounter{exercise}\theexercise]{\begin{tcolorbox}[colback=black!15, colframe=black!80, breakable, title=#1]}{\end{tcolorbox}}

% Proof

\newenvironment{proof}{\begin{tcolorbox}[colback=white, colframe=black!50, breakable, title=Proof. ]\setlength{\parskip}{0.8em}}{\hfill $\blacksquare$ \end{tcolorbox}}

\newenvironment{solution}{\begin{tcolorbox}[colback=white, colframe=black!50, breakable, title=Solution. ]\setlength{\parskip}{0.8em}}{\end{tcolorbox}}

\newcommand{\pder}{\partial\,}

\newcommand{\der}{\,\mathbf{d}\,}


\newcommand\define[1]{\emph{\bfseries Definition \the\value{section}.#1}} 
\newcommand\thm[1]{\emph{\bfseries Theorem \the\value{section}.#1}}
\newcommand\rmk[1]{\emph{\bfseries Remark \the\value{section}.#1}}
\newcommand\prop[1]{\emph{\bfseries Propositions \the\value{section}.#1}}

\usepackage{titlesec}

\titleformat{\section}{\bfseries\Large}{}{0em}{}

\AddToHook{cmd/section/before}{\clearpage}

\usepackage{hyperref}





\begin{document}
% \maketitle
\begin{titlepage}
    \begin{center}    
    \includegraphics[width=0.6\textwidth]{Tulane.png}\\[1cm]    
    
    \textsc{\Huge \CN}\\[0.5cm]
    \textsc{\large \Pf}\\[1.0cm]
    
    \textsc{\LARGE \Ti}\\[0.5cm]
    \textsc{\large \LN, \FN}\\
    {Master student in Statistics of Math Dept.}
    
    % Author and supervisor
    
    \vfill
    
    % Bottom of the page
    {\Large \emph{\today}}
    
    \end{center}
\end{titlepage}
    
    \setcounter{section}{1}    

    \section*{\Ti}

    \begin{exercise}
        Let \(\mathcal{P}\) be a family of probability measures. If \(T\) is complete for \(\mathcal{P}\) and \(U\) is equivalent to \(T\), show that \(U\) is complete for \(\mathcal{P}\). 
    \end{exercise}
    
    \begin{solution}
        For $T$, if it is complete, then for some function $g(T)$, 
        \[
            \int g(t)f(t)\der t=0 \Rightarrow \Pr(g(t)=0)=1. 
        \]
        Now, $T=h(U)$, and let $l(u)=g\circ h(u)$, we can get that 
        \[
            \int l(u)f(u)\der u=0 \Rightarrow \Pr(l(u)=0)=1. 
        \]
        So, $U$ is also complete. 
    \end{solution}
    
    \begin{exercise}
        Let \(X_{1}, \ldots, X_{n} \stackrel{i.i.d. }{\sim} \mathcal{N}\left(\mu, \sigma^{2}\right)\), \(\mu \in \mathbb{R}, \sigma^{2}>0 .\) Show that \(\bar{X}\) and \(S^{2}\) are independent. 
        
        (\rmk{2}: this classical result can be shown by means of characteristic functions. However, in the context of our course, a much shorter proof can be provided). 
    \end{exercise}
    
    \begin{solution}
        We know that $(\bar{X}, S^2)$ are sufficient. And $\bar{X}\sim N(\mu,\sigma^2/n)$. So, $\bar{X}$ is complete for $\mu$ (Because $N(\mu,\sigma^2/n)$ is full rank exponential family). 
        
        And $S^2$ only depends on $\sigma^2$ (Because $\frac{(n-1)S^2}{\sigma^2}=\sum_{i=1}^n(\frac{X-\bar{X}}{\sigma})^2\sim\chi_n^2$), it is an ancillary statistic. So, from Basu's theorem, they are independent. 
    \end{solution}
    % \printbibliography
\end{document}
