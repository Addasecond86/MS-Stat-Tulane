\chapter{}

\rmk{1} We follow the convention that the derivative with respect to a vector is the transpose of the gradient, the latter being a row vector. Wherever present, the acronym CRLB stands for “Cramer-Rao lower bound”. 

\begin{ex} \label{ex:5.1}
    For an underlying parametric family $\{f(x|\theta)\}_{\theta\in\Theta}$, let $\theta=H(\xi)$ be a reparametrization, where $H(\cdot)$ is differentiable and injective in a vicinity of $\xi$. Show that 
    \[
        I(\xi)=I(H(\xi))[H'(\xi)]^2. 
    \]
    \emph{Hint:} since\[
        I(\xi)=\mathbb{E}\left(\frac{\pder}{\pder \xi}\log f(x|H(\xi))\right)^2, 
    \]
    you can use the chain rule. 
\end{ex}

\begin{solution}
    \[
        \begin{aligned}
            I(\xi)&=\mathbb{E}\left(\frac{\pder \log f(x|H(\xi))}{\pder \xi}\right)^2\\
            &=\mathbb{E}\left(\frac{\pder \log f(x|H(\xi))}{\pder H(\xi)}\cdot\frac{\pder H(\xi)}{\pder \xi}\right)^2\\
            &=\mathbb{E}\left[\left(\frac{\pder \log f(x|H(\xi))}{\pder H(\xi)}\right)^2\cdot\left(\frac{\pder H(\xi)}{\pder \xi}\right)^2\right]\\
            &=\mathbb{E}\left(\frac{\pder \log f(x|H(\xi))}{\pder H(\xi)}\right)^2\cdot\left(\frac{\pder H(\xi)}{\pder \xi}\right)^2\\
            &=I(H(\xi))[H'(\xi)]^2. 
        \end{aligned}
    \]
\end{solution}

\begin{ex} \label{ex:5.2}
    In this problem, we finish a proof we started in class. Consider the following setting. 
    \begin{enumerate}[(i)]
        \item $\mathcal{P}=\{\mathbb{P}_\theta\}_{\theta\in\Theta}$, 1-dimensional exponential family generated by $(T,h)$, $\mathbf{X}\sim \mathcal{P}$, i.e., 
        \[
            f(\mathbf{x|\theta})=h(\mathbf{x})\exp\left(H(\theta)T(\mathbf{x})-B(\theta)\right); 
        \]
        \item $\tau(\theta):=\mathbb{E}_\theta T(\mathbf{X})$, $\eta=H(\theta)\in\mathcal{E}$; 
        \item the parameter functions $\theta\mapsto B(\theta), \tau(\theta), H(\theta)$ are, respectively, $C^1, C^1$ and $C^2$ in a
        vicinity of the true parameter value $\theta$ and satisfy $|\tau'(\theta)H'(\theta)H''(\theta)|>0$; 
        \item $\Theta$ is open in $\mathbb{R}$. 
    \end{enumerate}
    \begin{enumerate}[(a)]
        \item Prove that
        \[
        \tau(\theta):=\mathbb{E}_{\theta} T(\mathbf{X})=\frac{B^{\prime}(\theta)}{H^{\prime}(\theta)}, \quad  Var_{\theta} T(\mathbf{X})=\frac{B^{\prime \prime}(\theta)}{\left(H^{\prime}(\theta)\right)^{2}}-\frac{\eta^{\prime \prime}(\theta) B^{\prime}(\theta)}{\left(H^{\prime}(\theta)\right)^{3}}=\frac{B^{\prime \prime}(\theta)-\eta^{\prime \prime}(\theta) \tau(\theta)}{\left(\eta^{\prime}(\theta)\right)^{2}}
        \]
        You may assume that
        \[
            \frac{\partial}{\partial \theta} \int_{\mathbb{R}} f(\mathbf{x} \mid \theta) G(d \mathbf{x})=\int_{\mathbb{R}} \frac{\partial}{\partial \theta} f(\mathbf{x} \mid \theta) G(d \mathbf{x}), 
        \]
        \[
            \frac{\partial^{2}}{\partial \theta^{2}} \int_{\mathbb{R}} f(\mathbf{x} \mid \theta) G(d \mathbf{x})=\int_{\mathbb{R}} \frac{\partial^{2}}{\partial \theta^{2}} f(\mathbf{x} \mid \theta) G(d \mathbf{x}). 
        \]
        \item Conclude that \(Var_{\theta} T(\mathbf{X})\) attains the CRLB, i.e.,
        \[
         Var_{\theta} T(\mathbf{X})=\frac{1}{I(\tau(\theta))}
        \]
        Also conclude that, almost surely, it is the only unbiased estimator of \(\tau(\theta)\) to do so.
    \end{enumerate}
\end{ex}

\begin{solution}
    \begin{enumerate}[(a)]
        \item \label{ex:5.2.a} Let $\theta=H^{-1}(\eta)$, then 
        \[
            \mathbb{E}_\theta T(X) = A'(\eta) =\frac{\pder B(H^{-1}(\eta))}{\pder \eta}=B'(H^{-1}(\eta))\frac{1}{H'(\theta)}=\frac{B'(\theta)}{H'(\theta)}. 
        \]
        \[
            \begin{aligned}
                Var_\theta T(X)&=A''(\eta)\\
                &=\left(\frac{B'(H^{-1}(\eta))}{H'(H^{-1}(\eta))}\right)'\\
                &=\frac{B''(H^{-1}(\eta))(H^{-1}(\eta))'}{H'(H^{-1}(\eta))}-\frac{B'(H^{-1}(\eta))H''(H^{-1}(\eta))(H^{-1}(\eta))'}{\left(H'(H^{-1}(\eta))\right)^2}\\
                &=\frac{B''(\theta)}{(H'(\theta))^2}-\frac{B'(\theta)H''(\theta)}{(H'(\theta))^3}\\
                &=\frac{B''(\theta)-\eta''(\theta)\tau(\theta)}{(\eta'(\theta))^2}. 
            \end{aligned}
        \]
        \item \[
            I(\theta)=Var_\theta\left(\frac{f'(\mathbf{x|\theta})}{f(\mathbf{x|\theta})}\right)=Var_{\theta}(\eta'(\theta)T(\mathbf{x})-B'(\theta))=Var_\theta(T(\mathbf{x}))(\eta'(\theta))^2. 
        \]
        Then from Exercise \ref{ex:5.1}, 
        \begin{equation}
            I(\tau(\theta))=\frac{I(\theta)}{(\tau'(\theta))^2}=\left[\frac{\eta'(\theta)}{\tau'(\theta)}\right]^2Var_\theta T(\mathbf{x}). 
            \label{eq:5.2.1}
        \end{equation}
        And from Exercise \ref{ex:5.2} \ref{ex:5.2.a},
        \[
            Var_\theta T(X)=\frac{B''(\theta)-\eta''(\theta)\tau(\theta)}{(\eta'(\theta))^2}, \qquad \tau(\theta)=\frac{B'(\theta)}{\eta'(\theta)}. 
        \]
        \[
            \tau'(\theta)=\frac{B''(\theta)-\tau(\theta)\eta''(\theta)}{\eta'(\theta)}, 
        \]
        and \[
            \frac{\tau'(\theta)}{\eta'(\theta)}=Var_\theta T(X).
        \]
        So, 
        \[
            I(\tau(\theta))=\left[\frac{\eta'(\theta)}{\tau'(\theta)}\right]^2Var_\theta T(\mathbf{x})=\frac{\eta'(\theta)}{\tau'(\theta)}=\frac{1}{Var_\theta T(\mathbf{x})}. 
        \]
        If a statistic $\delta$ can attain the lower bond, then it can be written as following form: 
        \[
           \delta=a\left(\eta'(\theta)T(X)-B'(\theta)\right)+b.
        \]
        And if $\mathbb{E}(\delta)=\tau(\theta)$, $Var_\theta(\delta)=\frac{(\tau'(\theta))^2}{I(\theta)}$. Solve them, we can get that $a=1/H'(\theta)$, $b=\tau(\theta)$. So, the statistic can only be $T(X)$. 
    \end{enumerate}
\end{solution}

\begin{ex}
    Complete the following table: 
    \begin{center}
        \begin{tabular}{ccc}
            \hline
            distribution & parameter $\tau(\theta)$ & $I(\tau(\theta))$ \\ \hline
            $\mathcal{N}(\mu, \sigma^2)$ & $\mu$ &  \\
            $\mathcal{N}(\mu, \sigma^2)$ & $\sigma^2$ &  \\
            $B(n,p)$ & $p$ &  \\
            $Poi(\lambda)$ & $\lambda$ &  \\
            $\Gamma(\alpha, \lambda)$ & $\beta=\lambda^{-1}$ &  \\
            \hline
            \end{tabular}
    \end{center}
\end{ex}

\begin{solution}
    In order to simplify the problem, we only calculate the 1-dimension case: 
    \begin{itemize}
        \item \[
            p_\mu(x)=\frac{1}{\sqrt{2\pi}\sigma}\exp\left(-1/2\sigma^2(x^2-2\mu x+\mu^2)\right), 
        \]
        Then 
        \[
            h(x)=\frac{1}{\sqrt{2\pi}\sigma}\exp\left(-x^2/2\sigma^2\right), \eta(\mu) T(x)-B(\mu)=\frac{\mu}{\sigma^2}x-\frac{\mu^2}{2\sigma^2}. 
        \] 
        So, if $h(\mu)=\mu$, from \ref{eq:5.2.1}, 
        \[
            I(\mu)=\left(\frac{\eta'(\mu)}{h'(\mu)}\right)^2Var(T(X))=\frac{1}{\sigma^4}\sigma^2=1/\sigma^2. 
        \]
        \item \[
        \begin{aligned}
            \mathcal{L}_\sigma&=\log\left(\frac{1}{\sqrt{2\pi}\sigma}\exp\left(-1/2\sigma^2(x-\mu)^2\right)\right)\\
            &=-\frac{1}{2}\left(x-\mu\right)^2\sigma^{-2}-\log(\sqrt{2\pi}\sigma)
        \end{aligned}
        \]
        \[
            \mathcal{L}'_\sigma = \left(x-\mu\right)^2\sigma^{-3}-\sigma^{-1}, 
        \]
        \[
            \mathcal{L}''_\sigma = -3\left(x-\mu\right)^2\sigma^{-4}+\sigma^{-2}, 
        \]
        Then $E(X)=\mu$, $E(X^2)=Var(X)+(EX)^2=\sigma^2+\mu^2$, 
        \[
            I(\sigma)=-E(\mathcal{L''_\sigma})=2\sigma^{-2}. 
        \]
        \[
            I(\sigma^2)=\frac{I(\sigma)}{(2\sigma)^2}=\frac{1}{2\sigma^4}. 
        \]
        \item \[
            f_p(x)=\binom{n}{x}p^x(1-p)^{n-x}=\binom{n}{x}\exp\left(x\log(p)-x\log(1-p)+n\log(1-p)\right), 
        \]
        $\eta(p)T(x)-B(p)=\log\frac{p}{1-p}x-n\log\frac{1}{1-p}$, 
        \[
            I(np)=\left(\frac{\eta'(p)}{n}\right)^2Var(T(X))=\left(\frac{1}{np(1-p)}\right)^2np(1-p)=\frac{1}{np(1-p)}. 
        \]
        \[
            I(p)=\frac{I(np)}{1/n^2}=\frac{n}{p(1-p)}. 
        \]
        \item For Poisson distribution, $\eta(\lambda)=\log \lambda$, $T(x)=x$, $E(x)=Var(x)=\lambda$. So, 
        \[
            I(\lambda)=\frac{1}{\lambda^2}\lambda=1/\lambda. 
        \]
        \item For Gamma distribution, $\eta(\lambda)=-\lambda$, $T(x)=x$, $E(x)=\alpha/\lambda$, $Var(x)=\alpha/\lambda^2$. So,
        \[
            I(\alpha\beta)=\left(\frac{1}{-\alpha/\lambda^2}\right)^2\frac{\alpha}{\lambda^2}=\lambda^2/\alpha=\frac{1}{\alpha\beta^2}, 
        \]
        So, \[
            I(\beta)=\frac{I(\alpha\beta)}{(1/\alpha)^2}=\frac{\alpha}{\beta^2}. 
        \]
    \end{itemize}
\end{solution}

\begin{ex}
    Let $X\sim \Gamma(\alpha_0, \lambda)$, $\lambda>0$, where $\alpha_0$ is known. We see in the class that $I(\alpha_0/\lambda)=\lambda^2/\alpha_0$. Use this result to find $I(1/\lambda)$. 
\end{ex}

\begin{solution}
    Let $\xi=\alpha_0/\lambda$, $\theta=H(\xi)=\frac{1}{\alpha_0}\xi=1/\lambda$. Then
    \[
        I(1/\lambda)=I(\theta)=\frac{I(\xi)}{(H'(\xi))^2}=\frac{\lambda^2}{\alpha_0}\alpha_0^2=\alpha_0\lambda^2. 
    \]
\end{solution}

\begin{ex}
    Let \(X \sim \mathcal{N}\left(\mu, \sigma_{0}^{2}\right), \mu>0\), where \(\sigma_{0}^{2}>0\) is known. 
    \begin{enumerate}[(a)]
        \item Find \(I\left(\mu / \sigma_{0}^{2}\right)\) (hint: no calculations needed). 
        \item Use the result in (a) to compute \(I\left(\mu^{2}\right)\). 
    \end{enumerate}
\end{ex}

\begin{solution}
    \begin{enumerate}[(a)]
        \item From previous problem, $\eta(\mu)=\mu$, $T(X)=x/\sigma^2$. And $E(T(X))=\mu/\sigma^2$, $Var(T(X))=1/\sigma^2$. So, 
        \[
            I(\mu/\sigma^2)=\left(\frac{1}{1/\sigma^4}\right)1/\sigma^2=\sigma^2. 
        \]
        \item Let $\theta=\mu/\sigma^2$, $h(\theta)=\sigma^4\theta^2=\mu^2$. So, 
        \[
            I(h(\theta))=\frac{I(\theta)}{(h'(\theta))^2}=\frac{\sigma^2}{4\sigma^4\mu^2}=\frac{1}{4\sigma^2\mu^2}. 
        \]
    \end{enumerate}
\end{solution}

\begin{ex}
    Prove the lemma stated in class. 
    \begin{enumerate}[(i)]
        \item Suppose that the conditions A-(a),(b),(c) hold, and that
        \[
        \frac{\partial}{\partial \theta} \int_{\mathbb{R}^{n}} f(\boldsymbol{x} \mid \theta) G(d \boldsymbol{x})=\int_{\mathbb{R}^{n}} \frac{\partial}{\partial \theta} f(\boldsymbol{x} \mid \theta) G(d \boldsymbol{x}), \quad \theta \in \Theta .
        \]
        Then,
        \[
        \mathbb{E}_{\theta}\left(\frac{\partial}{\partial \theta} \log f(\boldsymbol{X} \mid \theta)\right)=0, \quad I(\theta)= Var_{\theta}\left(\frac{\partial}{\partial \theta} \log f(\boldsymbol{X} \mid \theta)\right). 
        \]
        \item If, in addition,
        \[
        \begin{gathered}
        \exists \frac{\partial^{2}}{\partial \theta^{2}} \log f(\boldsymbol{x} \mid \theta), \quad x \in \operatorname{supp} f(\boldsymbol{x} \mid \theta), \\
        \frac{\partial^{2}}{\partial \theta^{2}} \int_{\mathbb{R}^{n}} f(\boldsymbol{x} \mid \theta) G(d \boldsymbol{x})=\int_{\mathbb{R}^{n}} \frac{\partial^{2}}{\partial \theta^{2}} f(\boldsymbol{x} \mid \theta) G(d \boldsymbol{x}),
        \end{gathered}
        \]
        for \(\theta \in \Theta\), then,
        \[
        I(\theta)=-\mathbb{E}_{\theta}\left(\frac{\partial^{2}}{\partial \theta^{2}} \log f(\boldsymbol{X} \mid \theta)\right). 
        \]
    \end{enumerate}
\end{ex}

\begin{solution}
    \begin{enumerate}[(i)]
        \item \[
            \begin{aligned}
                E  \left[\left.{\frac {\partial }{\partial \theta }}\log f(X|\theta )\right|\theta \right]={}&\int _{\mathbb {R} }{\frac {{\frac {\partial }{\partial \theta }}f(X|\theta )}{f(X|\theta )}}f(X|\theta )\der x\\[3pt]={}&{\frac {\partial }{\partial \theta }}\int _{\mathbb {R} }f(X|\theta )\der x\\[3pt]={}&{\frac {\partial }{\partial \theta }}1=0.
            \end{aligned}
        \]
        \[
            \begin{aligned}
                I(\theta)&=E\left(\frac{\pder}{\pder \theta}\log f(X|\theta)\right)^2\\
                &=E\left(\frac{\pder}{\pder \theta}\log f(X|\theta)- E  \left[{\frac {\partial }{\partial \theta }}\log f(X|\theta )\right]\right)^2\\
                &=Var\left(\frac{\pder}{\pder \theta}\log f(X|\theta)\right)
            \end{aligned}
        \]
        \item \[
            \begin{aligned}
                \frac {\partial ^{2}}{\partial \theta ^{2}}\log f(X|\theta )&=\frac{\pder}{\pder \theta}\left(\frac{f'(X|\theta)}{f(X|\theta)}\right)\\
                &=\frac{\frac{\pder^2}{\pder\theta^2}f(X|\theta)}{f(X|\theta)}-\left(\frac{\pder}{\pder \theta}\log f(X|\theta)\right)^2
            \end{aligned}
        \]
        and
        \[
            E \left[{\frac {{\frac {\partial ^{2}}{\partial \theta ^{2}}}f(X|\theta )}{f(X|\theta )}}\right]=\int_{\mathbb{R}}{\frac {{\frac {\partial ^{2}}{\partial \theta ^{2}}}f(X|\theta )}{f(X|\theta )}}f(X|\theta )\der x={\frac {\partial ^{2}}{\partial \theta ^{2}}}\int _{\mathbb {R} }f(X|\theta )\der x=0. 
        \]
        So, \[
            \frac {\partial ^{2}}{\partial \theta ^{2}}\log f(X|\theta )=-\left(\frac{\pder}{\pder \theta}\log f(X|\theta)\right)^2, 
        \]
        \[
            E\left(\frac{\pder}{\pder \theta}\log f(X|\theta)\right)^2=-E\left(\frac {\partial ^{2}}{\partial \theta ^{2}}\log f(X|\theta )\right). 
        \]
    \end{enumerate}
\end{solution}

\begin{ex}
    Consider the following lemma. Assume that 
    \begin{enumerate}[(a)]
        \item \(\Theta\) is an open interval; 
        \item  \(A:=\{x ; f(x \mid \theta)>0\}\) is independent of \(\theta\); 
        \item \(\delta(X) \in \Delta\). 
    \end{enumerate}
    Let \(\psi(x \mid \theta)=\frac{\partial}{\partial \theta} \log f(x \mid \theta)\), assumed to exist at every pair \(x, \theta\). For some \(\varepsilon>0\), let \(b_{\theta}\) be a function that satisfies
    \[
    \begin{aligned}
    \mathbb{E}_{\theta} b_{\theta}^{2}(X) &<\infty \\
    \left|\frac{1}{\Delta} \frac{f(x \mid \theta+\Delta)-f(x \mid \theta)}{f(x \mid \theta)}\right| & \leq b_{\theta}(x) \quad \text { if }|\Delta|<\varepsilon .
    \end{aligned}
    \]
    Then,
    \[
    \mathbb{E}_{\theta} \psi(X, \theta)=0
    \]
    and
    \[
    \frac{\partial}{\partial \theta} \mathbb{E}_{\theta} \delta(X)=\mathbb{E}_{\theta}(\delta(X) \psi(X, \theta))= Cov_{\theta}(\delta, \psi)=\int \delta(x) \frac{\partial}{\partial \theta} \log f(x \mid \theta) G(d x)
    \]
    (and thus the \(C R L B\) is \(\left(\frac{\partial}{\partial \theta} \mathbb{E}_{\theta}(\delta)\right)^{2} / I(\theta)\)).

    Assume \(f\) is the density of a Cauchy \((0,1)\) random variable, and consider the location family associated with
    \[
    f(x-\theta)=\frac{1}{\pi} \frac{1}{1+(x-\theta)^{2}}, \quad \theta \in \mathbb{R} .
    \]
    Use the above lemma to establish the information inequality for the family \(\{f(x-\theta)\}_{\theta \in \mathbb{R}}\). 
    
    (suggestion: however clumsy the technical condition looks, this is an easy problem. For this location family, just show that
    \[
    \left|\frac{1}{\Delta} \frac{f(x \mid \theta+\Delta)-f(x \mid \theta)}{f(x \mid \theta)}\right| \leq 2+\varepsilon
    \]
    This can be done by expanding the left-hand side of the inequality. Now define
    \[
    b_{\theta}(X):=2+\varepsilon,
    \]
    and note that
    \[
    \mathbb{E}_{\theta} b_{\theta}^{2}(X)<\infty
    \]
    By using the lemma above, conclude that the information inequality holds for any estimator \(\delta(X)\) such that \(\left.\mathbb{E}_{\theta} \delta^{2}(X)<\infty\right)\)
\end{ex}

\begin{solution}
    For Cauchy$(0,1)$, 
    \begin{align*}
        &\quad\,\left|\frac{1}{\Delta} \frac{f(x \mid \theta+\Delta)-f(x \mid \theta)}{f(x \mid \theta)}\right|  \\
        &= \left|\frac{1}{\Delta} \frac{1+(x-\theta)^2-1-(x-\Delta-\theta)^2}{1+(x-\Delta-\theta)^2}\right|\\
        &= \left|\frac{1}{\Delta} \frac{2\Delta(x-\theta)-\Delta^2}{1+(x-\Delta-\theta)^2}\right|\\
        &= \left|2\frac{x-\theta}{1+(x-\Delta-\theta)^2}-\frac{\Delta}{1+(x-\Delta-\theta)^2}\right|\\
        &= \left|2\frac{x-\Delta-\theta}{1+(x-\Delta-\theta)^2}+\frac{\Delta}{1+(x-\Delta-\theta)^2}\right|\\
        &\leqslant2\frac{|x-\Delta-\theta|}{1+(x-\Delta-\theta)^2}+\frac{|\Delta|}{1+(x-\Delta-\theta)^2}\\
        &\leqslant2+\varepsilon. 
    \end{align*}
    Because $|x|<1+x^2$. Let $b(X)=2+\varepsilon$, then $\mathbb{E}b^2<\infty$. So, the information inequality holds for any estimator $\delta(X)$ with $\mathbb{E}_{\theta} \delta^{2}(X)<\infty$.
    \[
        \mathcal{L}'=\frac{\pder}{\pder \theta}\log f(x)=-\frac{2(x-\theta)}{1+(x-\theta)^2}, 
    \]
    \[
        \begin{aligned}
            E\left(\mathcal{L}'\right)^2&=4E\left(\frac{x-\theta}{1+(x-\theta)^2}\right)^2, 
        \end{aligned}
    \]
    \[
        \begin{aligned}
            E_{\theta}\left[\frac{X-\theta}{1+(X-\theta)^2}\right]^2
            &=\frac{1}{\pi}\int_{\mathbb R}\left[\frac{x-\theta}{1+(x-\theta)^2}\right]^2\frac{1}{1+(x-\theta)^2}\der x
            \\&=\frac{1}{\pi}\int_{\mathbb R}\frac{(x-\theta)^2}{(1+(x-\theta)^2)^3}\der x\\&=\frac{2}{\pi}\int_0^\infty\frac{t^2}{(1+t^2)^3}\der t
            \\&=\frac{1}{\pi}\int_0^\infty\frac{\sqrt u}{(1+u)^3}\der u
            \\&=\frac{1}{\pi}B\left(\frac{3}{2},\frac{3}{2}\right)
            \\&=\frac{1}{8}
        \end{aligned}
    \]
    So, $I(\theta)=1/2$, 
    \[
        Var(\delta)\geqslant2
    \]
\end{solution}

\begin{ex}
    The following theorem establishes a bound for the variance of an unbiased estimator in a multiparameter setting. 

    Suppose the following conditions hold. 
    \begin{enumerate}[(i)]
        \item \(\delta(\boldsymbol{X}) \in \Delta\) : an \(\mathbb{R}\)-valued unbiased estimator; 
        \item \(\boldsymbol{\Psi}(\boldsymbol{X} \mid \theta)\) : an \(\mathbb{R}^{r}\)-valued function with finite second moments; 
        \item \[
        \begin{aligned}
            \mathbb{R}^{r} \ni \gamma(\theta) &= Cov_{\theta}(\delta(\boldsymbol{X}), \boldsymbol{\Psi}(\boldsymbol{X} \mid \theta))\\
            &:=\mathbb{E}_{\theta}\left\{\left(\delta(\boldsymbol{X})-\mathbb{E}_{\theta} \delta(\boldsymbol{X})\right)\left(\boldsymbol{\Psi}(\boldsymbol{X} \mid \theta)-\mathbb{E}_{\theta}(\boldsymbol{\Psi}(\boldsymbol{X} \mid \theta))\right)\right\};
        \end{aligned}
        \]
        \item \[
            \begin{aligned}
                C(\theta) &= Cov_{\theta}(\boldsymbol{\Psi}(\boldsymbol{X} \mid \theta), \boldsymbol{\Psi}(\boldsymbol{X} \mid \theta))\\
                &:=\mathbb{E}_{\theta}\left\{\boldsymbol{\Psi}(\boldsymbol{X} \mid \theta)-\mathbb{E}_{\theta}(\boldsymbol{X} \mid \theta)\right\}\left\{\boldsymbol{\Psi}(\boldsymbol{X} \mid \theta)-\mathbb{E}_{\theta} \boldsymbol{\Psi}(\boldsymbol{X} \mid \theta)\right\}^T; 
            \end{aligned}
        \]
    \end{enumerate}
    Then,
        \[
            Var_{\theta} \delta(\boldsymbol{X}) \geq \gamma^T(\theta) C^{-1}(\theta) \gamma(\theta)
        \]
    We will break up its proof into simple steps. 
        \begin{enumerate}[(a)]
            \item Let \(\mathbf{a} \in \mathbb{R}^{r}\). Show that 
            \[
                Var_{\theta} \delta(\mathbf{X}) \geq \frac{\left[ Cov_{\theta}\left(\delta(\mathbf{X}), \mathbf{a}^T \boldsymbol{\Psi}(\mathbf{X} \mid \theta)\right)\right]^{2}}{ Var_{\theta}\left(\mathbf{a}^T \boldsymbol{\Psi}(\mathbf{X})\right)}; 
            \]
            \item Conclude that
            \[
                Var_{\theta} \delta(\mathbf{X}) \geq \sup _{\mathbf{a} \in \mathbb{R}^{n}} \frac{\left(\mathbf{a}^T \gamma(\theta)\right)^{2}}{\mathbf{a}^T C(\theta) \mathbf{a}}. 
            \]
            \item Consider the following result. 
            Let \(p\) be a \(r \times 1\) column vector, let \(P=p p^T\), and let \(Q\) be an \(r \times r\) real matrix. Then
            \[
            \sup _{\mathbf{a} \in \mathbb{R}^{r}} \frac{\mathbf{a}^T P \mathbf{a}}{\mathbf{a}^T Q \mathbf{a}}=\text { largest eigenvalue of } Q^{-1} P=p^T Q^{-1} p. 
            \]
            Use the lemma to conclude that
            \[
             Var_{\theta} \delta(\mathbf{X}) \geq \gamma^T(\theta) C^{-1}(\theta) \gamma(\theta). 
            \]
        \end{enumerate}
\end{ex}

\begin{solution}
    \begin{enumerate}[(a)]
        \item From Cauchy-Schwarz inequality, 
        \[
            [Cov(X,Y)]^2\leqslant Var(X) Var(Y). 
        \]
        So, \[
            Var(\delta(X))\geqslant \frac{[Cov(\delta(X), a^T\psi(X))]^2}{Var(a^T\psi(X))}. 
        \]
        \item $Cov(\delta(X), a^T\psi(X))=a^TCov(\delta(X), \psi(X))$, $Var(a^T\psi(X))=a^T\psi(T(X))a$. So, 
        \[
            Var(\delta(X))\geqslant \max_a\frac{[a^T\gamma(\theta)]^2}{a^TC(\theta)a}. 
        \]
        \item Using the lemma, $P=\gamma\gamma^T$
        \[
            Var(\delta(X))\geqslant \gamma^T(\theta)C^{-1}(\theta) \gamma(\theta). 
        \]
    \end{enumerate}
\end{solution}

\begin{ex}
    Consider the following set of assumptions. 
    \begin{enumerate}[(a)]
        \item \(\Theta \subseteq \mathbb{R}^{r}\) is an open rectangle; 
        \item the set
        \(
            A=\{\mathbf{x}: f(\mathbf{x} \mid \theta)>0\}
        \)
        is independent of \(\theta\); 
        \item \(\frac{\partial}{\partial \theta} f(\mathbf{x} \mid \theta)\) exists for all \(\theta \in \Theta\). 
    \end{enumerate}
    For \(\theta \in \mathbb{R}^{r}\), define the information matrix at \(\theta\) by
    \[
        I(\theta)=\mathbb{E}_{\theta}\left(\frac{\partial}{\partial \theta} \log f(\mathbf{X} \mid \theta)\right)\left(\frac{\partial}{\partial \theta} \log f(\mathbf{X} \mid \theta)\right)^T. 
    \]
    Prove the following theorem (a particular case of Theorem 2.6.6 in Lehmann \& Casella \((1998)\), p. 127). 

    \thm{1} (multiparameter information inequality) 
    
    Suppose the following assumptions hold. 
    \begin{enumerate}[(i)]
        \item \(\{f(x \mid \theta)\}_{\theta}\) is a family of generalized densities with respect to the \(\sigma\)-finite measure \(G(d x)\); 
        \item \(\mathrm{A}-(\mathrm{a}),(\mathrm{b}),(\mathrm{c})\); 
        \item \(\mathbb{E}_{\theta} \nabla_{\theta} \log f(\boldsymbol{X} \mid \theta)=0, \theta \in \Theta\);  
        \item \(I(\theta)\) is nonsingular, \(\theta \in \Theta\); 
        \item \(\mathbb{E}_{\theta} \delta^{2}(\boldsymbol{X})<\infty, \theta \in \Theta\); 
        \item \(\nabla_{\theta} \mathbb{E}_{\theta} \delta(\boldsymbol{X})=\int \delta(\boldsymbol{x}) \frac{\partial}{\partial \theta} f(\boldsymbol{x} \mid \theta) G(d \boldsymbol{x})\). 
    \end{enumerate}
    Then,
    \[
         Var_{\theta} \delta(\boldsymbol{X}) \geq \nabla_{\theta} \mathbb{E}_{\theta} \delta(\boldsymbol{X}) I(\theta)^{-1} \nabla_{\theta} \mathbb{E}_{\theta} \delta(\boldsymbol{X})^T, \quad \theta \in \Theta
    \]
    (hint: use the previous problem).
\end{ex}

\begin{solution}
    In previous problem, let $\psi(X)=\frac{\pder }{\pder \theta}\log f(x)$, then
    \[
        C=I(\theta); 
    \]
    \[
        \begin{aligned}
            \nabla_\theta E(\delta)^T&=\frac{\partial}{\partial \theta}\int \delta(X)f(X)\der x\\
            &=\int\delta(X)f'(X)\der x\\
            &=\int\delta(X)\frac{\pder}{\pder \theta}\log(f(X))f(X)\der x\\
            &=\int\delta(X)\psi(X)f(X)\der x\\
            &=E\left(\delta(X)-E(\delta(X))\right)\left(\psi(X)-E(\psi(X))\right)\\
            &=Cov(\delta, \psi)\\
            &=\gamma. 
        \end{aligned}
    \]
    Hence, we finish the proof. 
\end{solution}

\begin{ex}
    Let \(X_{1}, \ldots, X_{n} \stackrel{\text { i.i.d. }}{\sim} \mathcal{N}\left(\mu, \sigma^{2}\right), \mu \in \mathbb{R}, \sigma^{2}>0\). 
    \begin{enumerate}[(a)]
        \item Prove that
        \[
        I\left(\mu, \sigma^{2}\right)=\left(\begin{array}{cc}
        \frac{n}{\sigma^{2}} & 0 \\
        0 & \frac{n}{2 \sigma^{4}}
        \end{array}\right)
        \]
        Here you can use the equality
        \[
        \mathbb{E}_{\theta} \frac{\partial}{\partial \theta} \log f(\mathbf{X} \mid \theta) \frac{\partial}{\partial \theta} \log f(\mathbf{X} \mid \theta)^T=-\mathbb{E}_{\theta} \frac{\partial^{2}}{\partial \theta \partial \theta^T} \log f(\mathbf{X} \mid \theta). 
        \]
        \item Conclude that the variance of an UMVU estimator does not necessarily attain the CRLB. 
    \end{enumerate}
\end{ex}

\begin{solution}
    \begin{enumerate}[(a)]
        \item Knowing that 
        \[
            I(\theta;\mathbf{X})=nI(\theta;X), 
        \]
        \[
            I(\mu)=\frac{n}{\sigma^2}. 
        \]
        \[
            I(\sigma^2)=\frac{n}{2\sigma^4}. 
        \]
        \[
            \frac{\pder}{\pder \mu \,\pder \sigma}\log(f(x))=\frac{\pder}{\pder \sigma \,\pder \mu}\log(f(x))=-2(x-\mu)\sigma^{-3}, 
        \]
        \[
            E\left(-2(x-\mu)\sigma^{-3}\right)=0. 
        \]
        So, 
        \[
            E\left(\frac{\pder}{\pder \mu \,\pder \sigma^2}\log(f(x))\right)=E\left(\frac{\pder}{\pder \sigma^2 \,\pder \mu}\log(f(x))\right)=0.
        \]
        \item Consider $\frac{(n-1)S^2}{\sigma^2}\sim \chi^2
        (n-1)$, 
        \[
            Var\left(\frac{\sigma^2}{n-1}\frac{(n-1)S^2}{\sigma^2}\right)=\frac{\sigma^4}{(n-1)^2}2(n-1)=\frac{2\sigma^4}{n-1}>\frac{2\sigma^4}{n}. 
        \]
    \end{enumerate}
\end{solution}

\begin{ex}
    (n.b.: the purpose of this exercise is to provide a simple 1-dimensional example where the variance of the UMVU estimator does not attain the CRLB) 
    
    Let \(X \sim \operatorname{Poi}(\lambda), \lambda>0\), and let
    \[
        \delta(X):= \begin{cases}1, & \text { if } X=0 \\ 0, & \text { otherwise }\end{cases}. 
    \]
    \begin{enumerate}[(a)]
        \item Conclude that \(\delta(X)\) is UMVU for \(\mathbb{E}_{\lambda} \delta(X)\). 
        \item Find \(\mathbb{E}_{\lambda} \delta(X)\) and \(\operatorname{Var}_{\lambda} \delta(X)\). 
        \item Find the CRLB for \(\operatorname{Var}_{\lambda} \delta(X)\) (hint: use the formula for the change of variables in the information function). 
        \item Conclude that the CRLB cannot be met. 
        \item Why doesn't this contradict the result we showed in class on the attainment of the CRLB in exponential families? (see problem \#2 in this Problem set.)
    \end{enumerate}
\end{ex}

\begin{solution}
    \begin{enumerate}[(a)]
        \item $\delta(X)$ is unbiased, and $X$ is a sufficient statistic. So, $\delta(X)$ is UMVU. 
        \item \[
            E(\delta(X))=1\cdot e^{-\lambda}=e^{-\lambda}, \ Var(\delta(X))=E(\delta(X))^2-(E\delta(X))^2=\frac{e^\lambda-1}{e^{2\lambda}}. 
        \]
        \item \[
            Var(\delta)\geqslant\frac{\left(-e^{-\lambda}\right)^2}{I(\lambda)}=\frac{\lambda}{e^{2\lambda}}.
        \]
        \item When $\lambda>0$, $e^{\lambda}-1>\lambda$. Hence, CRLB cannot be met. 
        \item Because $\delta(X)$ is not $C^1$. 
    \end{enumerate}
\end{solution}

\begin{ex}
    Let \(\mathbf{X}\) and \(\mathbf{Y}\) be independent random vectors, where \(\mathbf{X} \sim f(\mathbf{x} \mid \theta), \mathbf{Y} \sim g(\mathbf{y} \mid \theta), \theta \in \Theta\), where \(f\) and \(g\) are generalized densities. Assume the following conditions hold. 
    \begin{enumerate}[(i)]
        \item A-(a),(b), (c); 
        \item \[
            \mathbb{E}_{\theta}\left(\frac{\partial}{\partial \theta} \log f(\mathbf{X} \mid \theta)\right)=0=\mathbb{E}_{\theta}\left(\frac{\partial}{\partial \theta} \log g(\mathbf{X} \mid \theta)\right), \quad \theta \in \Theta. 
        \]
    \end{enumerate}
    \begin{enumerate}[(a)]
        \item Show that \(I_{\mathbf{X}, \mathbf{Y}}(\theta)=I_{\mathbf{X}}(\theta)+I_{\mathbf{Y}}(\theta)\), namely, the information contained in each subsample adds up to the information in the whole sample. 
        \item Use (a) to rewrite the CRLB for an i.i.d. sample \(X_{1}, \ldots, X_{n}\). 
    \end{enumerate}
\end{ex}

\begin{solution}
    \begin{enumerate}[(a)]
        \item Let $\psi(X)=\frac{\pder }{\pder \theta}\log f(X; \theta)$, then 
        \[
            \begin{aligned}
                I_{X,Y}(\theta)&=E\left(\frac{\pder}{\pder \theta}\log f(X|\theta)f(Y;\theta)\right)^2\\
                &=E\left(\frac{\pder}{\pder \theta}\log f(X|\theta)+\frac{\pder}{\pder \theta}\log f(X|\theta)\right)^2\\
                &=E\left(\frac{\pder}{\pder \theta}\log f(X|\theta)\right)^2+E\left(\frac{\pder}{\pder \theta}\log f(Y;\theta)\right)^2\\
                &=I_X(\theta)+I_Y(\theta)
            \end{aligned}
        \]
        \item Let $\delta(X)$ be an unbiased statistics for $\theta$, 
        \[
            I_X(\theta)=\sum_{i=1}^nI_{X_i}(\theta)=nI_x(\theta). 
        \]
        \[
            Var_\theta(\delta(X))\geqslant\frac{1}{nI_x(\theta)}. 
        \]
    \end{enumerate}
\end{solution}


\begin{ex}
    Consider an estimator \(\delta(\mathbf{X}) \in \Delta\) under a parametric family \(\mathbf{X} \sim \mathcal{P}=\{f(\mathbf{x} \mid \theta)\}_{\theta \in \Theta}\), where \(f(\mathbf{x} \mid \theta)\) is a generalized density. Suppose, in addition, that the following conditions hold. 
    \begin{enumerate}[(i)]
        \item \(\mathbb{E}_{\theta} \delta(\mathbf{X})=g(\theta), \theta \in \Theta\), where the function \(g(\cdot)\) is differentiable; 
        \item the estimator can be explicitly written as
        \[
        \delta(\mathbf{x})=g(\theta)+\frac{g^{\prime}(\theta)}{I(\theta)} \frac{\partial}{\partial \theta} \log f(\mathbf{x} \mid \theta), \quad I(\theta)>0, \quad \theta \in \Theta. 
        \]
        \item \[
            \frac{\partial}{\partial \theta} \int_{\mathbb{R}^{n}} f(\mathbf{x} \mid \theta) G(d \mathbf{x})=\int_{\mathbb{R}^{n}} \frac{\partial}{\partial \theta} f(\mathbf{x} \mid \theta) G(d \mathbf{x}), \quad \theta \in \Theta .
        \]
        Show that \(\operatorname{Var}_{\theta} \delta(\mathbf{X})\) attains the CRLB. 
    \end{enumerate}
\end{ex}

\begin{solution}
    \begin{align*}
        Var(\delta(X))&=E(\delta(X)-E\delta(X))^2\\
        &=E\left(\frac{g'(\theta)}{I(\theta)}\frac{\pder}{\pder \theta}\log f(X|\theta)\right)^2\\
        &=\left(\frac{g'(\theta)}{I(\theta)}\right)^2E\left(\frac{\pder}{\pder \theta}\log f(X|\theta)\right)^2\\
        &=\left(\frac{g'(\theta)}{I(\theta)}\right)^2I(\theta)\\
        &=\frac{(g'(\theta))^2}{I(\theta)}\\
        &=\frac{\left(\frac{\pder}{\pder \theta}E(\delta(X))\right)^2}{I(\theta)}\\
        &=CRLB. 
    \end{align*}
\end{solution}

