\chapter{}

\rmk{1} As a reminder, for a canonical exponential family
\[
f(\mathbf{x} \mid \eta)=h(\mathbf{x}) \exp \{\langle\eta, T(\mathbf{x})\rangle-A(\eta)\}, \quad \eta \in \operatorname{int}(\mathcal{E}) \neq \emptyset
\]
it is true that
\[
\frac{\partial^{k}}{\partial \eta^{k}} \int f(\mathbf{x} \mid \eta) G(d \mathbf{x})=\int \frac{\partial^{k}}{\partial \eta^{k}} f(\mathbf{x} \mid \eta) G(d \mathbf{x}), \quad k \in \mathbb{N}
\]
where \(\frac{\partial^{k}}{\partial \eta^{k}}\) represents the vector derivative with respect to \(\eta\).

\rmk{2} As another reminder, the application of the theorem on asymptotic efficiency essentially boils down to verifying the conditions
\begin{enumerate}[(a)]
    \item \(\int f\left(x_{1} \mid \theta\right) G\left(d x_{1}\right)\) can be differentiated twice under the integral sign; 
    \item the third derivative of \(\log f\left(x_{1} \mid \theta\right)\) is uniformly bounded by an integrable random variable (i.e., \(M\left(X_{1}\right)\) such that \(\left.\mathbb{E}_{\theta}\left|M\left(X_{1}\right)\right|<\infty\right)\). 
\end{enumerate}
In all problems where \(\theta\) appears, it denotes the parameter or vector of parameters, accordingly. 

\begin{ex}
    Let \(\left\{X_{n}\right\}_{n \in \mathbb{N}} \stackrel{\text { i.i.d. }}{\sim} \mathcal{N}(\theta, 1), \theta \in \mathbb{R}\). Recall that the Hodges estimator is defined by
\[
\delta(\mathbf{X})=\left\{\begin{array}{cl}
\bar{X}, & \text { if }|\bar{X}| \geq n^{-1 / 4} \\
a \bar{X}, & \text { if }|\bar{X}|<n^{-1 / 4}
\end{array}\right.
\]
where \(0<a<1\). For \(g(\theta)=\theta\), prove that
\[
\sqrt{n}(\delta(\mathbf{X})-g(\theta)) \stackrel{d}{\rightarrow} \mathcal{N}(0, v(\theta)), \quad n \rightarrow \infty
\]
where
\[
v(\theta)=\left\{\begin{array}{cc}
1, & \text { if } \theta \neq 0 \\
a^{2}, & \text { if } \theta=0
\end{array}\right.
\]
(suggestion: for the case \(\theta \neq 0\), break up
\[
\sqrt{n}(\delta(\mathbf{X})-\theta)=\sqrt{n}(\bar{X}-\theta) 1_{\left\{|\bar{X}| \geq n^{-1 / 4}\right\}}+a \sqrt{n}(\bar{X}-\theta) 1_{\left\{|\bar{X}|<n^{-1 / 4}\right\}}+(a-1) \theta \sqrt{n} 1_{\left\{|\bar{X}|<n^{-1 / 4}\right\}} .
\]
Now show that \(\mathbb{P}_{\theta}\left(|\bar{X}|<n^{-1 / 4}\right) \rightarrow 0\) as \(\left.n \rightarrow \infty\right)\).
\end{ex}

\begin{solution}
    For the case $\theta=0$, $P(|\bar{X}|\geqslant n^{-1/4})\to 0$, 
    \begin{align*}
        \sqrt{n}(\delta(\mathbf{X})-\theta)&=\sqrt{n}(\bar{X}-\theta) 1_{\left\{|\bar{X}| \geq n^{-1 / 4}\right\}}+\sqrt{n}(a\bar{X}-\theta) 1_{\{|\bar{X}|<n^{-1/4}\}}\\
        &=a\sqrt{n}\bar{X}. 
    \end{align*}
    $Var(a\sqrt{n}\bar{X})=a^2. $

    \noindent For the case $\theta\neq 0$, 
    \[
        \begin{aligned}
            P(|\bar{X}|<n^{-1/4})&=P(\sqrt{n}|\bar{X}-\theta+\theta|<n^{1/4})=P(|Z+\sqrt{n}\theta|<n^{1/4})\\
            &=\Phi(n^{1/4}-\sqrt{n}\theta)-\Phi(-n^{1/4}-\sqrt{n}\theta)\to 0. 
        \end{aligned}
    \]
    So, $\sqrt{n}(\delta(X)-\theta)\to \sqrt{n}(\bar{X}-\theta)\to \mathcal{N}(0,1)$. 
\end{solution}



\begin{ex}
    \label{ex:7.2}
    Determine whether or not the location families based on the following densities have unique roots. 
    \begin{enumerate}[(a)]
        \item \label{ex:7.2.a} \(f(x)=\frac{1}{\sqrt{2 \pi}} \exp \left\{-\frac{1}{2} x^{2}\right\}, x \in \mathbb{R}\); 
        \item \(f(x)=\frac{e^{-x}}{\left(1+e^{-x}\right)^{2}}, x \in \mathbb{R}\). 
    \end{enumerate}
\end{ex}

\begin{solution}
    \begin{enumerate}[(a)]
        \item \[
            f(\mathbf{x}, \theta)=(2\pi)^{-n/2} \exp\left(-\frac{1}{2}\sum_{i=1}^n(x_i-\theta)^2\right)
        \]
        \[
            l(\theta)=-\frac{n}{2}\log 2\pi-\frac{1}{2}\left(\sum_{i=1}^n(x_i-\theta)^2\right). 
        \]
        \[
            \frac{\pder l(\theta)}{\pder \theta}=\sum_{i=1}^nx_i-n\theta=0\Rightarrow \theta=\sum x_i/n. 
        \]
        So, it has the unique root. 
        \item \[
            f(\mathbf{x}, \theta)=\exp\left(-\sum_{i=1}^n x_i-n\theta\right)\prod_{i=1}^n(1+e^{-x_i-\theta})^{-2}. 
        \]
        \[
            l(\theta)=-\sum_{i=1}^n x_i-n\theta-2\sum_{i=1}^n\log(1+e^{-x_i-\theta}). 
        \]
        \[
            \frac{\pder l(\theta)}{\pder \theta}=-n+2\sum_{i=1}^n\frac{e^{-x_i-\theta}}{1+e^{-x_i-\theta}}
        \]
        This log likelihood function is monotone decreasing, and $l(\theta)\to n$, when $\theta\to-\infty$; $l(\theta)\to -n$, when $\theta\to\infty$. So, it must have a unique root. 
    \end{enumerate}
\end{solution}

\begin{ex}
    Consider a sample \(X_{1}, \cdots, X_{n}\) from a Poisson distribution conditioned to be positive, i.e.,
    \[
        \mathbb{P}\left(X_{1}=x\right)=\frac{\theta^{x} e^{-\theta}}{x !\left(1-e^{-\theta}\right)} . \quad x \in \mathbb{N} .
    \]
    Show that the likelihood equation has a unique root for all values of \(x\). 
\end{ex}

\begin{solution}
    \[
        f(X)=\frac{\theta^{\sum x_i} e^{-n\theta}}{\prod x_i! (1-e^{-\theta})^n}. 
    \]
    \[
        l(\theta)=\sum x_i \log \theta-n\theta-\sum_{i=1}^n\log x_i!-n\log (1-e^{-\theta}). 
    \]
    \[
        \frac{\partial}{\partial\theta}l(\theta)=\frac{\sum x_i}{\theta}-\frac{n}{1-e^{-\theta}}=0. 
    \]
    \[
        \frac{\theta}{1-e^{-\theta}}=\bar{X}. 
    \]
    The range of left side is $(1, \infty)$ and the left side is monotone increasing. So, unless $\bar{X}=1$, $\theta$ has a unique root. 
\end{solution}

\begin{ex}
    The so-called digamma function \(\Gamma^{\prime}(\alpha) / \Gamma(\alpha), \alpha>0\), is an increasing function of \(\alpha\). Use this fact to prove that, for a sample \(X_{1}, \cdots, X_{n} \stackrel{\text { i.i.d. }}{\sim} \Gamma(\alpha, 1)\), the likelihood equation has a unique root. Can we arrive at the same conclusion by using a general result for canonical exponential families?
\end{ex}

\begin{solution}
    Let $\psi(\alpha)$ be digamma function. 
    \[
        f(X)=\frac{(\prod x_i)^{\alpha-1}e^{-\sum x_i}}{n\Gamma(\alpha)}. 
    \]
    \[
        l(\theta)=(\alpha-1)\sum_{i=1}^n \log x_i-n\log (\Gamma(\alpha))-\sum_{i=1}^n x_i. 
    \]
    \[
        \frac{\partial}{\partial\theta}l(\theta)=\sum_{i=1}^n\log x_i-n\psi(\alpha)=0. 
    \]
    \[
        \psi(\alpha)=\frac{\sum_{i=1}^n \log x_i}{n}>0. 
    \]
    And because $\log (\alpha)-1/(2\alpha)>\psi(\alpha)>\log (\alpha)-1/\alpha$, we can find a $\alpha_0$, which satisfies the above equation. 

    \noindent For canonical exponential families, $l'(\eta)/l(\eta)=\sum T(x_i)-nA'(\eta)$. Then, because $A(\eta)$ is convex, $A'(\eta)$ is increasing. And $E(T(X))=A'(\eta)$, we can also find a unique $\eta$ which satisfies $A'(\eta)=\sum T(x_i)/n$. 
\end{solution}

\begin{ex}
    Let \(\left\{Y_{n}\right\}_{n \in \mathbb{N}}\) be a random sequence defined on a probability space \((\Omega, \mathcal{F}, \mathbb{P})\). This sequence is said to be bounded in probability (or tight) if
    \[
        \forall \varepsilon>0, \exists m(\varepsilon)>0 \text { such that } \mathbb{P}\left(\left|Y_{n}\right|>m(\varepsilon)\right)<\varepsilon, \quad n \in \mathbb{N} .
    \]
    Notation: \(Y_{n}=O_{\mathbb{P}}(1)\). 
    \begin{enumerate}[(a)]
        \item Let \(\left\{M\left(X_{n}\right)\right\}_{n \in \mathbb{N}}\) be an i.i.d. sequence, \(\mathbb{E}\left|M\left(X_{1}\right)\right|<\infty\). Prove that
        \[
        Y_{n}:=\frac{1}{n} \sum_{i=1}^{n} M\left(X_{i}\right)
        \]
        is bounded in probability. 
        \item Let \(\left\{Y_{n}\right\}_{n \in \mathbb{N}}\) and \(\left\{Z_{n}\right\}_{n \in \mathbb{N}}\) be random sequences such that \(Y_{n}=O_{\mathbb{P}}(1)\) and \(Z_{n}=o_{\mathbb{P}}(1)\) (i.e., \(\left\{Z_{n}\right\}_{n \in \mathbb{N}}\) converges to zero in probability). Prove that \(Y_{n} Z_{n}=o_{\mathbb{P}}(1)\). 
    \end{enumerate}
\end{ex}

\begin{solution}
    \begin{enumerate}[(a)]
        \item Because $Y_n\sim \mathcal{N}(\mu, \sigma^2)$. $\forall\ \varepsilon$, we can find a $A$ that satisfies: 
        \[
            \begin{aligned}
                P(|Y_n|>A)&=P(Y_n>A)+P(Y_n<-A)\\
                &=P\left(\frac{A-\mu}{\sigma}<Z_n\right)+P\left(Z_n<\frac{-A-\mu}{\sigma}\right)\\
                &=1-\Phi\left(\frac{A-\mu}{\sigma}\right)+\Phi\left(\frac{-A-\mu}{\sigma}\right)<\varepsilon. 
            \end{aligned}
        \]
        So, $Y_n$ is bounded in probability. 
        \item $\forall\ \varepsilon>0$, let $M$ satisfies $P(|Y_n|>M)<\varepsilon$, 
        \begin{align*}
            P(|Y_nZ_n|>\varepsilon)&=P(|Y_nZ_n|>\varepsilon, |Y_n|>M)+P(|Y_nZ_n|>\varepsilon, |Y_n|<M)\\
            &\leqslant \varepsilon +P(|Z_n|>\varepsilon/M)\\
            &\leqslant \varepsilon+\frac{\varepsilon}{M}=\varepsilon'. 
        \end{align*}
        $\forall\ \varepsilon>0$, $\exists\ m(\varepsilon)=M\varepsilon/(M+1)$, such that 
        \[
            P(|Y_nZ_n|>m(\varepsilon))<\varepsilon. 
        \]
    \end{enumerate}
\end{solution}

\begin{ex}
    (this problem revisits the key steps of the proof of asymptotic efficiency of a consistent sequence of roots)

    For some \(\theta, \theta_{0} \in \Theta=(\underline{\theta}, \bar{\theta})\) (without loss of generality, \(\left.\theta>\theta_{0}\right)\), if the likelihood function is smooth enough (more specifically, three times differentiable in \(\theta\) over \(\Theta\) ), then we can use the Taylor expansion (with Lagrange residual)
    \begin{equation}
        \label{eq:7.6}
        \ell^{\prime}(\hat{\theta}_n)=\ell^{\prime}\left(\theta_{0}\right)+\left(\widehat{\theta}_{n}-\theta_{0}\right) \ell^{\prime \prime}\left(\theta_{0}\right)+\frac{1}{2}\left(\widehat{\theta}_{n}-\theta_{0}\right)^{2} \ell^{\prime \prime \prime}\left(\theta^{*}\right)
    \end{equation}
    where \(\theta^{*} \in\left(\theta_{0}, \theta\right)\) and the derivatives are all with respect to \(\theta\). 
    \begin{enumerate}[(a)]
        \item If \(\theta\) is a root of the likelihood equation, conclude that
        \[
            \sqrt{n}\left(\theta-\theta_{0}\right)=\frac{(1 / \sqrt{n}) \ell^{\prime}(\theta_0)}{-(1 / n) \ell^{\prime \prime}\left(\theta_{0}\right)-(1 / 2 n)\left(\theta-\theta_{0}\right) \ell^{\prime \prime \prime}\left(\theta^{*}\right)}. 
        \]
        \item  Prove that
        \[
            \frac{1}{\sqrt{n}} \ell^{\prime}\left(\theta_{0}\right) \stackrel{d}{\rightarrow} \mathcal{N}\left(0, I\left(\theta_{0}\right)\right), \quad n \rightarrow \infty. 
        \]
        \item Prove that
        \[
            -\frac{1}{n} \ell^{\prime \prime}\left(\theta_{0}\right) \stackrel{\mathbb{P}}{\rightarrow} I\left(\theta_{0}\right), \quad n \rightarrow \infty. 
        \]
        \item In expression (\ref{eq:7.6}), assume \(\theta\) is replaced by a consistent sequence of roots \(\left\{\hat{\theta}_{n}\right\}\). Show that
        \[
        \frac{1}{n} \ell^{\prime \prime \prime}\left(\theta_{n}^{*}\right)
        \]
        is bounded in probability. Use this fact and the previous parts to show that the sequence of roots \(\left\{\widehat{\theta}_{n}\right\}\) is asymptotically efficient. 
    \end{enumerate}
\end{ex}

\begin{solution}
    \begin{enumerate}[(a)]
        \item Here, $\ell(\theta)$ is log-likelihood function. $\theta$ is root, 
        \[
            \ell^{\prime}\left(\theta_{0}\right)+\left(\theta-\theta_{0}\right) \ell^{\prime \prime}\left(\theta_{0}\right)+\frac{1}{2}\left(\theta-\theta_{0}\right)^{2} \ell^{\prime \prime \prime}\left(\theta^{*}\right)=0, 
        \]
        \[
            \left(\theta-\theta_{0}\right)=\frac{\ell^{\prime}\left(\theta_{0}\right)}{-\ell^{\prime \prime}\left(\theta_{0}\right)-\frac{1}{2}\left(\theta-\theta_{0}\right) \ell^{\prime \prime \prime}\left(\theta^{*}\right)}, 
        \]
        \[
            \sqrt{n}\left(\theta-\theta_{0}\right)=\frac{(1/\sqrt{n})\ell^{\prime}\left(\theta_{0}\right)}{-(1/n)\ell^{\prime \prime}\left(\theta_{0}\right)-\frac{1}{2n}\left(\theta-\theta_{0}\right) \ell^{\prime \prime \prime}\left(\theta^{*}\right)}. 
        \]
        \item \begin{align*}
            \frac{1}{\sqrt{n}} \ell^{\prime}\left(\theta_{0}\right) =\sqrt{n}/n \sum\left(\log'(f(X_i|\theta_0))\right)
        \end{align*}
        Noticing that $E(\log'(f(X_i|\theta_0)))=0$, $Var(\log'(f(X_i|\theta_0)))=I(\theta_0)$, $E(\log'(f(X_i|\theta_0)))\stackrel{d}{\to}\mathcal{N}(0, I(\theta_0)/n)$. So, 
        \[
            \frac{1}{\sqrt{n}} \ell^{\prime}\left(\theta_{0}\right) =\sqrt{n}/n \sum\left(\log'(f(X_i|\theta_0))\right)\stackrel{d}{\to}\mathcal{N}(0, I(\theta_0)). 
        \]
        \item \[
            -\frac{1}{n}l''(\theta_0)=\frac{1}{n} \sum\frac{f'^2(X_i|\theta_0)-f(X_i|\theta_0)f''(X_i|\theta_0)}{f^2(X_i|\theta_0)}\stackrel{P}{\to}I(\theta_0)+E(\log''(f(X_i|\theta_0)))=I(\theta_0). 
        \]
        \item Because $\ell'''(X_i, \theta)<M(X_i)$,  let $M(X)=\max (M(X_1), \cdots, M(X_n))$. \[
            \frac{1}{n} \ell^{\prime \prime \prime}\left(\theta_{n}^{*}\right)=\frac{1}{n}\sum \ell'''(X_i|\theta)<M(X)\to E(M(X)). 
        \]
        So, $\frac{1}{n} \ell^{\prime \prime \prime}\left(\theta_{n}^{*}\right)$ is a bound in probability. 

        Finally, $n\to\infty$, $\widehat{\theta}_n\to\theta_0$, \[
            \sqrt{n}(\widehat{\theta}_n-\theta)\sim\frac{\mathcal{N}(0, I(\theta))}{I(\theta)}=\mathcal{N}\left((0, {\frac{1}{I(\theta)}}\right). 
        \]
    \end{enumerate}
\end{solution}

\begin{ex}
    (this problem is about casting new light on a well-known result) 

    Consider an i.i.d. sample \(X_{1}, \cdots, X_{n}\) from the location family \(f(x-\mu), \mu \in \mathbb{R}\), where
    \[
        f(x)=\frac{1}{\sqrt{2 \pi} \sigma_{0}} \exp \left\{-\frac{1}{2 \sigma_{0}^{2}} x^{2}\right\}
    \]
    and \(\sigma_{0}\) is known. 
    \begin{enumerate}[(a)]
        \item Show that the likelihood equation has a unique root. Is it related to the ML estimator? 
        \item Conclude that the sequence of roots thus obtained is consistent. 
        \item Conclude that such sequence is asymptotically efficient. Calculate the asymptotic variance. 
    \end{enumerate}
\end{ex}

\begin{solution}
    \begin{enumerate}[(a)]
        \item Same as (\ref{ex:7.2}.\ref{ex:7.2.a}). It is MLE. 
        \item $\mu=\sum x_i/n$. From LLN, 
        \[
            \delta(X)=\frac{\sum_{i=1}^nX_i}{n}\to\bar{X}=\mu. 
        \]
        So, it is consistent. 
        \item \[
            \bar{X}\sim\mathcal{N}(\mu,\sigma_0^2/n)\Rightarrow\sqrt{n}(\delta-\mu)\to \mathcal{N}\left(0, \sigma_0^2\right). 
        \]
        Because $I(\mu)=1/\sigma^2$, $\delta$ is asymptotically efficient. 
    \end{enumerate}
\end{solution}

\begin{ex}
    Let \(X_{1}, \cdots, X_{n} \stackrel{\text { i.i.d. }}{\sim} \operatorname{NegBin}(m, p)\), i.e.,
    \[
        \mathbb{P}\left(X_{1}=x\right)=\left(\begin{array}{c}
        m+x-1 \\
        m-1
        \end{array}\right) p^{m}(1-p)^{x}, \quad x=0,1,2, \cdots
    \]
    Find an EL estimator of \(p\). 
\end{ex}

\begin{solution}
    Assume $m$ is known. 
    \[
        \ell'(p)=\frac{nmp^{nm-1}(1-p)^{\sum x_i}+p^{nm}\sum x_i(1-p)^{\sum x_i-1}}{p^{nm}(1-p)^{\sum x_i}}=\frac{nm(1-p)+p\sum x_i}{p(1-p)}.
    \]
    $p\neq 0$ or $1$, $\ell'(p)=0$ has the unique root $\hat{p}=\frac{nm}{nm-\sum x_i}$. So, it is also the efficient likelihood estimator. 
\end{solution}

\begin{ex}
    Let \(X_{1}, \cdots, X_{n} \stackrel{\text { i.i.d. }}{\sim} \mathcal{N}\left(\theta, a \theta^{2}\right), \theta>0\), and \(a>0\) is known. Find an explicit expression for an EL estimator of \(\theta\). 
\end{ex}

\begin{solution}
    \[
        \ell(\theta)=-\frac{n}{2}\log\left(2\pi a\theta^2\right)-\frac{1}{2a\theta^2}\sum(x_i-\theta)^2.
    \]
    \[
        \ell'(\theta)=-\frac{n}{\theta}+\frac{\sum(x_i-\theta)^2}{a\theta^3}+\frac{\sum(x_i-\theta)}{a\theta^2}= 0. 
    \]
    \[
        \hat{\theta}=\frac{2\sum x_i}{1\pm\sqrt{1+4an}}. 
    \]
    Because $\theta>0$, the root is unique, i.e., $\hat{\theta}=\frac{2\sum x_i}{1+\sqrt{1+4an}}$. 
\end{solution}

\begin{ex}
    We now extend the theorem on efficient likelihood (EL) estimators for canonical exponential families to the multiparameter case. 

    Let \(X_{1}, \cdots, X_{n}\) be an i.i.d. sample from a multiparameter canonical exponential family. Under assumptions analogous to those for the univariate case, with probability \(\rightarrow 1\) as \(n \rightarrow \infty\) there exist solutions
    \[
        \widehat{\theta}_{n}=\hat{\theta}_{n}(\mathbf{X})
    \]
    of the likelihood equations such that 
    \begin{enumerate}[(i)]
        \item \(\hat{\theta}_{j n} \stackrel{\mathbb{P}}{\rightarrow} \theta_{j}, n \rightarrow \infty\); 
        \item \(\sqrt{n}\left(\widehat{\theta}_{n}-\theta\right) \stackrel{d}{\rightarrow} \mathcal{N}\left(0, I(\theta)^{-1}\right), n \rightarrow \infty\); 
        \item \(\widehat{\theta}_{j n}\) is asymptotically efficient in the sense that
        \[
            \sqrt{n}\left(\hat{\theta}_{j n}-\theta_{j}\right) \stackrel{d}{\rightarrow} \mathcal{N}\left(0,\left(I(\theta)^{-1}\right)_{j j}\right), \quad n \rightarrow \infty. 
        \]
    \end{enumerate}
    Here, we define
    \[
        I(\theta)=\mathbb{E}_{\theta}\left[\left(\frac{\partial}{\partial \theta} \log f(x \mid \theta)\right)\left(\frac{\partial}{\partial \theta} \log f(x \mid \theta)\right)^T\right]
    \]
    As before, under smoothness we have
    \[
        I(\theta)=-\mathbb{E}\left(\frac{\partial^{2}}{\partial \theta \partial \theta^T} \log f(x \mid \theta)\right)
    \]
    So, let \(X_{1}, \cdots, X_{n}\) be an i.i.d. sample from a multiparameter canonical exponential family
    \[
        f(x \mid \boldsymbol{\eta})=h(x) \exp \{\langle\boldsymbol{\eta}, T(x)\rangle-A(\boldsymbol{\eta})\}
    \]
    Assume that \(\mathcal{E} \neq \emptyset, \mathcal{E} \subseteq \mathbb{R}^{k}\), is open and that identifiability holds. 
    \begin{enumerate}[(a)]
        \item Conclude that, with probability \(\rightarrow 1\) as \(n \rightarrow \infty, \widehat{\boldsymbol{\eta}}_{\mathrm{ML}}\) exists and is unique, and find its expression. 
        \item Conclude that
        \[
            \sqrt{n}\left(\widehat{\boldsymbol{\eta}}_{\mathrm{ML}}-\boldsymbol{\eta}\right) \stackrel{d}{\rightarrow} \mathcal{N}\left(0, \frac{1}{Var_{\boldsymbol{\eta}}(T)}\right), \quad n \rightarrow \infty
        \]
        i.e., \(\widehat{\boldsymbol{\eta}}_{\mathrm{ML}}\) is an EL estimator of \(\boldsymbol{\eta}\).
    \end{enumerate}
\end{ex}

\begin{solution}
    多参数的情况不写了, 不可能考. 
\end{solution}

\begin{ex}
    An estimator \(\delta_{n}(\mathbf{X})\) is called \(\sqrt{n}\)-consistent to \(\theta\) if \(\sqrt{n}\left(\delta_{n}(\mathbf{X})-\theta\right)\) is bounded in probability. Let \(\left\{X_{n}\right\}_{n \in \mathbb{N}}\) be an i.i.d. sequence such that \(\mathbb{E} X_{1}^{2}<\infty\). Show that \(\bar{X}_{n}\) is \(\sqrt{n}\)-consistent to \(\mu:=\mathbb{E} X_{1}\). 
    
    (hint: Prohorov's theorem ensures that boundedness in probability is equivalent to the existence of a subsequence that converges weakly). 
\end{ex}

\begin{solution}
    这题和上面那个有啥区别吗? 
\end{solution}

\begin{ex}
    Let \(\tilde{\theta}_{n}(\mathbf{X})\) be a \(\sqrt{n}\)-consistent estimator of \(\theta\) under the log-likelihood \(\ell(\theta \mid \mathbf{X})\). Under the conditions of the asymptotic normality theorem for EL estimators, it can be shown that the estimator
    \begin{equation}
        \label{eq:7.12}
        \delta_{n}(\mathbf{X}):=\tilde{\theta}_{n}(\mathbf{X})-\frac{\ell^{\prime}\left(\tilde{\theta}_{n}(\mathbf{X})\right)}{\ell^{\prime \prime}\left(\tilde{\theta}_{n}(\mathbf{X})\right)}
    \end{equation}
    is also asymptotically efficient (Theorem 6.4.3 in Lehmann and Casella (1998)).
    Let \(\{f(x-\theta)\}_{\theta \in \Theta}\) be a location family based on the symmetric (around zero) and a.c. density \(f\) with finite second moments. Assume that the likelihood equation
    \[
    \frac{\partial}{\partial \theta} \log f(\mathbf{x} \mid \theta)=0
    \]
    has potentially multiple roots. Use (\ref{eq:7.12}) to find an asymptotically efficient estimator for \(\theta\) 

    (hint: look at the previous problem). 

    \rmk{3} It is clear that the expression (0.3) stems from one step of Newton-Raphson's method.
\end{ex}

\begin{solution}
    We know that $\sqrt{n}(\bar{X}-\theta)$ is bounded in probability, so we can use Theorem 6.4.3 to find an asymptotically efficient estimator for $\theta$: 
    \[
        \delta_n(X)=\bar{X}-\frac{\ell'(\bar{X})}{\ell''(\bar{X})}. 
    \]
    吐了, 这题到底在干啥. 
\end{solution}

\begin{ex}
    Let \(\left\{X_{n}\right\}_{n \in \mathbb{N}} \stackrel{\text { i.i.d. }}{\sim} \mathcal{N}\left(\mu, \sigma^{2}\right), \mu \in \mathbb{R}, \sigma^{2}>0\), and let \(h(x)=x^{2}\). 
    \begin{enumerate}[(a)]
        \item Construct asymptotic approximations to \(\mathbb{E}_{\theta} h(\bar{X})\) and \(Var_{\theta} h(\bar{X})\). 
        \item Based on the properties of Gaussian distributions, compute \(\mathbb{E}_{\theta} h(\bar{X})\) and \(Var_{\theta} h(\bar{X})\) explicitly and compare the expressions with the approximations obtained in (a). Do the approximations orders match? 
    \end{enumerate}
\end{ex}

\begin{solution}
    \begin{enumerate}[(a)]
        \item $\hat{x}_{ML}=\bar{X}$. From Delta method, take $T(X)=\bar{X}$, $h(x)=x^2$.
        \[
            n(\bar{X}^2-\mu^2)\stackrel{d}{\to}1/2 \sigma^2 2 \chi_1^2=\sigma^2\chi_1^2. 
        \]
        $E(\bar{X}^2): \mu^2+\sigma^2/n$, $Var(\bar{X}^2): \frac{2\sigma^4}{n^2}$. 
        \item $\bar{X}\sim \mathcal{N}(\mu,\sigma^2/n)$. 
        \[
            E(\bar{X}^2)=\sigma^2/n+\mu^2, \quad Var(\bar{X}^2)=\mu^4+6\mu^2\sigma^2/n+3\sigma^4/n^2.
        \]
        The order is match. 
    \end{enumerate}
\end{solution}

\begin{ex}
    Consider the following setup. 
    \begin{itemize}
        \item \(\left\{X_{n}\right\}_{n \in \mathbb{N}}\) : i.i.d. sequence, \(X_{1} \sim \mathcal{P}=\left\{\mathbb{P}_{\theta}\right\}_{\theta \in \Theta}\); 
        \item \(\mathbb{E}_{\theta} X_{1}=\mu, Var_{\theta} X_{1}=\sigma^{2}>0, \mathbb{E}_{\theta} X_{1}^{4}<\infty\); 
        \item \(h: I \rightarrow \mathbb{R}\) is a 4-times differentiable function, where \(\mathbb{P}_{\theta}\left(X_{1} \in I\right)=1, \theta \in \Theta\); 
        \item \(\left|h^{(i v)}(x)\right| \leq M<\infty, x \in I\); 
        \item \(c_{n}=1+\frac{a}{n}+O\left(\frac{1}{n^{2}}\right), n \in \mathbb{N}, a \in \mathbb{R}\). 
    \end{itemize}
    Consider the estimator \(\delta_{n}(\bar{X})=h\left(c_{n} \bar{X}\right)\). Show that
    \[
        Var_{\theta} \delta_{n}(\bar{X})=\frac{\sigma^{2}}{n}\left[h^{\prime}(\mu)\right]^{2}+O\left(\frac{1}{n^{2}}\right) .
    \]
\end{ex}

\begin{solution}
    6.2.a. 
\end{solution}

\begin{ex}
    Let \(\left\{X_{n}\right\}_{n \in \mathbb{N}} \stackrel{\text { i.i.d. }}{\sim} \mathcal{N}\left(\mu, \sigma^{2}\right), \mu \in \mathbb{R}, \sigma^{2}>0\), and let \(h(x)=e^{x^{4}}\). It is clear that all moments of \(X\) and all derivatives of \(h\) exist. 
    \begin{enumerate}[(a)]
        \item Show that \(\mathbb{E}\left|h\left(\bar{X}_{n}\right)\right|=\infty, n \in \mathbb{N}\). Conclude that
        \[
            \mathbb{E}_{\theta} \sqrt{n}\left[h\left(\bar{X}_{n}\right)-h(\mu)\right]^{2}=\infty, \quad n \in \mathbb{N} .
        \]
        \item By contrast, assuming \(\mu \neq 0\), show that \(\sqrt{n}\left[h\left(\bar{X}_{n}\right)-h(\mu)\right]\) has a normal limit distribution with finite variance. 
        \item Conclude that the theorem seen in class on the approximation of moments does require bounded derivatives. 
    \end{enumerate}
\end{ex}

\begin{solution}
    \begin{enumerate}[(a)]
        \item \[
            E\left(e^{x^4}\right)=\int A\exp\left(B(x-\mu)^2+x^4\right)\der x
        \]
        $x\to\infty$, $A\exp\left(B(x-\mu)^2+x^4\right)\to\infty$. So, $E\left(e^{x^4}\right)\to\infty$. Samilarly, 
        \[
            E(e^{2x^4})=\infty. 
        \]
        So, $E(\sqrt{n}(h(\bar{X}_n)-h(\mu))^2)=\infty$. 
        \item $Var(\bar{X})=\sigma^2/n<\infty$. From Delta method, it has a normal limit distribution with finite variance. 
        \item 
    \end{enumerate}
\end{solution}

\begin{ex}
    The Hardy-Weinberg principle from population genetics describes an idealized situation where there are no evolutionary influences. 
    
    Assume that two alleles \(A\) and \(a\) appear in a population's gene pool with frequencies \(\theta\) and \(1-\theta, 0<\theta<1\), respectively. The principle states that, in equilibrium, the frequencies of the genotypes \(A A, A a\) and \(a a\) will remain constant from generation to generation at
    \[
    p_{1}(\theta)=\theta^{2}, \quad p_{2}(\theta)=2 \theta(1-\theta), \quad p_{3}(\theta)=(1-\theta)^{2},
    \]
    respectively.
    
    Assume that a sample \(\mathbf{X}=\left(X_{1}, X_{2}, X_{3}\right)\) is drawn. If the population is very large, then we can suppose that the sampling was done with replacement, whence
    \[
    \mathbb{P}(\mathbf{X}=\mathbf{x} \mid \theta)=\frac{n !}{x_{1} ! x_{2} ! x_{3} !} p_{1}(\theta)^{x_{1}} p_{2}(\theta)^{x_{2}} p_{3}(\theta)^{x_{3}}, \quad \sum_{i=1}^{3} x_{i}=n .
    \]
    What is the probability that \(\widehat{\theta}_{\mathrm{ML}}\) exists, and how does it behave as \(n \rightarrow \infty\) ?
\end{ex}



