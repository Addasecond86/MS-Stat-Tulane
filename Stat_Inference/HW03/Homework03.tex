\documentclass[14pt]{elegantbook}

\newcommand{\CN}{BIOS 7040\\[0.5cm] Statistical Inference I}
\newcommand{\Ti}{Homework 3}
\newcommand{\Pf}{Dr. Srivastav}
\newcommand{\FN}{Zehao}
\newcommand{\LN}{Wang}
\usepackage[fontsize=14pt]{fontsize}

\usepackage{enumitem}
\renewcommand{\chaptername}{Homework}
\begin{document}
\begin{titlepage}
	\begin{center}
		
		\includegraphics[width=0.6\textwidth]{Tulane.png}\\[1cm]

		\textsc{\Huge \CN}\\[0.5cm]
		\textsc{\large \Pf}\\[1.0cm]

		\textsc{\LARGE \Ti}\\[0.5cm]
		\textsc{\large \LN, \FN}\\
		{Master student in Statistics of Math Dept.}

		\vfill

		{\Large \emph{Update: \today}}

	\end{center}
\end{titlepage}

\tableofcontents
\setcounter{page}{0}
\setcounter{chapter}{2}
\chapter{}

    \begin{exercise*}[1]
        Four buses carrying 148 students from the same school arrive at a football stadium. The buses carry, respectively, 40, 33, 25, and 50 students. One of the students is randomly selected. Let $X$ denote the number of students that were on the bus carrying the randomly selected student. One of the 4 bus drivers is also randomly selected. Let $Y$ denote the number of students on her bus. 
        \begin{enumerate}[(a)]
            \item Which of $E[X]$ or $E[Y]$ do you think is larger? Why? 
            \item Compute $E[X]$ and $E[Y]$. 
        \end{enumerate}
    \end{exercise*}

    \begin{solution}
        \begin{enumerate}[(a)]
            \item I think $E[X]$ is larger. Because the probability of selecting a student from the bus carrying more students is larger than the bus carrying less students. So, the expected students number would be larger. For $E[Y]$, $4$ buses have a same probability to be selected. 
            \item \[E[X]=40\cdot\frac{40}{148}+33\cdot\frac{33}{148}+25\cdot\frac{25}{148}+50\cdot\frac{50}{148}\approx39.2838\]
            \[E[Y]=(40+33+25+50)\cdot\frac{1}{4}=37. \]
        \end{enumerate}
    \end{solution}

    \begin{exercise*}[2]
        One of the numbers 1 through 10 is randomly chosen. You are to try to guess the number chosen by asking questions with ``yes-no" answers. Compute the expected number of questions you will need to ask in each of the following two cases: 
        \begin{enumerate}[(a)]
            \item Your $i_{th}$ question is to be ``Is it $i$ ?'' $i = 1, 2, 3, 4, 5, 6, 7, 8, 9, 10$. 
            \item With each question you try to eliminate one half of the remaining numbers, as nearly 
            as possible. 
        \end{enumerate}
    \end{exercise*}

    \begin{solution}
        \begin{enumerate}[(a)]
            \item Let $K$ be the number of questions you need to ask. First of all, we only need 9 questions to know the correct number in 10 numbers. Because, if you have already asked 9 questions and still not get a ``yes'' answer, the right number can only be the one left, which means you needn't ask another question. So, $K=1,\cdots,8$ means the number is $1, \cdots, 8$, respectively. And $K=9$ means the number is $9$ or $10$. So, you only need to ask $9$ questions at most. 
            Then, knowing the numbers are uniformly distributed, we have 
            \[
                \Pr(K=1)=\frac{1}{10}, \quad \cdots \quad \Pr(K=8)=\frac{1}{10}, \quad \Pr(K=9)=\frac{2}{10}. 
            \]
            So, 
            \[
                E[K]=\sum_{i=1}^8i\cdot\frac{1}{10}+9\cdot\frac{2}{10}=5.4. 
            \]
            \item Let $K$ be the number of questions you need to ask. 
            \begin{itemize}
                \item $K=1$, there are only $5$ numbers left. 
                \item $K=2$, there are only $3$ or $2$ numbers left. 
                \item $K=3$, there are only $2$ or $1$ numbers left. 
            \end{itemize}
            When there is only 1 number left, it is the correct number. So, with 3 steps, there are $2\times 2=4$ cases need one more step. Then, 
            \[E[K]=3\cdot\frac{6}{10}+4\cdot\frac{4}{10}=3.4. \]
        \end{enumerate}
    \end{solution}

    \begin{exercise*}[3]
        Let X be a binomial random variable with parameters $(n, p)$. What value of $p$ maximizes $P\{X = k\}$, $k = 0, 1, \cdots, n$? This is an example of a statistical method used to estimate $p$ when a binomial $(n, p)$ random variable is observed to equal $k$. If we assume that n is known, then we estimate $p$ by choosing that value of $p$ which maximizes $P\{X = k\}$. This is known as the method of maximum likelihood estimation. 
    \end{exercise*}

    \begin{solution}
        \begin{align*}
            P(X=k)=\binom{n}{k}p^k(1-p)^{n-k}, 
        \end{align*}
        Maximizing $P(X=k)$ equals to maximizing $p^k(1-p)^{n-k}$, and because $\ln(p^k(1-p)^{n-k})$ has the same monotonicity with $p^k(1-p)^{n-k}$, 
        finding a $p$ to reach the maximum of $p^k(1-p)^{n-k}$ is equivalent to finding the $p$ to maximize $k\ln p+(n-k)\ln(1-p)$. 
        \[\frac{\partial (k\ln p+(n-k)\ln(1-p))}{\partial p}=\frac{k}{p}-\frac{n-k}{1-p}. \]
        Let the above equation equals to $0$, i.e., 
        \[
            \frac{k}{p}-\frac{n-k}{1-p}=0\Rightarrow \frac{k}{p}=\frac{n-k}{1-p}p\Rightarrow\frac{1-p}{p}=\frac{n-k}{k}
        \]
        \[
            \Rightarrow\frac{1}{p}-1=\frac{n}{k}-1\Rightarrow p=\frac{k}{n}.
        \]
        And when $p>\frac{k}{n}$, $p^k(1-p)^{n-k}$ is decreasing. $p<\frac{k}{n}$, $p^k(1-p)^{n-k}$ is increasing. So, $p=\frac{k}{n}$ can maximize $P(X=k)$. 
    \end{solution}

    \begin{exercise*}[4]
        Suppose that the number of events that occur in a specified time is a Poisson random variable with parameter $\lambda$. If each event is counted with probability $p$, independently of every other event, show that the number of events that are counted is a Poisson random variable with parameter $\lambda p$. Also, give an intuitive argument as to why this should be so. As an application of the preceding result, suppose that the number of distinct uranium deposits in a given area is a Poisson random variable with parameter $\lambda = 10$. If, in a fixed period of time, each deposit is discovered independently with probability  $1/50$, find the probability that 
        \begin{enumerate}[(a)]
            \item exactly $1$; 
            \item at least $1$; 
            \item at most $1$ deposit is discovered during that time.  
        \end{enumerate}
    \end{exercise*}

    \begin{solution}
        Let $X$ be the Poisson R.V. and $N$ be the number of counted. Then, 
        \begin{align*}
            \Pr(N=n)&=\sum_{i=0, n\leq i}^\infty \Pr(N=n|X=i)\Pr(X=i)\\
            &=\sum_{i=n}^\infty \binom{i}{n}p^n(1-p)^{i-n}\frac{e^{-\lambda}\lambda^i}{i!}, \text{ (when $i<n$, $\Pr(N=n|X=i)=0$.)}\\
            &=\sum_{i=n}^\infty \frac{i!}{n!(i-n)!}p^n\lambda^{i-n}\lambda^{n-i}(1-p)^{i-n}\frac{e^{-\lambda}\lambda^{i}}{i!}\\
            &=\sum_{i=n}^\infty \frac{1}{(i-n)!}\lambda^{i-n}(1-p)^{i-n}\frac{e^{-\lambda}(\lambda p)^{n}}{n!}\\
            &=\sum_{i=n}^\infty \frac{(\lambda-\lambda p)^{i-n}}{(i-n)!}\frac{e^{-\lambda}(\lambda p)^{n}}{n!}
        \end{align*}
        And we know that $e^x=\sum_{i=0}^\infty\frac{x^i}{i!}$, So, 
        \[
            \Pr(N=n)=e^{\lambda-\lambda p}e^{\lambda}\frac{(\lambda p)^n}{n!}=\frac{e^{-\lambda p}(\lambda p)^n}{n!}\sim Poi(\lambda p). 
        \]
        $\lambda$ is the number of events in a specified time, and $p$ is the probability of counting each event. So, $\lambda p$ is the number of events that are counted. 
        \begin{enumerate}[(a)]
            \item $\Pr(N=1)=e^{-\lambda p}\frac{(\lambda p)^1}{1!}=e^{-0.2}\cdot 0.2=0.1637$. 
            \item $\Pr(N\geq 1)=1-\Pr(N=0)=1-e^{-0.2}=0.1813$ . 
            \item $\Pr(N\leq 1)=\Pr(N=0)+\Pr(N=1)=e^{-0.2}+e^{-0.2}\cdot 0.2=0.9825$.
        \end{enumerate}
    \end{solution}

    \begin{exercise*}[5]
        Two boys play basketball in the following way. They take turns shooting and stop when a basket is made. Player $A$ goes first and has probability $p_1$ of making a basket on any throw. Player $B$, who shoots second, has probability $p_2$ of making a basket. The outcomes of the successive trials are assumed to be independent. Find the frequency function for the total number of attempts. 
    \end{exercise*}

    \begin{solution}
        Let $N$ be the attempt times. Then, 
        $P(N=1)=p_1$, $P(N=2)=(1-p_1)p_2$, $\cdots$, $P(N=2n-1)=(1-p_1)^{n-1}(1-p_2)^{n-1}p_1$, $P(N=2n)=(1-p_1)^n(1-p_2)^{n-1}p_2$. 

        In general, for $k\in\mathbb{Z}^+$, 
        \[
            P(N=n)=\left\{ \begin{matrix}
                (1-p_1)^{(n-1)/2}(1-p_2)^{(n-1)/2}p_1, & n=2k+1, \\
                (1-p_1)^{n/2}(1-p_2)^{(n-2)/2}p_2, & n=2k. 
            \end{matrix} \right.
        \]
    \end{solution}

    

\end{document}